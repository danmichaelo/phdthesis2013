\section{Markup languages}
\label{sec:markuplanguages}

A \Term{markup language} is a formal language that is used to structure and
annotate text. A good introduction into theory and history of markup languages
until the late 1980s is given by \textcite{Coombs1987}. The origin of markup
languages is attributed to \Person[William]{Tunnicliffe} who in 1967 made a
presentation about the ``Separation of Information Content of Documents from
their Format'' and to \Person[Stanley]{Rice} \cite{Goldfarb1996}.  Tunnicliffe
proposed to use descriptive tags (called \Term[see{descriptive markup}]{generic
coding}) to mark up structural elements of publications instead of procedural
formatting codes. Their ideas later evolved into the \tacro{Generalized Markup
Language}{GML} that led to the \tacro{Standard Generalized Markup
Language}{SGML} \cite{Goldfarb1990}.  Both languages could be adopted to
different kinds of documents by using a \acro{DTD} (see~\ref{sec:xmlschemas}).
This shifted markup languages to data structuring languages, such as
\acro{SGML}'s successor \acro{XML} (see sec.~\ref{sec:xml}).  Eventually a
\emph{general} markup language is nothing but a kind of a readable \term{data
structuring language} with strong support of character strings and other
ordered lists, but little support of \term[data type]{data types} and
additional relationships,beside simple hierarchy.  The document structure
imposed by markup languages is usually described as an \tacro{ordered hierarchy
of content objects}{OHCO} \cite{DeRose1997,Renear1996}, although there are
attempts to extend markup languages with better support of multi-hierarchical
structures \cite{Pondorf2010}.

\subsection{General markup types}
\label{sec:markuptypes}
The possibility to separate content and presentation has always been a strong
argument for markup languages. With \Term{descriptive markup} in contrast to
\Term{presentational markup} authors can focus on the structure of documents
and mark what a text element \emph{is} instead of how it should look like
\cite{Coombs1987}. Elements of a descriptive markup language are declared on
the basis of their meaning --- the markup is also called
\emph{semantic}.\index{semantic!markup|see{descriptive markup}}\footnote{This
term is brought up with every new data structuring language: \acro{SGML},
\acro{XML}, \acro{RDF}\ldots} However the distinction between layout and
significance is less clear than it seems. Attributes like font size, color, and
style may both serve readability and imply a special meaning. The meaning
combined with a set of attributes can differ between texts or even within one
text. And markup does not have to be descriptive per se. Sometimes bold text is
just bold. Existing markup languages like \acro{HTML} therefore often mix
descriptive and presentational elements.\footnote{\acro{HTML} has descriptive
elements like \texttt{<title>}, \texttt{<h1>}, \texttt{<em>}, and
\texttt{<code>} but also purely presentational elements like \texttt{<i>} and
\texttt{<tt>}. With \acro{HTML5} the standard is shifted more to descriptive
markup with presentational capabilities separated in
\acro{CSS} \cite{Hickson2009}.}

In most applications markup is not presented directly to the user but only
visible by means of its effects. A heading may be marked up as such but the
reader can only differentiate based on layout properties (font size and
style, section numbers etc.). With \acro{WYSIWYG} editors, that provide
direct visual feedback while editing, the descriptive markup is even hidden to
the author. Although \term{word processor}s support some separation of structure
and layout via so called styles or templates, a `What You See Is What You
Mean' paradigm is difficult to achieve. Unless descriptive markup has some
useful impact that is directly visible to the author (for instance automatic
table of contents based on heading markup), one can unlikely motivate its usage.
The concrete meaning of markup not only depends on the concrete markup language
but also on its actual use and implications, and authors easily misuse markup
elements for something other than their intended purpose. Eventually the
markup's meaning is \emph{only} its actual usage if we follow
\person[Ludwig]{Wittgenstein}. For a computer program, on the other hand,
markup can only have an effect in form of \Term{procedural markup} that
indicates what to do with a given piece of text. Although procedural markup
is mostly hidden to the user (descriptive and presentational markup is mapped
to it internally), it affects the usage of markup as well. An author of an
\acro{HTML} document may use specific tags not (only) to indicate the
meaning of some text or to shape its layout, but also to accomplish some
behaviour in search engines, browsers and other programs that process the
document.

Beside this three types of document markup there is a fourth class that
gained importance with \acro{SGML} and document types: a \Term{prescriptive
markup language} imposes a set of rules, which all matching documents
must follow. A descriptive markup language in contrast describes document
structures \emph{a posteriori} rather than prescribing rules over them
\cite{Quin1996,Renear2000}. The difference between the two lies in the
purpose of \term{validation}, and applies not to single pieces of markup
but to markup languages as a whole \cite{Piez2001}. Prescriptive markup implies
a strict validation that checks whether a document instance fits to
a given document or not. Descriptive markup, in contrast, only provides
loose validation but many degrees of freedom to describe documents
of possibly unknown structure.

\subsection{Text markup languages}
We will now look at some popular concrete markup languages for textual
documents:

\begin{description}

\item[\acro{HTML}, \acro{TEI}, and \term{DocBook}] are the major text
markup languages. All were based on \acro{SGML} and migrated to
\acro{XML} around the turn of the millennium. Each language covers a
specific document type and mixes descriptive markup with some presentational
markup \cite{Hickson2009,Burnard2007,Walsh2010}.

\item[\TeX] %\index{\TeX}\index{\LaTeX} % TODO
is a powerfull programming language
for typesetting, created by \Person[Donald]{Knuth} in 1978 \cite{Knuth1984}.
It is popular especially in mathematics and related disciplines because of
its strong support of formulae. Its popular extension \LaTeX\ defines a rich
variety of macros; however there is no clear separation between descriptive
markup, presentational typesetting, and programming \cite{Lamport1994}.

\item[\term{troff}] is a presentational markup language that evolved from the
very first text formatting program \acro{RUNOFF} (1964). Several macro
packages add descriptive markup for different document types
\cite{Dougherty1987}. Troff has mainly been replaced by \LaTeX\ and DocBook.

\item[\tacro{OpenDocument Format}{ODF}] is an \acro{XML} based file format for
several kinds of office documents. It includes detailed capabilities to encode
text formatting, partly based on \acro{CSS} and \acro{XSL-FO}
\cite{OpenDocument2012}. Its extent and complexity limits the use a markup
language to a subset of \acro{ODF}, but to lesser degree than the rival
``Office Open XML'' file format by Microsoft.

\item[\term{Lightweight markup languages}] have a simple syntax, designed to
be easy for a human to read and to enter with a plain text editor. Popular
lightweight markup languages include \term{reStructuredText}, \term{Markdown},
\term{Textile}, \acro{POD}, and several Wiki syntaxes. Examples of lightweight
markup languages are shown in table~\ref{tab:markuplanguages}, different
markup languages cover same or overlapping document models by different syntax.

\item[\tacro{Rich Text Format}{RTF}] is a file format developed by
\term{Microsoft} in 1987 to exchange text documents. It is not meant to
directly be created by people but has a readable syntax and some concepts
similar to \TeX\ \cite{Burke2003}.

\item[\tacro{Extensible Stylesheet Language -- Formatting Objects}{XSL-FO}]
is an \acro{XML} based markup language, that is used to describe the layout
of documents, based on an area model \cite[par. 6 and 4]{Berglund2006}. The 
area model defines pages, blocks, and lines, partly derived from the \acro{CSS}
formatting model.\footnote{The \acro{XSL} family of standards and its
relation to \acro{CSS} and \acro{XML} is somehow confusing. In short
the \Tacro{Extensible Stylesheet Language}{XSL} consists of \emph{i}) a
query language for addressing parts of an \acro{XML} document (\acro{XPath}),
\emph{ii}) a programming language for transforming \acro{XML} documents
(\acro{XSLT}) that uses \acro{XPath}, \emph{iii}) an area model that defines
layout properties, and \emph{iv}) an \acro{XML} syntax to specify documents
based in the are model (\acro{XSL-FO}). \acro{XSL} builds on concepts of
\acro{CSS} and the \Tacro{Document Style Semantics and Specification
Language}{DSSSL} but not as its subset or superset.\label{fn:xsl}}

\item[\tacro{Cascading Style Sheets}{CSS}] is a stylesheet language
to describe the presentation of elements in document markup languages
\cite{Bos2009}. Introduced first by \Person[H\r{a}kon Wium]{Lie} in 1994 for
\acro{HTML}, it has since been adopted for several other document types. In
contrast to the other languages listed above, \acro{CSS} does not markup text
but describe the layout properties of elements in marked up documents.
\end{description}

\begin{sidewaystable}
    \begin{tabularx}{\textwidth}{|l|X|X|X|X|}
    \hline
    \textbf{language} & \textbf{bold face} & \textbf{italic face} &
    \textbf{monospace} & \textbf{sub-/superscript} \\
    \hline \term{DocBook}
    & \verb|<emphasis| \verb|role='strong'>|
    & \verb|<emphasis>|, \verb|<firstterm>|
    & \verb|<code>|, \verb|<varname>|
    & \verb|<subscript>|,  \verb|<superscript>|
    \\
    \hline \acro{HTML}
    & \verb|<b>| 
    & \verb|<t>|
    & \verb|<tt>|, \verb|<code>|
    & \verb|<sub>|, \verb|<sup>| 
    \\
    \hline \acro{TEI}
    & \verb|<hi rend="bold">|
    & \verb|rend="italics"|
    & \verb|rend="typewriter"|
    & \verb|rend="subscript"|, \verb|rend="superscript"| \\
    \hline \TeX\sindex[acronym]{tex@\TeX}
    & \verb|\textbf{text}| 
    & \verb|\textit{text}|, \verb|\emph{text}|
    & \verb|\texttt{text}|, \verb|\verb#text#|
    & \verb|^{text}|, \verb|_{text}| \\
    \hline \acro{RTF}
    & \verb|{\b text}| 
    & \verb|{\i text}|
    & \verb|{\fmodern text}| 
    & \verb|{\sub text}|, \verb|{\sup text}| \\
    \hline \term{MediaWiki}
    & \verb|'''text'''|
    & \verb|''text''|
    & \verb|<tt>|, \verb|<code>|
    & \verb|<sub>|, \verb|<sup>| \\
    \hline \term{Markdown} 
    & \verb|**text**|, \verb|__text__|
    & \verb|*text*|, \verb|_text_|
    & \verb|``text``|
    & \verb|<sub>|, \verb|<sup>| \\
    \hline \term{Textile}
    & \verb|*text*|
    & \verb|_text_|
    & \verb|@text@|
    & \verb|~text~|, \verb|^text^| \\
    \hline \term{reStructuredText}
    & \verb|**text**| 
    & \verb|*text*|
    & \verb|``text``| 
    & \verb|:sub:`text`|, \verb|:sup:`text`| \\
    \hline \acro{POD}
    & \verb|B<text>|
    & \verb|I<text>|
    & \verb|C<text>| 
    & -- \\
    \hline GNU \term{troff} (man)
    & \verb|.fam B text .fam|, \verb|\fBtext\fP|
    & \verb|.fam I text .fam|, \verb|\fItext\fP|
    & \verb|.fam C text .fam|, \verb|\fCtext\fP| 
    & \verb|^text^|, \verb|~text~| \\
    \hline
    \end{tabularx}
  \caption{Comparision of markup languages}
  \label{tab:markuplanguages}
\end{sidewaystable}


% TODO: change font and use \acro
\begin{figure}
\centering
\begin{tikzpicture}[orm,ml/.style={fill=white,inner sep=3pt,align=center}]
 \draw[|-|] (1.2,4.5) to node[ml] {\acro{HTML}} +(6*1.2,0);
 \draw[|-|] (1.2,3.7) to node[ml] {\acro{TEI}, \term{DocBook}} +(3*1.2,0);
 \draw[|-|] (2.4,2.9) to node[ml] {\TeX,~\LaTeX,~troff} +(5*1.2,0);
 \draw[|-|] (0.0,2.0) to node[ml] {\acro{ODF}} +(2*1.2,0);
 \draw[ -|] (2.4,2.0) to node[ml] {Lightweight\\markup languages} +(4*1.2,0);
 \draw[ -|] (7.2,2.0) to node[ml] {\acro{RTF},\\\acro{XSL-FO}} +(2*1.2,0);
 \draw[|-|] (0.0,1.1) to node[ml] {\acro{CSS}} +(8*1.2,0);
 \node[align=center] (s) at (0,0) {\ormbf storage\\\ormbf format};
 \node[align=center] (e) at (4.8,0) {\ormbf edit\\\ormbf format};
 \node[align=center] (o) at (9.6,0) {\ormbf output\\\ormbf format};
 \draw[thick,<->] (s) to node[above] {load/save} (e);
 \draw[thick, ->] (e) to node[above] {export} (o);
 \coordinate[below=0.3 of e.south] (e2);
 \draw[thick] (s.south east) to[out=-20,in=180] (e2);
 \draw[thick,->] (e2) to[out=0,in=180] (o);
\end{tikzpicture}
\caption{Main applications of some markup languages}
\label{fig:mlapps}
\end{figure}

Each language implies or defines a \Term{document model} with entities such as
pages, paragraphs, tables, lists, lines characters etc. \term[character
encoding]{Character encodings} as described in section~\ref{sec:characters}
provide the fundament of such models. Markup languages and document models are
shaped by the focus of their application. We can divide \emph{i}) storage
formats, which are mainly used to store and exchange documents, \emph{ii})
output formats, which procedurally or descriptively trigger a display device,
and \emph{iii}) edit formats, which are used to create, analyze, and modify
documents (figure~\ref{fig:mlapps}). Edit formats require the markup language
to be expressed in a specific \term[syntax!of markup languages]{syntax};
basically most markup languages are foremost defined by a markup syntax and
`markup' is often used as a synonym for a markup language's syntax. Syntax,
however, should not be confused with the document model.
Table~\ref{tab:markuplanguages} list four common concepts and their expression
in syntax of different markup languages. As shown in
table~\ref{tab:titleinmarkup}, the same document (here it is just a title) can
be expressed differently in different form.  Following the radical position of
\textcite{Renear2009}, either the forms do not represent the same document or
the document does not change if we transform one form to the other. But what is
the document, if we only have markup language in form of syntax?

\begin{table}
\begin{minipage}{\linewidth} 
\renewcommand{\footnoterule}{}
\begin{tabular}{rl}
\acro{HTML} & \texttt{Radiative \&beta;-decay in <sup>141</sup>Ce} \\
\LaTeX & \verb|Radiative \textbeta-decay in ^{141}Ce| \\
Textile & \verb|Radiative &beta;-decay in [^141^]Ce| \\
non-standard markup & \texttt{Radiative \textbeta-decay in \^{}1\^{}4\^{}1Ce} \\
plain Unicode\footnote{This special title can be expressed in plain
Unicode because the subscript digits 1 and 4 happen to be defined as codepoint
\U{00B9} and \U{2074} and the upright beta is defined as codepoint \U{03B2}.} 
 & \texttt{Radiative \textbeta-decay in \textsuperscript{141}Ce} \\
simplified transcription &\texttt{Radiative beta-decay in 141Ce} \\
\end{tabular}
\caption{An article title marked up in syntax of several markup languages}
\label{tab:titleinmarkup}
\end{minipage}
\end{table}

Defining a markup language by its syntax introduces some problems with escaping,
concatenating, nesting, and alternatives. First, characters acting as syntax
elements cannot directly be used in text but must be escaped or forbidden.
% most basic example: characters as null-terminated string
For instance in \acro{HTML} the less-than sign `$<$' (\U{3C}) must be escaped as
`\verb|&lt;|', `\verb|&#60;|', or `\verb|&#x3C;|'. Second, syntax elements often
cannot arbitrarily be nested and/or concatenated. For instance in Textile
`\verb|**__B__**|' encodes a bold, italic letter `B' but if it is surrounded by
other letters, it must be put in square brackets as `\verb|A[**__B__**]C|'. 
Alternatives occur especially if there is no isomorphism between syntax and 
document model (\ref{fig:mlalt}). As each markup element has a special
meaning you can rarely derive general rules for all parts of the syntax.

\begin{figure}
\begin{tabular}{rcl|l}
\verb|<b><i>A</i></b>| &$\cong$& \verb|<i><b>A</b></i>| 
  & `\textbf{\textit{A}}' \\
\verb|<i><i>A</i></i>| &$\cong$& \verb|<i>A</i>| & `\textit{A}' \\
\verb|<i>A</i><i>B</i>| &$\cong$& \verb|<i>AB</i>| & `\textit{AB}' \\
\verb|<sup>a</sup><sup>b</sup>| &$\cong$& \verb|<sup>ab</sup>| 
  & $^\text{ab}$ \\
\verb|a<sub>b<sup>c</sup></sub>|&$\ncong$& \verb|a<sup>b<sub>c</sub></sup>| 
  & $\text{a}^{\text{b}_\text{c}}$ $\neq$ $\text{a}_{\text{b}^\text{c}}$  \\
\verb|a<sup><sup>b</sup></sup>| &$\ncong$& \verb|a<sup>b</sup>| 
  & $a^{^\text{b}}$ $\neq$ $a^\text{b}$ \\
\end{tabular}
\caption{Ambiguities and structural element differences in markup syntax}
\label{fig:mlalt}
\end{figure}

In most markup languages it is not even obvious which parts of a markup syntax
are nested (tree model) and which are concatenated (sequence/event model). The
interpretation of an element can depend on its position in a tree of elements,
on its position in a sequence of elements, and/or on its position relative to
other elements (see table~\ref{tab:mleventtree}, elements are shown underlined
for better readability). In practise all markup language syntaxes have some
tree-based parts and some event-based parts.

\begin{table}
\begin{tabular}{rcl}
\textbf{syntax} & \textbf{model} & \textbf{document instance} \\
MediaWiki & events &
  \texttt{\uline{\tq\tq\tq} bold~~\uline{\tq\tq} bold-italic \uline{\tq\tq\tq}
~~~~~~~italic \uline{\tq\tq}} \\
invalid \acro{HTML} & events & \texttt{\uline{<b>} bold
  \uline{<i>} bold-italic \uline{</b>}~~~~~~~italic \uline{</i>}} \\
\acro{HTML} & tree & 
  \texttt{\uline{<b>} bold
  \uline{<i>} bold-italic \uline{</i></b><i>}italic \uline{</i>}} \\
\end{tabular}
\caption{Event model vs. tree model in markup languages}
\label{tab:mleventtree}
\end{table}

A widespread example of markup for metadata exists in citation styles and 
forms of heading, such as \acro{ISBD}. Character strings that result from
library cataloging rules, have mainly used for printing catalog cards and
to support other forms of retrieval. Although often described as 
\term{record}, they can better be viewed as a special kind of document.
As shown by \textcite{Thomale2010} for \acro{MARC}, the origin as markup
has also influenced bibliographic data formats.\footnote{The treatment
of existing markup in cataloging, which would result in markup in markup,
has not been analyzed deeply yet. Most cataloging rules do not preserve 
font styles, emphasis, or even non-latin characters but make use of
transcriptions and comments.}

% Rules: (AACR2, RAK, RDA\ldots): Auszeichnungen werden in der Regel nicht
% übernommen oder durch einen
% Ggf. zusätzlichen Kommentar in einer Fußnote angegeben
% RAK-WB §§ 116, 206, 207
% RAK-WB und AACR haben weniger mit Daten und Informationen zu tun als mit dem
% Aussehen und Sortieren von Katalogkarten
% nicht lateinische 
% Most cataloging rules do not preserve markup like font styles, emphasis and
% subscript/superscript, or even non latin characters

% TODO: Example: Zero-terminated string vs. fixed length strings
% most basic example: null-terminated string (subset of Unicode-String!)

Despite different applications and syntaxes, among markup languages you can
find sets of shared concepts. Especially lightweight markup languages build
on a common document model and differ mainly in details of syntax.
Table~\ref{tab:markuplanguages} compares selected inline markup (bold,
italics, monospace, subscript and superscript) in 11 different markup
languages. Without loss of generality, we can limit analysis to inline
formatting because that may more often occurr in other fields of metadata
(see table~\ref{tab:titleinmarkup}). An integration of common parts will 
lead to a document model like the area model of \acro{XSL-FO} or the 
formatting model of \acro{CSS}. Much work of such integration has been done
in the \Term{pandoc} document converter software written by
\Person[John]{MacFarlane}.\footnote{\url{http://johnmacfarlane.net/pandoc/}}
Pandoc implements a set of parsers and a set of writers for various textual
markup languages. They connect through a representation of parsed documents
in a common document model, implemented as data structure in Haskell.

\subsectionexample{The CSL text markup model}
The \Tacro{Citation Style Language}{CSL} is an \acro{XML} based formal 
language for describing the formatting of in-text citations, notes and 
bibliographies \cite{Zelle2012}. \acro{CSL} does not require fixed input and
output formats. Citations and bibliographic records must only conform to a
common model and formatted bibliographies can be produced in different markup 
languages such as  \acro{HTML}/\acro{CSS}, \acro{RTF}, and \TeX. Parts of 
the input may be  formatted by additional markup as shown in
table~\ref{tab:titleinmarkup}. \acro{CSL} processors must be aware of this
markup and transform it into a common text markup model. The markup includes 
italics, bold, and small caps as flags; nestable subscript and superscript;
and quotations in up to two levels (pairwise quotation marks are treated 
as markup that can be modified by specific \acro{CSL} styles). The implied
text markup model is shown in figure \ref{fig:cslmarkupmodel}. The model
abstracts from alternatives as listed in figure~\ref{fig:mlalt}.

% TODO: missing abbr
% TODO: nodecor of quotation marks?!
% TODO: Unicode NFC (composed grapheme)
% TODO: superscript glyphs in Unicode

\begin{figure}[h]
\centering
\begin{tikzpicture}[orm]

\entity (c) {Character};

\unary[left=of c,label=left:bold,yshift=5mm](bold){} edge[plays] (c);
\unary[left=of c,label=left:italic,yshift=0mm](italic){} edge[plays] (c);
\unary[left=of c,label=left:small caps,yshift=-5mm](sc){} edge[plays] (c);

\draw[decoration={brace},decorate]
(sc.south west)++(-18mm,0) -- +(0,+13mm) node[anchor=east,pos=0.5] {\ormind{2}};

\binary[left=of c,xshift=2mm,label=left:\ormleft{protected},yshift=-12mm](p){} edge[plays] (c);

\value[below=of p.one south,label=below:\ormvalues{case,decor}]{Protect} edge[plays] (p.one south);

\binary[right=of c,yshift=5mm,unique=2,label=above:has] (r) {} edge[required by] (c);
\value[right=of r] {GraphemeCluster} edge[plays] (r);

\binary[right=of c,yshift=-5mm,unique=2,label=above:has] (s) {} edge (c);
\value[right=of s,label=right:\ormvalues{1,2}] {QuotationLevel} edge[plays] (s);

\ternary[above=12mm of c,unique=1-3,skip unique=2,unique=1-2:2,
  label=above:contains\ldots at\ormind{1}](sc){} edge[plays] (c);
\entity[left=5mm of sc]{String~!} edge[plays] (sc);
\value[right=5mm of sc,label=right:\ormvalues{1,2,\ldots}]{Position} edge[plays] (sc);

\binary[below=1 of c,unique,unique=2,xshift=2mm] (lc) {};
\plays (c) to (lc.one north);
\node[above=0.1 of lc.two north] {has};
\entity[right=of lc] (L) {Level} edge[plays] (lc);

\value[right=1.6 of L](S){Shift} edge[required by]
  node[roles,unique=2,label=below:is]{} (L);

\constraint[right=0 of S,text width=1.8cm]
  {\textbraceleft supscript, superscript\textbraceright};

\binary[below=of L,unique=1,unique=2,label=below:\ormleft{includes}] (n) {};
\plays (L) to (n.one north);
\plays (L) to (n.two north);

\limits (n.west) -| node[constraint=xor]{} (lc.two south);

\node[rule=1,anchor=north west] at (-3,-3.2) 
{character positions of a string must have consecutive positions.};
\node[rule=2,anchor=north west] at (-3,-3.7) 
{only if supported by the character's grapheme cluster.};

\end{tikzpicture}
\caption{Text markup model of \acrostyle{CSL}}
\label{fig:cslmarkupmodel}
\end{figure}

% TODO: describe/mention "GraphemeCluster"

