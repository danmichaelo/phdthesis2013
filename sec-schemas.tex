\section{Schema languages}
\label{sec:schemas}

\begin{quotation}% 
Und er kommt zu dem Ergebnis: / Nur ein Traum war das Erlebnis.\\
Weil, so schließt er messerscharf, / 
Nicht sein \emph{kann}, was nicht sein \emph{darf}. \\
\quotationsource\Person[Christian]{Morgenstern}:
 \textit{Die unmögliche Tatsache} (1910) 
\end{quotation}

\noindent \Term[Schema language]{Schema languages}, \Tacro{data definition
languages}{DDL}, or \Term{data description languages} are primarily used to
further restrict existing data structuring systems, such as \term{database
model}s, \term{data structuring language}s, or \term{markup language}s. The
process of declaring a  schema is often called \Term{data definition} and the
result is a \term{data format} or \term[logical schema]{(logical) data schema}.
In \term{data modeling}, these schemas are located at the data realm
(figure~\ref{fig:datamodeling} at page \pageref{fig:datamodeling}), although
some schemas (e.g. \acro{RDF} schemas) also span to conceptual realm if used as
conceptual modeling languages (see section~\ref{sec:modelangs}). 

The purpose of a schema can be both prescriptive specification of documents to
be created and validation of existing documents. The expression of schemas in
dedicated schema language better allows for sharing and analysis of schemas,
independent of particular applications. Different schemas expressed in the same
schema language can be used by a \Term{validator} or \term{parser}
(figure~\ref{fig:schemavalidator}). Validators ensure common data structures
based on shared schemas.  Without such schema, data from one application may be
rejected or lead to unexpected results in the other. The validator acts as
interpreter that processes a schema in its schema language as program to
transform documents as input to document analysis as output.

Each schema language has an inherent model of the data that schemas further
restrict.  In the case of regular expressions and Backus-Naur Form
(part~\ref{sec:bnf}) this model is a simple sequence of characters. \acro{XML}
schema languages restrict \acro{XML} documents (part~\ref{sec:xmlschema}),
\acro{RDF} schema languages restrict \acro{RDF} graphs
(part~\ref{sec:rdfschemas}), and \acro{SQL} schemas restrict databases.  Other
methods used as schema languages (programming languages, forms, and the data
format description language) are summarized briefly at the end of this section
(part~\ref{sec:otherschemas}).

\begin{figure}[h]
\centering
\begin{tikzpicture}
\node[draw] (v) {Validator};
\node[left=20mm of v] (d) {Document};
\node[right=16mm of v] (a) {Document analysis};
\node[draw,below=10mm of v] (s) {Schema};
\draw[->] (s) to node[anchor=east] {control} (v);
\draw[->] (d) to (v);
\draw[->] (v) to (a);
\draw[out=180, in=-90,->] (s) to node[anchor=north,yshift=-1mm] {boilerplate} (d);
\draw [->] ($(s.east)-(0.25,0.25)$) arc(-145:155:.5);
\node[right=7mm of s] {extension/restriction};
\end{tikzpicture}
\caption{Schema languages allow to express schemas for multiple applications}
\label{fig:schemavalidator}
\end{figure}

\subsection{Regular Expressions and Backus-Naur Form}
\label{sec:bnf}

\tacro{Regular expressions}{RE} and \tacro{Backus-Naur Form}{BNF} are popular
schema languages for sequences of symbols. Various similar notations for
\acro{RE} and \acro{BNF} exist \cite{Goyvaerts2011} and elements of both are
used within other schema languages. \acro{RE} and \acro{BNF} both originate
from the mathematical definition of \term{regular language}s (\acro{REG}) and
\term{context-free language}s (\term{CFL}), which can be expressed as
\term{formal grammar}s by \acro{RE} and \acro{BNF}, respectively (see
section~\ref{sec:formallanguages}).  Other formal languages, such as visibly
pushdown languages or boolean grammars, rarely have their own schema languages.
Instead, they can be used as restricted or extended variants of \acro{REG} and
\acro{BNF}.

\Term{Regular expressions} are commonly used to define search patterns that
match a set of character strings. In the mid-1960s \Person[Ken]{Thompson}
introduced \acro{RE} by inventing a clever implementation for the text editor
\term{QED} for the \acro{CTSS} operating system. \acro{RE} were then
popularized by the command line tool `\term{grep}' that became a standard
feature of \term{Unix}. Descriptions and references of regular expressions for
practical applications are given by \textcite{Friedl2006}, the underlying
theory and implementations are described by \textcite{Cox2007}.  In short, a
regular expression is a sequence of character symbols that expresses a
\term[regular language]{regular (Chomsky Type 3) language}. When applied as
search pattern, simple character sequences match themselves, except special
\term{metacharacters} (\texttt{|}, \texttt{*}, \texttt{?}, \texttt{+},
\texttt{(}, \texttt{)},\ldots). Metacharacters hierarchically group expressions
into optional, repeatable, and alternative parts.  The vertical bar denotes
alternatives, a question mark denotes an option, star and plus denote zero or
more respectively one or more repetitions, and parentheses group expressions.
To match a metacharacter, it must be escaped with a  backslash, which also acts
as metacharacter. Several extensions of the traditional \acro{RE} notation have
added more metacharacters as abbreviation, for instance the dot to match any
character or `\verb|\d|' to match any digit.  Another common extension, that
goes beyond \term{regular language}s, are \term{back-references}: for instance
`\verb|(a+)(b+)\1\2|' matches the language of words with $i>0$ occurrences of
\verb|a|, followed by $j>0$ occurrences of \verb|b|, followed by the same
sequence of \verb|a|s and \verb|b|s again \cite{Carle2009}. The most popular
notation for regular expressions is \tacro{Perl Compatible Regular
Expressions}{PCRE}. Some extensions of \acro{PCRE} allow to express
context-free grammars and more complex types of \term{formal language}s.

\Tacro{Backus-Naur Form}{BNF} was introduced by \Person[John]{Backus} and
extended by \Person[Pete]{Naur} to define the syntax of the \term{Algol 60}
programming language. The name was proposed by \Person[Donald]{Knuth} in a
letter to the Communications of the ACM \cite{Knuth1964}. \acro{BNF} since has
widely been adopted to specify context-free languages. A schema (called
\term[formal grammar]{grammar}) consists of a set of named \Term[rule (BNF
syntax)]{rules} that map \Term{non-terminal symbols} to one or more alternative
sequences of symbols, that can be non-terminal symbols or \Term{terminal
symbol}s.  A language in \acro{BNF} is defined over all sequences of
terminal symbols (usually all character strings) as the set of all sequences
that can be constructed by the grammar's \term{rewriting system} (see 
page~\pageref{p:rewritingsystem}). There are
many notations how to write down \acro{BNF} grammars, for instance
\person[Niklaus]{Wirth} syntax \cite{Wirth1977}, \Term{Extended Backus-Naur
Form} \cite{ISO14977}, \Tacro{Augmented Backus-Naur Form}{ABNF}
\cite{Crocker2008}, and \acrostyle{W3C-BNF} \cite[sec. 6]{Bray1998}. Some
variants introduce new grammar elements like grouping, options, repetitions,
numeric factors, exceptions, etc. However, all variants can express the same
set of \term{formal grammar}s (Type-2 in the \term{Chomsky hierarchy}) and
grammars in one variant can be converted to any other notation.  Additional
extensions to specific formal languages are often added by textual comments
that explain language constraints, such as the uniqueness requirement of
attribute names in \acro{XML}.  A summary of the syntax variant that is used
throughout this thesis is given in table~\ref{tab:bnfsummary}, some other syntax
variants are shown in figure~\ref{fig:bnfsyntax}. Its
non-\acro{BNF} extensions difference and conjunction
also allow for boolean grammars.\footnote{General context free languages are 
not closed under difference, but the difference between a language in 
\acro{CFL} and a language in \acro{REG} is also in \acro{CFL}, so one can
use regular expressions as subtrahend in \acro{BNF}.} 
There are also graphical notations for
\acro{BNF}, known as \Term{syntax diagram} or \Termstyle{railroad diagram}. A
typical variant is depicted in figure~\ref{fig:bnfsyntax}.\footnote{Repeating
groups and boolean extensions are more difficult to picture.}
% See \textcite{Voss2011a} for a short discussion about railroad diagrams for boolean grammars.}
A popular example of the use of syntax diagrams is the specification
of \acro{JSON} \cite{Crockford2002}.  The visual form of its publication, that
can easily be followed by readers, was one reason for the quick success of
\acro{JSON}.

\begin{table}
\begin{tabular}{|c|l|c|l|}
\hline
\multicolumn{2}{|c|}{non-terminal symbols} &
\multicolumn{2}{c|}{terminal symbols} \\
\hline

\bnf{s}\texttt{ =} & defines a rule for symbol \bnf{s} &
\bnfstring{"\ldots"}  & sequence of characters \\

\bnf{a b} & sequence of symbols &
\bnfstring{\#xX} & character with Unicode codepoint \bnfstring{X} \\

\bnf{a | b} & alternative symbols & 
\bnfstring{[\ldots]}  & regular expression character class \\

%\bnfstring{/\ldots/} & regular expression \\

\cline{3-4}
\bnf{a?} & optional symbol  \\
\bnf{a+} & repeatable symbol \\ 
\bnf{a*} & optional repeatable symbol \\

\cline{3-4}
\bnf{a:}\sffamily{n}      & repeat {\sffamily n} times
  & \multicolumn{2}{c|}{extended non-terminal symbols} \\
\cline{3-4}
\bnf{a:}\sffamily{n-m}    & repeat {\sffamily n} to {\sffamily m} times 
  & \bnf{a - b} & difference \\
\bnf{a:}\sffamily{n-}     & repeat at least {\sffamily n} times 
  & \bnf{a \& b} & conjunction \\

\hline
\end{tabular}
\caption{Summary of \acrostyle{BNF} syntax with extensions}
\label{tab:bnfsummary}
\end{table}


\begin{figure}
\centering
\begin{tikzpicture}[
  line/.style={>=stealth',thick,draw},
  branch/.style={line,rounded corners,to path={|- (\tikztotarget)}},
  other branch/.style={line,rounded corners,to path={-| (\tikztotarget)}},
  nonterminal/.style={draw,rectangle,minimum size=6mm,
    thick,text height=1.5ex,text depth=.25ex},
  terminal/.style={nonterminal,rounded corners=3mm},
  code/.style={font=\ttfamily},
  lab/.style={right,text width=2cm,align=right}
]
\matrix(m)[matrix of nodes,nodes in empty cells,row sep=3mm]{
  & \node{\textbf{BNF}}; & \node{\textbf{diagram}};
\\
  \node[lab]{rule}; & \node[code]{name = }; &
  \draw[line,->] ++(-10mm,0) to 
    node[pos=0.3,yshift=2mm] {name} ++(20mm,0);
\\
  \node[lab]{terminal symbol}; & 
  \node[code]{"abc"}; & 
  \node[terminal](t){abc};
  \draw[<-,line] (t.west) to ++(-6mm,0);
  \draw[line] (t.east) to ++(5mm,0);
\\
  \node[lab]{non-terminal symbol}; & 
  \node[code](name){name}; 
  \node[code,below=0 of name]{<name>}; 
& 
  \node[nonterminal](n){name};
  \draw[<-,line] (n.west) to ++(-5mm,0);
  \draw[line] (n.east) to ++(4mm,0);
\\
  \node[lab]{sequence}; & 
\node[code](seq){a, b}; 
\node[code,below=0 of seq]{a b}; 
&  
  \coordinate (c);
  % some BNF variants use whitespace instead of comma
  \node[nonterminal,left=2mm of c](a){a};
  \draw[<-,line] (a.west) to ++(-3mm,0);
  \node[nonterminal,right=2mm of c](b){b};  
  \draw[line,->] (a) to (b);
  \draw[line] (b.east) to ++(2mm,0);
\\
  \node[lab]{alternatives}; & \node[code]{a | b}; & 
  \node[nonterminal](a){a};
  \node[nonterminal,below=2mm of a](b){b};
  \draw[<-,line] (a.west) to ++(-5mm,0) coordinate (c);
  \draw[line] (a.west) to ++(-7mm,0);
  \draw[line] (a.east) to ++(4mm,0) coordinate (d);
  \draw[line] (a.east) to ++(6mm,0);
  \draw (c) edge[branch,->] (b.west);
  \draw (b.east) edge[other branch,->] (d);
\\};

%\draw (m-2-1.north west) to (m-2-3.north east);
%\draw (m-1-1.south west) to (m-1-3.south east);
%\draw (m-1-2.north west) to (m-6-2.south west);
%\draw (m-1-2.north east) to (m-6-2.south east);

\draw[dotted,xshift=2mm] (m.north east) to (m.south east);

\matrix[row sep=3mm,right=0 of m.north east,matrix anchor=north west,
        matrix of nodes,nodes in empty cells]{
  & \node{\textbf{BNF}}; & \node{\textbf{diagram}};
\\
  \node[left]{optional}; &
  \node[code](a){(\ldots)?}; 
  \node[code,below=0 of a]{[\ldots]~}; 
&
  \draw[line] ++(-11mm,0) to ++(3mm,0) coordinate (x)
           to ++(7mm,0) coordinate (y) to ++(7mm,0) coordinate (z)
           to ++(3mm,0);
  \node[nonterminal,below=2mm of y](a){\ldots};
  \draw (x) edge[branch,->] (a.west);
  \draw (a.east) edge[other branch,->] (z);
\\
  \node[left]{repetition}; & 
    \node[code](a){\verb|{|\ldots\verb|}|~};
    \node[code,below=0 of a](a){(\ldots)*};
    \node[code,below=0 of a]{*(\ldots)}; 
  &
  \draw[line] ++(-11mm,0) to ++(3mm,0) coordinate(x) 
           to ++(7mm,0) coordinate (y) to ++(7mm,0) coordinate (z)
           to ++(3mm,0);
  \node[nonterminal,below=2mm of y](a){\ldots};
  \draw (x) edge[branch,<-] (a.west);
  \draw (a.east) edge[other branch,<-] (z);
\\
  \node[left]{}; & \node[code]{(\ldots)+}; & 
  \draw[line] ++(-11mm,0) to ++(3mm,0) coordinate (x)
           to ++(7mm,0) coordinate (y) to ++(7mm,0) coordinate (z)
           to ++(3mm,0);
  \node[nonterminal,fill=white,anchor=center] at (y) {\ldots};
  \coordinate[below=5mm of y](a);
  \draw (x) edge[branch,<-] (a);
  \draw (a) edge[other branch,-] (z);
\\
  \node[left]{}; & \node[code]{\ldots:n}; & 
  \draw[line] ++(-11mm,0) to ++(3mm,0) coordinate(x) 
           to ++(7mm,0) coordinate (y) to ++(7mm,0) coordinate (z)
           to ++(3mm,0);
  \node[nonterminal,below=2mm of y](a){\ldots};
  \draw (x) edge[branch,<-] (a.west);
  \draw (a.east) edge[other branch,<-] (z);
  \node at (8mm,-4mm) {n};
\\
};
\end{tikzpicture}
\caption{Backus-Naur Form and syntax diagram elements}
\label{fig:bnfsyntax}
\end{figure}

A general problem of Regular Expressions and Backus-Naur Form in practice is
the conflict between ease of understanding and performance of implementation. In
many cases, one formal language can be defined by multiple \term{formal
grammar}s of different structure. If the primary purpose of a schema is a
specification that can automatically be used to implement an efficient parser,
the grammar may be less readable for humans. To give a tiny example, a comment
in \acro{XML} must not contain the sequence `\texttt{--}`.  The specification
\cite{Bray1998,Bray2008} defines the syntax of a comment without explicitly
using this sequence as:

\begin{lstlisting}[language=BNF]
  comment   = "<!--" ((char - "-") | "-" (char - "-")))* "-->"
\end{lstlisting}

\noindent This makes an efficient parser, but is more difficult to grasp than
the following rule in natural language: a {\sf comment} starts with
`\verb|<!--|', ends with `\verb|-->|', and may contain any (possibly empty)
sequence of {\sf char}, that does not contain `\verb|--|' and does not end with
`\verb|-|'. One can translate this to \acro{BNF}, equivalent to the specification, 
for instance as:

\begin{lstlisting}[language=BNF]
  chars     = char+
  comment   = "<!--" (chars? - (chars? "--" chars? | chars? "-") "-->"
\end{lstlisting}


\subsection{XML schema languages}
\label{sec:xmlschemas}

Since its first publication, \acro{XML} (section~\ref{sec:xml}) contains
\tacro{Document Type Definition}{DTD} as native schema languag \cite{Bray1998}.
\acro{DTD} origins in \acro{SGML} and uses a specific syntax, similar to
\term{Backus-Naur Form}, so all \acro{XML} parsers must implement one parser
for \acro{DTD} and one parser for \acro{XML} element structure.  \acro{DTD} is
criticized for its syntax and because of limited expressibility.  For instance
it does not support specification of \acro{XML} namespaces without hard-coding
the namespace prefix \cite{Bray2009}. For this reason several additional schema
languages for \acro{XML} were created with \term{XML Schema} and \term{RELAX
NG} as most popular instances. Formal comparison of these languages, which are
all based on \term{formal grammar}s, are provided by \textcite{Murata2005} and
by \textcite{Stuehrenberg2010}. \term{Schematron} is another schema language
that more differs from \acro{DTD}, \acro{XSD}, and \term{RELAX NG}.  To some
degree, query languages such as \term{XPath} and \term{XML Query} can also be
used as (part of) \acro{XML} schemas. As described by
\textcite{Necasky2006,Necasky2008}, by \textcite{Sengupta2006}, and by
\textcite{Chen2010}, \acro{XML} schema languages only cover the logical level
of data modeling, but conceptual modeling languages for \acro{XML} are needed
to better describe semantics of \acro{XML} documents. In practice, however,
conceptual modeling for \acro{XML} is rarely applied \cite{Mohan2009} or no
explicit schema is used at all \cite{Wilde2008}. The following section covers
\acro{DTD}, \term{RELAX NG}, \term{Schematron}, and \term{XML Schema}, in this
order.

\subsubsection{Document Type Definitions}
\label{sec:dtd}

Element content in \acro{DTD} is specified in a special \acro{BNF} variant with
\acro{XML} element names as non-terminal symbols. Element declarations can
directly be mapped to \acro{BNF} grammar rules as shown in
example~\ref{ex:dtdbnf}.

\begin{example}[h]
\begin{lstlisting}[language=DTD]
<!ELEMENT bib ((author+|editor+), year?, title, keyword*)>
<!ATTLIST author given   CDATA #IMPLIED 
                 surname CDATA #REQUIRED>
<!ATTLIST editor given   CDATA #IMPLIED 
                 surname CDATA #REQUIRED>
\end{lstlisting}

A \xml{bib} element contains at least one \xml{author} or at least one
\xml{editor} element, followed by an optional \xml{year}, a \xml{title}
element, and an an optional list of one or more \xml{keyword} elements.  Both
\xml{author} and \xml{editor} must have a \xml{surname} attribute and may have
a \xml{given} attribute (the distinction between elements and attributes is not
expressible in pure \acro{BNF}).

\begin{lstlisting}[language=BNF]
  bib    = ( editor+ | author+ ) year? title keyword*
  author = ( given? surname )
  editor = ( given? surname )
\end{lstlisting}

\caption{\acrostyle{DTD} element declaration mapped to \acrostyle{BNF} rules}
\label{ex:dtdbnf}
\end{example}

In addition, the special symbol \dtd{EMPTY} is used for elements without child
elements (\lstinline[language=BNF]{emptytag} in the \acro{XML} grammar,
figure~\ref{fig:xmlbnf}), the symbol \dtd{ANY} can be replaced by any element
name, and the symbol \dtd{\#PCDATA} is used for arbitrary character content.
If \acro{XML} elements should be allowed next to character content, one must
define both in form of so called \Term{mixed content}, that is a rule of the
form \texttt{(\#PCDATA | $e_1$ | $\ldots$ | $e_n$)*} with $1 \le i \le n, n \ge
1$ where $e_i$ is an element name.  Processing instructions and comments are
not treated as part of the document. \acro{DTD} is the only \acro{XML} schema
language that supports declaration of \acro{XML} entities (with the keyword
\dtd{ENTITY}). The other languages treat a document as parsed model of a
document where entity references and CDATA sections have been replaced by
equivalent content (see figure~\ref{fig:xmlparsing} and page~\ref{p:xmlmodel}).
\acro{DTD} syntax for attribute declaration is different to element
declaration, as attributes have no order and cannot be repeated at one element.
Attributes can be marked as optional (keyword \dtd{\#IMPLIED}) or mandatory
(keyword \dtd{\#REQUIRED}) and there are some limited capabilities to restrict
possible attribute values, for instance to one value from a predefined list.
Character content of \acro{XML} element can not be restricted. One can specify
default values for attributes and disallow changing attribute values (keyword
\dtd{#FIXED}. Simple integrity condition are possible, by declaring some
attribute values as unique identifiers (keyword \dtd{ID}) and some attribute
values as pointers to these identifiers (keywords \dtd{IDREF} and
\dtd{IDREFS}). A proposed \acro{ISO} standard to further extend \acro{DTD} has
received little attention \cite{ISO19757-9}.

\subsubsection{RELAX NG}
\label{sec:relaxng}

The REgular LAnguage for XML Next Generation (\term{RELAX NG}) was developed as
merger of \tacro{Tree Regular Expressions}{TREX} and \tacro{Regular Language
description for XML}{RELAX}, both experimental \acro{XML} schema languages,
created in 2000/2001 \cite{VanDerVlist2003}. RELAX~NG was standardized at
\acro{OASIS} and published at \acro{ISO} in 2003 and 2008
\cite{ISO19757-2:2008}.  A RELAX~NG schema can be written in an \acro{XML}
syntax and in a more readable compact syntax, which is similar to
\term{Backus-Naur Form}. Both forms can be translated to its counterparts
without loss of information. The grammar from example~\ref{ex:dtdbnf} could be
written in RELAX NG Compact as shown in example~\ref{ex:rnc} (the
datatype \xml{xs:gYear} is added to the \bnf{year} element):

\begin{example}
\begin{lstlisting}[language=rnc]
element bib {
  ( element editor { person }+ | element author { person }+ ), 
  element year { xs:gYear }?,
  title, 
  keyword*
}
person = { 
  attribute given { text }?
  attribute surname { text }
}
\end{lstlisting}
\caption{Grammar rules in RELAX NG Compact syntax}
\label{ex:rnc}
\end{example}

\term{RELAX NG} unifies syntax for element and attribute declaration and it
provides functionality that goes beyond \acro{DTD}. In particular, it better
supports grouping, combining, and annotating grammar rules. It further adds
context-sensitive content models, namespaces, unordered content, and datatypes
for character content.  Datatypes are not defined in \term{RELAX NG} but only
referenced fro  other specifications. The most common datatypes are those from
\term{XML Schema}, for instance \xml{xs:integer} for character sequences that
represent integer values.  Identifier attributes (\dtd{ID}, \dtd{IDREF},
\dtd{IDREFS}) and default attribute values are not supported but an official
extension exists to add these features \cite{Clark2001}. Additional rules can
be embedded in \term{RELAX NG} schema by using other schema languages,
especially \term{Schematron}.

% -no support for XSDL's numeric occurrence constraints (minOccurs/maxOccurs)

\subsubsection{Schematron}

\term{Schematron} is a rule-based \acro{XML} schema language, expressed in
\acro{XML} \cite{VanDerVlist2007}. It was first proposed in 1999 by
\Person[Rick]{Jelliffe} and later standardized as Schematron~1.3 (2000),
Schematron~1.5 (2001) and Schematron~1.6 (2002). Schematron was published as
ISO~19757-3 (\citeyear{ISO19757-3}) and an extended version is being published
in 2011. The publication at \acro{ISO}  may be one reason why Schematron is
less known and less used than other schema languages described in this section.
In contrast to grammar-based languages, a Schematron schema does not specify
the whole tree structure of an \acro{XML} document, but it defines a set of
additional constraints.  Schematron constraints are expressed by two
\term{XPath} expressions: the \xml{context} defines which part of a document to
check and the \xml{test} specifies a condition that must be true for each
context element. A condition can trigger a message if it is met (with
\xml{report}) or if it is not met (with \xml{assert}). One or more conditions
with same \xml{context} are grouped in a \xml{rule} and one or more rules are
grouped in a \xml{pattern}, which may further be associated to phases.
Schematron also supports variables and abstract patterns as pattern templates,
as shown in example~\ref{ex:schematron}: this \term{Schematron}  schema can be
used to restrict author and editor names from the previous schemas (example
\ref{ex:dtdbnf} and \ref{ex:rnc}) by checking that given names are not empty
strings and surnames do not end with a dot, unless there is a given name.
Such interdependencies are difficult or not possible to express in
other schema languages.

\begin{example}
\begin{lstlisting}[language=XML]
<schema xmlns="http://www.ascc.net/xml/schematron">
  <title>Person name checks</title>
  <pattern abstract="true" id="person">
    <rule context="$p[@given]">
      <report test="0 < string-length(normalize-space(@given))"
      >person with valid given name</report>
    </rule>
    <rule context="$p[@surname and not(@given)]">
      <let name="s" value="normalize-space(@surname)"/>
      <let name="l" value="string-length($s)"/>
      <assert test="0 < $l and substring($s,$l) != '.'"
      >Surname without given name must not be abbreviated</assert>
    </rule>
  </pattern>
  <pattern is-a="person">
    <param name="p" value="author"/>
  </pattern>
  <pattern is-a="person">
    <param name="p" value="editor"/>
  </pattern>
</schema> 
\end{lstlisting}
\caption{Simple Schematron schema, restricting person names}
\label{ex:schematron}
\end{example}

Schematron conditions can also be documented with icons, diagnostic messages,
and other hints.  In contrast to other XML schema languages, Schematron is
primarily targeted to human users, as its core output of validation are
messages. This makes schematron together with its structured output format
\tacro{Schematron Validation Report Language}{SVRL} more a reporting language
for \term{business rule}s.  Schematron can report (violations of) several rules
as independent properties of an \acro{XML} document.  By this, it can be
applied more flexible and less strict than a global schema, which eventually
has a boolean output as either valid or not valid.

\subsubsection{XML Schema}
\label{sec:xmlschema}

\tacro{XML Schema}{XSD} is the most comprehensive and most used schema language
for \acro{XML} documents. It started as an extension of \acro{DTD} in
\acro{XML} syntax, influenced by several other schema languages that were
discussed as candidates for an official \acro{W3C} recommendation during 1998
and 2000.\footnote{In detail, \acro{XSD} was influenced at least by the
\tacro{Document Definition Markup Language}{DDML} and \tacro{Schema for
Object-Oriented XML}{SOX}, both proposed as \acro{W3C} note in 1999;
\tacro{Document Content Description}{DCD} a \acro{W3C} submission from 1998;
and \tacro{XML Data Reduced}{XDR}, proposed by Microsoft in 1998-2001.}
\acro{XSD}~1.0 became \acro{W3C} recommendation in 2001 with a second edition
in 2004. \acro{XSD}~1.1 is being published as extended successor in 2011. The
recommendation consists of one part on structure \cite{Thompson2004} and one
part on datatypes \cite{Biron2004}, the second of which also referred to in
other schema languages such as \acro{OWL} (page~\pageref{sec:rdfschemas}).

% first description, then evlautaion/comparision

An \term{XML Schema} mainly consists of element declarations, attribute
declarations, and type definitions. Declarations define which \acro{XML}
elements and attributes may or must exist in an \acro{XML} document, while type
definitions are referenced by element and attribute declarations. There are
also methods for grouping and reusing parts of a schema, annotations as human-
and machine-targeted documentation, and uniqueness and reference constraints
that go beyond the identifier attributes (\dtd{ID}, \dtd{IDREF}, \dtd{IDREFS})
in \acro{DTD}.  \acro{XSD}~1.1 adds assertions and boolean conditions (as
\term{XPath} expressions) that can be used for conditional constraints and type
assignment, similar to rules in \term{Schematron}.  Element and attribute
declarations are similar to declarations in \acro{DTD}, but element and
attribute content is always defined with a type.  There are simple types that
define constraints on strings for attribute values and text-only content, and
there are complex types that define constraints on attributes and child
elements. \acro{XSD} offers various syntax variants to express the same format,
but all are very verbose and hard to read and write compared to \acro{DTD} and
\term{RELAX NG} Compact.\footnote{You could use a compact syntax for
\acro{XML}, but proposals such as \textcite{Wilde2003} have not received much
adoption.} For this reason \acro{XSD} is mostly created with specialized
\acro{XSD} editors and similar software tools that provide graphical user
interfaces or automatic generation of schemas.  Example~\ref{ex:xsd} shows a
schema in \acro{XSD} that combines the rules from example~\ref{ex:dtdbnf} to
\ref{ex:schematron}.  Repeatability and optionality can be specified with a
minimal (\xml{minOccurs}) and maximal number (\xml{maxOccurs}) for elements or
with \xml{use="required"} for attributes.  Element content is defined with
\xml{xs:complexType} as ordered sequence (\xml{xs:sequence}), or as choice
(\xml{xs:choice}) of other components. One can group elements with
\xml{xs:any}, if the elements of a set can occur in any order, but with at most
one occurrence of each element --- this allows specification of unordered
content to a more limited degree than \term{RELAX NG}. Wildcards, similar but
more customizable then the \dtd{ANY} keyword in \acro{DTD}, are also supported.

\begin{example}
\begin{lstlisting}[language=XML]
<xs:schema xmlns:xs="http://www.w3.org/2001/XMLSchema">
  <xs:element name="bib">
    <xs:complexType>
      <xs:sequence>
        <xs:choice>
          <xs:element name="author" type="person" 
                       maxOccurs="unbounded"/>
          <xs:element name="editor" type="person" 
                       maxOccurs="unbounded"/>
        </xs:choice>
        <xs:element name="year" type="xs:gYear" minOccurs="0"/>
        <xs:element name="title" type="xs:string"/>
        <xs:element name="keyword" type="xs:string" 
                     minOccurs="0" maxOccurs="unbounded"/>
      </xs:sequence>
    </xs:complexType>
  </xs:element>
  <xs:complexType name="person">
    <xs:attribute name="given" type="myString"/>
    <xs:attribute name="surname" type="myString" use="required">
      <xs:alternative test="not(../@given)">
        <xs:simpleType name="noAbbrev">
          <xs:restriction base="myString">
            <xs:pattern value=".*[^\.]"/>
          </xs:restriction>
        </xs:simpleType>
      </xs:alternative>
    </xs:attribute>
  </xs:complexType>
  <xs:simpleType name="myString">
    <xs:restriction base="xs:string">
      <xs:minLength value="1"/>
      <xs:whiteSpace value="collapse"/>
    </xs:restriction>
  </xs:simpleType>
</xs:schema>
\end{lstlisting}
\caption{XML Schema, including features of \acrostyle{XSD}~1.1}
\label{ex:xsd}
\end{example}

Derived types can be defined based on existing types by constraints
(\xml{xs:restriction}) or by allowing additional content (\xml{xs:extension}).
Simple types can also be combined to union types (\xml{xs:union}) and list
types (\xml{xs:list}).  A predefined set of (mainly simple) types is provided
in the second part of \acro{XSD} specification \cite{Biron2004}.  The practical
relevance of these \acro{XSD} datatypes, also beyond pure applications of
\acro{XML}, demands a more detailed description.  As noted in the
recommendation, \acro{XSD} datatypes were influenced by data types in
programming and database languages and by ISO~11404 (\textcite{ISO11404}).
A datatype consists of a \Term{value space}, which is a set of abstract values;
a \Term{lexical space}, which is set of Unicode strings used to denote the
values; a surjective \Term{lexical mapping} function that maps from the lexical
space to the value space; and an \term{equality relation} for the value space
(figure~\ref{fig:xsdtypes}).  Furthermore datatypes may have an \term{order
relation} for the value space, a \Term{canonical mapping} that maps each value
from the value space to its preferred representation in the lexical space. In
addition, there is an identity relation that in most cases is the same as the
equality relation.\footnote{The distinction between equality and identity was
introduced in \acro{XSD}~1.1. For instance the lexical values \texttt{-0} and
\texttt{0} were both mapped to value zero for the datatype \xml{xs:float}, but
now they are mapped to non-identical but equal values, to better mirror
floating point number encoding (see page~\pageref{sec:numberencodings}).
Another example from \xml{xs:float} is the value of \texttt{NaN} (not a
number), which is identical but not equal to itself.} There are 50 build-in
datatypes, each referencable by an \acro{URI}.  Grouped by purpose one can
identify:

\begin{figure}
\centering
\begin{tikzpicture}[every line/.style=thick]
\node (l) {lexical space};
\node[right=4cm of l] (v) {value space};
\draw [->,out=20,in=160] (l) to node[yshift=3mm]{lexical mapping} (v);
\draw [dotted,<-,out=-15,in=-165] (l) to node[yshift=-3mm]{canonical mapping} (v);
\draw [dotted,<->] ($(v.east)-(0.05,0.18)$) arc(-140:140:.3);
\draw [<->] ($(v.east)-(0.25,0.25)$) arc(-145:145:.5);
\draw [<->] ($(v.east)-(0.4,0.30)$) arc(-150:150:.7);
\node[right=10mm of v,yshift=5mm] {identity};
\node[right=10mm of v] {equality};
\node[right=10mm of v,yshift=-5mm] {order};

\node[below=15mm of l.east,minimum width=12mm] (o) {\texttt{1}};
\node[below=19mm of l.east,minimum width=12mm] (t) {\texttt{true}};
\node[below=23mm of l.east,minimum width=12mm] (z) {\texttt{0}};
\node[below=27mm of l.east,minimum width=12mm] (f) {\texttt{false}};

\node[right=35mm of o,yshift=-2mm] (true) {true};
\node[right=35mm of z,yshift=-2mm] (false) {false};

\draw[->] (t) to (true);
\draw[->] (o) to (true);
\draw[->] (f) to (false);
\draw[->] (z) to (false);
\draw[<-,dotted,in=-170,out=-10] (t) to (true);
\draw[<-,dotted,in=-170,out=-10] (f) to (false);

\draw[<->,out=0,in=0] (true.east) to node(x){} (false.east);

\node[anchor=west,yshift=4.5mm] at(x) {not identical};
\node[anchor=west] at(x) {not equal};
\node[anchor=west,yshift=-4mm] at(x) {not ordered};

\draw[decoration={brace},decorate] (f.south west) to node[anchor=east,xshift=-1mm]
  {\xml{xs:boolean}} (o.north west);

\end{tikzpicture}
\caption{Lexical space and value space of \acrostyle{XSD} datatypes with boolean as example}
\label{fig:xsdtypes}
\end{figure}

\begin{itemize}
 \item a type for \term{Unicode} strings, limited to the characters allowed
       in \acro{XML} (\xml{xs:string}), and two types for strings with normalized
	   whitespace (\xml{xs:normalizedString} and \xml{xs:token}),
 \item various numeric types (see table~\ref{tab:xsdnumeric}),
 \item a boolean type (\xml{xs:boolean}, as shown in figure~\ref{fig:xsdtypes}),
 \item twelve different types for dates, times, and their parts or combinations,
 \item two binary types for encoded sequences of bytes (\xml{xs:base64Binary} and \xml{xs:hexBinary}),
 \item a type for \acro{URI}s (\xml{xs:anyURI}) and for tuples of namespace
       \acro{URI} and local name to represent \acro{XML} element names with 
	   namespaces (see example~\ref{ex:xmlns}),
 \item a type defined equivalent to \xml{xs:QName} but for \acro{XML} Notations
       (\xml{xs:NOTATION}),
 \item six special types derived from \xml{xs:token} to represent differnt kinds of
       identifiers (\xml{xs:language}, \xml{xs:Name}, \xml{xs:NCName}, \xml{xs:ID}, 
	   \xml{xs:IDREF}, \xml{xs:NMToken}),
 \item derived list types for three token types (\xml{xs:ENTITIES},
  \xml{xs:IDREFS}, \xml{xs:NMTOKENS}),
 \item the special types \xml{xs:anyType}, \xml{xs:anySimpleType}, and 
  \xml{xs:anyAtomicType} as base types of all other predefined types.
\end{itemize}


\begin{table}
  \centering
  \begin{tabular}{ll}
	\textbf{datatype}           & \textbf{value space} \\ 
	\hline
\xml{xs:anyType} & all types (simple and complex)\\
--\xml{xs:anySimpleType} & all simple types \\
-- --\xml{xs:anyAtomicType} & all simple types that are not lists or unions\\
-- -- --\xml{xs:decimal}     & $\set[x/10^y]{x \in \mathbb{Z}, y \in \mathbb{Z}_{\geq0}} \subset \mathbb{R}$ \\ 
-- -- -- --\xml{xs:integer}  & $\mathbb{Z} =$ \format{Int} \\
-- -- -- -- --\xml{xs:long}     & \formati{Int}{64} \\
-- -- -- -- -- --\xml{xs:int}                & \formati{Int}{32} \\
-- -- -- -- -- -- --\xml{xs:short}              & \formati{Int}{16} \\
-- -- -- -- -- -- -- --\xml{xs:byte}               & \formati{Int}{8} \\
-- -- -- -- --\xml{xs:nonNegativeInteger} & $\mathbb{Z}_{\geq0} =$ \format{UInt} \\
-- -- -- -- -- --\xml{xs:positiveInteger}    & $\mathbb{Z}_{>0}$ \\
-- -- -- -- -- --\xml{xs:unsignedLong}       & \formati{UInt}{64} \\
-- -- -- -- -- -- --\xml{xs:unsignedInt}        & \formati{UInt}{32} \\
-- -- -- -- -- -- -- --\xml{xs:unsignedShort}      & \formati{UInt}{16} \\
-- -- -- -- -- -- -- -- --\xml{xs:unsignedByte}       & \formati{UInt}{8} \\
-- -- -- -- --\xml{xs:nonPositiveInteger} & $\mathbb{Z}_{\leq0}$ \\
-- -- -- -- -- --\xml{xs:negativeInteger}    & $\mathbb{Z}_{<0}$ \\
-- -- --\xml{xs:float}              & 32 bit IEEE~754 floating point (except sNaN) \\
-- -- --\xml{xs:double}             & 64 bit IEEE~754 floating point (except sNaN) \\
  \end{tabular}
  \caption{Predefined numeric datatypes in \acrostyle{XML} Schema}
  \label{tab:xsdnumeric}
\end{table}

The recommendation groups its datatypes into primitive types and derived types.
Each derived type must have exactly one base type with \xml{xs:anyType} as
base type of all other simple types. A subset of the derivation tree is shown
in table~\ref{tab:xsdnumeric}. A derived type can specify restrictions in 
predefined \term{constraining facet}s that serve to normalize or
constrain its lexical space and/or its value space.  Possible facets include
length of the lexical representation, a regular expression pattern that all
lexical values must match, and lower/upper bounds on its value space for
ordered values.  Beside derivation one can define new types as lists or as
unions of existing types. It should be noted, that the lexical mapping of an union type may be 
no function, because the same lexical value can represent multiple values for
different types (for instance the string \texttt{"2"} and the number 2). 

The \acro{XSD} type system includes some more caveats: for instance primitive
datatypes are disjoint (the number 2 as \xml{xs:float} is incomparable to the
number 2 as \xml{xs:double} and as \xml{xs:decimal}). \xml{xs:float} and
\xml{xs:double} have no common base type although they both decode subsets of
$\mathbb{R}$ and there are unrelated types with the same value space
(\xml{xs:base64Binary} and \xml{xs:hexBinary}, which both map to the set of
finite-length sequences of binary bytes). Another problem may arise from the
fact that validation with a \acro{XSD} instances may modify a document by
adding default values and type information.  The set of \acro{XSD} datatypes
has also been criticized for being an arbitrary selection \cite{Ogbuji2002}.
For instance there are various types for dates and times but no types for other
dimensions such as length and geographic location.  Facets like length of the
lexical value neither consider Unicode normalization forms which can be
relevant to textual content.  Alternatives and extensions to \acro{XSD} data
types such as ``Extensible Datatypes'' and the \tacro{Character Repertoire
Description Language}{CREPDL}, both published as part of ISO~19757
(\citeyear{ISO19757:2008}), are mostly ignored in the \acro{XSD} community,
possibly because of the ``Not Invented Here syndrome'': unlike \term{RELAX NG}
and \term{Schematron}, \acro{XSD} is not designed by a small team, but by a
large working group that incorporates interests of database vendors and other
companies. As a result, the specification is large, complex, and difficult to
implement with all features. \acro{XSD} is also criticized for being difficult
to read and write without additional tools, it does not allow \Term[ambiguous
grammar]{ambiguous rules} that result in multiple \term{term parse} for the
same \acro{XML} document,\footnote{The concept of ambiguous and deterministic
rules in \acro{XML} is an extension of ambiguous and deterministic grammars for
formal languages (section~\ref{sec:formallanguages}) to tree languages. A
detailed analysis of computability classes for \acro{XML} schema languages is
given by \textcite{Stuehrenberg2010}, based on \textcite{Murata2005}.} and for
other reasons \cite{Clark2002}. As pointed out by \textcite{Ogbuji2002}, there
is a cultural gap between use of \acro{XML} technology with a background in
\term{markup language}s (see section~\ref{sec:markuplanguages}) and user with a
background in relational databases and object-oriented development.  \acro{XSD}
clearly origins from the second culture. 

\subsection{RDF schema languages}
\label{sec:rdfschemas}

Schema languages for \acro{RDF} (see section~\ref{sec:rdf}) are also known as
\term{ontology} languages, with the \tacro{Web Ontology Language}{OWL} as their
most popular instance.  \acro{RDF} schema languages are usually based on
\term{formal logic} systems such as \term{first-order predicate logic} or
\term{description logic}, and they originate in \term{knowledge representation}
languages from \term{artificial intelligence} research.  In its most general
form, an \acro{RDF} ontology can be described as any collection of \acro{RDF}
resources, at least if the resources are used (or proposed to be used) in
\acro{RDF} triples either as predicate, or as datatype, or as object if the
predicate is \rdf{rdf:type}.\footnote{This automatically makes \rdf{rdf:type}
part of an ontology, because it is used as predicate.} \acro{RDF} resources
from one ontology are usually summarized as \acro{RDF} vocabulary under one
common namespace. In addition, ontologies can include rules that restrict the
use of their resources in \acro{RDF} data, and that allow logical inference for
\term{entailment} in a specific \term{entailment regime} (see
page~\pageref{page:entailment}).  Given this definition of an ontology, an
\acro{RDF} schema language is a defined method to specify and describe
ontologies.  Most current \acro{RDF} schema languages are also \acro{RDF}
ontologies, that means these languages are expressed by \acro{RDF} triples. An
exception is the definition of \acro{RDF} datatypes, that make use of
\acro{XML} Schema (see section~\ref{sec:xmlschemas}). Example~\ref{ex:rdfodata}
shows some \acro{RDF} data together with parts of the \tacro{Friend of a
Friend}{FOAF} ontology and the \term{DBPedia} ontology.  The example already
shows that in practice one rarely deals with a schema language as a whole but
with specific features of schema languages.

Although a general schema language for \acro{RDF} graphs could be created as a
\term{rewriting system} on \term{graph} patterns (see
page~\pageref{p:rewritingsystem}), existing schema languages do not allow
arbitrary graph patterns, but adhere to a object-oriented system with classes,
properties, and individuals. In short, \acro{RDF} schema languages provide
resources to document parts of an ontology (\rdf{rdfs:isDefinedBy},
\rdf{rdfs:label}, \rdf{rdfs:comment}, \rdf{vs:term_status}\ldots), to define
(specific kinds of) classes and properties (\rdf{rdf:type}, \rdf{rdfs:Class},
\rdf{rdf:Property}, \rdf{owl:AnnotationProperty},
\rdf{owl:FunctionalProperty}\ldots), and to define rules and relations between
these entities (\rdf{rdfs:subClassOf}, \rdf{owl:disjointWith},
\rdf{rdfs:domain}, \rdf{rdfs:range}\ldots).  

\begin{example}[htp]
\centering
\begin{lstlisting}[language=turtle]
# RDF vocabularies
@prefix rdf:   <http://www.w3.org/1999/02/22-rdf-syntax-ns#> .
@prefix rdfs: <http://www.w3.org/2000/01/rdf-schema#> .
@prefix dc:    <http://purl.org/dc/elements/1.1/> .
@prefix foaf: <http://xmlns.com/foaf/0.1/> .
@prefix dbp:   <http://dbpedia.org/property/> .
@prefix owl:   <http://www.w3.org/2002/07/owl#> .
@prefix vs:     <http://www.w3.org/2003/06/sw-vocab-status/ns#> .

# RDF data that uses the RDF resources foaf:Person, foaf:name,
# foaf:birthday, dbp:dateOfBirth, and xs:date from ontologies.
<http://viaf.org/viaf/616522> rdf:type foaf:Person ;
  foaf:name "Dennis M. Ritchie" ;
  foaf:birthday "09-09" ;
  dbp:dateOfBirth "1941-09-09"^^xs:date .

# Parts of RDF ontologies, described by RDF schema languages:

## FOAF ontology
<http://xmlns.com/foaf/0.1/> rdf:type owl:Ontology ;
   dc:title "Friend of a Friend (FOAF) vocabulary" .

foaf:Person rdf:type rdfs:Class ;
   rdfs:isDefinedBy <http://xmlns.com/foaf/0.1/> ;
   rdfs:label "Person" ;
   rdfs:subClassOf foaf:Person ;
   owl:disjointWith foaf:Document, foaf:Organization, foaf:Project ;
   vs:term_status "stable" .

foaf:birthday rdf:type owl:AnnotationProperty, owl:FunctionalProperty ;
   rdfs:comment """The birthday of this Agent, represented in
                    mm-dd string form, eg. '12-31'.""" ;
   rdfs:domain foaf:Agent ;
   rdfs:range rdfs:Literal ;

## DBPedia ontology
dbp:dateOfBirth rdf:type rdf:Property ;
   rdfs:label "DATE OF BIRTH"@en .
\end{lstlisting}
\caption{RDF data and its ontology, described by \acrostyle{RDF} schema languages}
\label{ex:rdfodata}
\end{example}

The most basic schema language is the basic \acro{RDF} vocabulary with the
property \rdf{rdf:type}, the resource \rdf{rdf:Property}, the datatype
\rdf{rdf:XMLLiteral}, and the rules of \term{rdf-entailment} \cite[section
3]{Hayes2004}. This minimal ontology language is extended by
\textcite{Brickley2004} to the \acro{RDF} Vocabulary Description Language,
which is called \Tacro{RDF Schema}{RDFS}. \acro{RDFS} is further extended by
the \acro{OWL} family of ontology languages. \acro{OWL} was first published as
working draft by \textcite{Dean2002} and extended to \acro{OWL}2
\cite{Schneider2009}. \acro{OWL} has since superseded other general ontology
languages such as \acro{DAML+OIL}.  In addition, there are rule languages, such
as the \Tacro{Rule Interchange Format}{RIF} \cite{Kifer2010} and \Tacro{SPARQL
Inferencing Notation}{SPIN} \cite{Knublauch2011}.  The usability of ontology
languages can partly be improved by specific syntaxes, such as the Manchester
Syntax for \acro{OWL} \cite{Horridge2009} and by using specialized ontology
editors. One can also use other logic languages and even natural language. In
fact many ontologies contain additional rules and descriptions that are only
provided in natural language, because using a formal rule language would be too
complex or too strict.  For instance the restriction of \rdf{foaf:birthday} to
strings of the form ``\texttt{mm-dd}'' in example~\ref{ex:rdfodata} is only
given as deontic comment, although it could also be expressed as formal
constraint. Some rules can also be expressed in \term{Notation3}: its syntax
supports statements in \term{first-order predicate logic} by using formulae
with quantification and logical implication. The processing of this rules,
however, is much less supported by applications then standard \acro{RDFS} and
\acro{OWL}. To reduce complexity and to facilitate implementation and
computation, \acro{OWL} is split up in three different sublanguages
\cite{Dean2002} or three different profiles (\acro{OWL2}). With limited
expressibility there are the following dialects:
% \cite{OWL2:2009}

\begin{itemize}
\item \acrostyle{OWL}-Full contains all features of the original \term{Web Ontology Language}.
  It is now replaced by \acrostyle{OWL2}.
\item \acrostyle{OWL}-Lite was intended as subset of \acro{OWL} for easy 
  implementation. As it turned out to be more complex then intended, 
  \acrostyle{OWL}-Lite is now deprecated.
\item \acrostyle{OWL DL} and \acrostyle{OWL2 DL} are similar subsets of \acro{OWL} and
  \acrostyle{OWL2} that can be mapped to an extension of \term{description logic} with
  useful computational properties. The underlying logic model of \acrostyle{OWL2 DL} is 
  called $\mathcal{SQOIQ}$ \cite{Horrocks2006}.
\item 
  \acrostyle{OWL2 EL}, \acrostyle{OWL2 QL}, and
  \acrostyle{OWL2 RL} are subsets of \acrostyle{OWL2}, each
  designed for efficient computability in different application
  scenarios: \acrostyle{EL} has polynomial time reasoning complexity,
  \acrostyle{QL} supports fast query answering, and \acrostyle{RL} can
  be implemented using rule-based reasoning systems.
\end{itemize}

\acro{OWL} also adds two new numeric data types (\xml{owl:real} and
\xml{owl:rational}).  Although \acro{RDFS} and \acro{OWL} allow undecidable
statement (\acro{RDFS} is not on \acrostyle{OWL DL}) most schemas only use a
limited set of features. Furthermore, most existing ontologies can easily be
patched to belong to a less complex dialect \cite{Wang2006,MartinezGil2010}. A
general property of \acro{RDF} schema languages is their use of the \term{Open World Assumption} as default:
An \acro{RDF} ontology
assumes that any unspecified statements may exist (unless declared otherwise,
for instance with statements like \rdf{owl:oneOf}, \rdf{owl:disjointWith}, and 
\rdf{owl:complementOf}), while non-\acro{RDF} schemas assume that unspecified
data elements are not allowed (unless declared otherwise, for instance with
wildcards, such as \xml{ANY} in \acro{XML}).

\subsection{SQL schemas}
\label{sec:sqlschemas}

The first data definition language that was used under this name, was used to
define schemas of \term{relational database}s and later specified as subset of
\acro{SQL}.  In addition to this \term{SQL schema} definition language, SQL-92
introduced a schema manipulation language and schema information tables, which
provide views to database schemas in form of SQL tables \cite{Date1997}.
Alternatives to the SQL schema language, such as \term{Tutorial D} by
\textcite{Date2006} have little practical relevance, so most \acro{SQL} schemas
should be written in one common schema language. The exact specification of
this language, however, much depends on the particular \acro{RDBMS}, because
each database system implements its own subset of some version of the
\acro{SQL} standard with its own additions and modifications. In general, a
schema consists of a set of \sql{CREATE} statements.  The most important
statement is \sql{CREATE TABLE} to define database tables.  In addition one can
define derived datatypes with \sql{CREATE DOMAIN} or a similar command, and
group table and datatype names in namespaces with \sql{CREATE SCHEMA}. Derived
datatypes can have their own default values and an optional \sql{CHECK} clause,
for instance to limit the possible values to a custom range, but this features
are not fully supported by all \acro{RDBMS}.  A table definition consists of a
non-empty sequence of column definitions, where the order of fields is
irrelevant for most applications. Each column is identified by a name, unique
per table, it is associated with a datatype (see below) and it can have an
optional default value. The default value which may also be generated
automatically with the \sql{IDENTITY} statement (also known as auto-increment)
and a selected numbering scheme).  In addition, each column and the table as a
whole can be shaped by the following constraints:

\begin{itemize}
 \item \sql{UNIQUE} marks column values to be unique per table, so each 
   row can be identified by its value, unless the value is \sql{NULL}.
 \item \sql{NOT NULL} enforces a column to not contain \sql{NULL} value.
 \item \sql{PRIMARY KEY} marks up to one column per table. 
   It implies \sql{UNIQUE} and \sql{NOT NULL}.
 \item \sql{FOREIGN KEY} values must reference an existing \sql{PRIMARY KEY} in a 
   specific table. This constraint is important to connect multiple tables of one schema.
 \item \sql{CHECK} can limit values based on boolean expressions over one or more
   columns. For instance one can check that the date of death is not before the date of birth
   for a table with two columns \sql{BIRTH} and \sql{DEATH}.
\end{itemize}

\acro{SQL}-92 also introduced a \sql{CREATE ASSERTION} statement for
constraints that can span multiple tables, but this feature is not supported by
most \acro{RDBMS}. Triggers and other settings may also shape the actual
database content, similar to a schema, but this highly depends on the specific
database application. In addition to tables, there can be views, defined with
\sql{CREATE VIEW}. Views are virtual tables that map to normal tables via an
\acro{SQL} query.  If there is a one-to-one relationship between rows in a view
and rows in tables it refers to, among other limitations, the view may be
counted as part of a schema.  The list of \acro{SQL} datatypes for column
definitions has slightly changed with each version of the standard, and support
of datatypes differs among \acro{RDBMS}. In summary there are:

% see http://www.postgresql.org/docs/8.4/static/datatype.html

\begin{itemize}
\item numeric types, including integer types and decimal types with fixed or arbitrary precision, 
  and floating-point types
\item character string types with variable or fixed length, possibly padded by whitespaces
\item temporal types for dates, times, and intervals
\item binary data types for sequences of bits or sequences of bytes
\item a boolean type (not exactly supported by most systems)
\item an \sql{XML} type to store \acro{XML} documents and fragments
\item an \sql{ARRAY} type for one- or multi-dimensional sequences of same type and a 
  (rarely supported) \sql{MULTISET} type for unordered collections of same type value. Some 
  systems also support non-standard nested tables and associative arrays.
\item composite types to group one or more unconstrained columns to be used as datatype
  in another column
\item types for specialized domains, such as monetary types and spatial types
\end{itemize}

% TODO: refer to patterns!
% graph data types are rare
% for instance Microsoft SQL Server 2008 added a 
% HIERARCHYID type to store the path in a tree structure to better
% support hierarchical structures.

Example~\ref{ex:ormsql} in section~\ref{sec:docobj} includes a simple
\acro{SQL} schema for illustration.

\subsection{Other schema languages}
\label{sec:otherschemas}

% The only alternative: orm? automatically generated SQL schemas?

Given \acro{XML} schema languages for \acro{XML} data, \acro{RDF} schema
languages for \acro{RDF} data and \acro{SQL} schema language for \acro{SQL}
data one can think of many more schema languages for each particular data
structuring method: there are several proposals for \acro{JSON} and \acro{YAML}
(\term{Kwalify}, \term{JSON Schema}, \term{Rx}, \term{Simple Declarative
Language} etc.) and several \term{data binding language}s come with their own
\term{data definition language} (see part~\ref{sec:databinding} and
table~\ref{tab:dslrpcformats}). Terminology among schema languages differs ---
the same concept may be called `schema', 'format', or 'message' in different
languages --- but in general schema languages share a set of common concepts,
similar to \term{type system}s of programming languages: there is a set of
predefined, possibly extensible data types and data elements can be named,
grouped, and constraint as mandatory or optional. In addition one can often
find methods to combine, extend and restrict schemas.  To give an example,
example~\ref{ex:rxtypes} lists predefined data types and methods of
constraining in the \term{Rx} schema language \cite{Signes2008}.
Table~\ref{tab:protocolbuffersdatatypes}
(page~\pageref{tab:protocolbuffersdatatypes}) includes another example with
data types from \term{Protocol Buffers}.  Some schema languages make use of
\term{regular expression}s and elements from \term{context free grammar} when
defining hierarchical or sequential data.  Methods to specific
context-sensitive rules which correlate multiple data elements are less
frequent, and they are often restricted to a set of rule types, such as
referential constraints with keys and \term{foreign keys}.  Less limited
languages to defined and describe data structures will be covered as
\term{modeling language}s in section~\ref{sec:modelangs}.  In the following,
programming languages, forms, and the \tacro{Data Format Description
Language}{DFDL} will be explained briefly.

\begin{example}
\begin{tabularx}{\textwidth}{|l|X|}
\hline
\textbf{Type} & \textbf{Content} \\
\hline
nil  & the lack of a value (undef, null, nil, etc) \\
\hline
def  & any defined value; anything that is not nil \\
\hline
bool & a value that is either true or false \\
\hline
num  & a number; may be parameterized by a range (fixed or inclusive or exclusive minimum and/or maximum value) \\
\hline
int  & an integer; may be parameterized by a range \\
\hline
str  & a string, even the empty string \\
\hline
one  & any of bool, num, int, and str \\
\hline
arr  & a list of values all of one type; may be parameterized by a range as length \\
\hline
seq  & a list of values of different, given types \\ % tuple
\hline
rec  & a record of named entries, each with its own schema and a required/optional flag \\
\hline
map  & a map of names to values, with all the values of one type \\
\hline
any  & either anything at all, or any of a given list of types (union) \\
\hline
all  & a combination of types (intersection) \\
\hline
\end{tabularx}
\caption{Predefined data types and elements of the Rx schema language}
\label{ex:rxtypes}
\end{example}


\subsubsection{Programming languages}

Many schemas are expressed in some
programming language, either explicitly by the language's type system or
implicitly by the behavioral logic of a specific program.

A schema, expressed in a particular schema language, can also be viewed as
source code that is run by a validator (see image~\ref{fig:schemavalidator}).
One can create a program for validation of documents in any Turing-complete
\term{programming language}. This implies programming languages can be used as
schema languages for decidable data format. Such program makes the data format
implicit because the program will contain many parts which are not directly
related to the data format. With implicit schemas, hard-coded in programming
languages, it is more difficult to share and compare schemas. These also
applies partly for schemas that are automatically created by other tools, so
the created schema is only a compiled artifact from an original schema in
another program or language. In exchange validation may be more performant and
provide better document analysis, for instance more helpful error messages.  In
practice many data formats have no explicit schema, so the programs that read
and write data are the only reliable source to infer common data structures.
Even programs that make use of explicit schemas may modify its rules, for
instance by adding practical limits on length of values or other constraints
which cannot be expressed in the particular schema language. Therefore one
should always distinguish the explicit part of a schema, as expressed in a
schema language, and the implicit part of a schema, as only implemented in
applications.  Especially the meaning of a data element can often better be
determined by looking at its use in applications instead of relying on labels
and descriptions in official schemas.

% Programming language contain type systems which are covered
% well in literature. See \ref{sec:dsl} for the relation
% between type systems and data structuring languages.

\subsubsection{Forms and questionnaires}
\label{sec:forms}

Standardized \term[form]{forms} and questionnaires are ubiquitous methods to
collect and edit data, predating all digital data management. The
\textit{Marcufli Formulae} from the late 7th century can be considered as first
collection of forms \cite{Zeumer2001}: it contained boilerplates for deeds,
that were used in the Frankish Empire. Modern printed forms have been used from
the 19th century on, also with legal pleadings and contracts as first
applications. Their primary purpose is attestation: documentation of
statements. In contrast to other documents, a form highlights the relevant
statements to be documented by forcing them into predefined structures. By
this, a form can be seen as kind of schema and as visible interface to a data format.
The form gives a frame to actual data and it can influence what and how
data is entered, for instance in form of good or bad usability. A deeper
analysis of dedicated forms could explain some artifacts in the data that was
edited through them, but this would go beyond the scope of this thesis. Most
aspects of the history of forms still have to be written. For a starting point
see \textcite{Becker2007} and \textcite{Grosse1980}.

\subsubsection{Data Format Description Language}
\label{sec:dfdl}

The \tacro{Data Format Description Language}{DFDL} is an extension to a subset
of \term{XML Schema} to describe binary and text formats for mapping them to
\acro{XML} \cite{Beardsmore2007,Powell2011}. Version~1.0 of \acro{DFDL} has
been published as proposed recommendation in 2011. A \acro{DFDL} schema
consists of an \acro{XSD} schema that defines \acro{XML} elements (\acro{XML}
attributes are not supported) and mappings from these elements to data elements
in the non-\acro{XML} format to be described. \acro{DFDL} supports sequences,
choices, grouping, optional, and repeatable elements, so one can use it like
\term{Backus-Naur Form} to parse word in formal languages to \term{syntax
tree}s in \acro{XML}. \acro{DFDL} also supports unordered sequences,
delimiters, length indicators and padded fields of fixed length, tags,
terminators, default values, escape characters, regular expressions etc. that
facilitate the description of binary and textual formats. A subset of XPath~2.0
can be used to express dependencies and calculations that even span local
boundaries, similar to \xml{context} and \xml{test} expressions in
\term{Schematron}.

%\TODO{Should I include an example of a DFDL schema?}

