\section{Databases}
\label{sec:databases}

\begin{quotation}%
Historically, data base systems evolved as generalized access methods [\ldots]
As a result, most data base systems emphasize the question of how data may be
stored or accessed, but they ignore the question of what the data means to the
people who use it.\\
\quotationsource \Person[John F.]{Sowa} \citeyear{Sowa1976}: 
\textit{Conceptual Graphs for a Data Base Interface}
\end{quotation}

\noindent
A \Term{database} is a managed collection of data with some common structure.
The general form of a database is shaped by its \Term{database model}, which
is implied by the particular implementation of a \Tacro{database management 
system}{DBMS}. Overviews of database models can be found in 
\textcite{Silberschatz2010}; \textcite{Elmasri2010}; \textcite{Navathe1992};
and \textcite{Kerschberg1976}. Although you can identify some general model
types, the exact definitions of specific database models differ. As pointed 
out by \textcite[chapter 9]{Kent1978}, comparisons should not confuse data models 
and concrete implementations. Database models rarely occur as such, but only 
as abstractions of \acro{DBMS} implementations. The model can be derived from 
an explicit specification, from the \acro{DBMS}' schema language
(see section~\ref{sec:sqlschemas}), and from its query \acro{API} (see 
section~\ref{sec:apis}).

% a database model is a set of conceptual tools to model representations in a database.

The development of database systems and models is not a logical chain of
improvements, but driven by trends and products. In the 1960s the difference
between file systems, as described in the previous section, and database systems 
was still small --- for instance IBM marketed a series of mainframe \acro{DBMS}es
`Formatted File System'. Simple databases were (and still are) build of plain
\term[record]{records} and tables without any elaborated model
(section~\ref{sec:records}). The hierarchical model (section~\ref{sec:hierarchicaldatamodel})
and the network model (section~\ref{sec:networkdatamodel}) were designed close to
properties of the underlying storage media. A better separation between
\term{logical level} and \term{physical level} was introduced with the
relational model (section~\ref{sec:rdbms}). Since the 1970s (in database research) and
the 1980s (in commercial products), it has become the preferred database model, 
at least in its interpretation by \acro{SQL}. In the 1980s object oriented
databases (section~\ref{sec:oodbms}) and graph databases appeared. Object orientation
had impact on relational databases, which partly evolved to object-relational
databases. Graph databases experience some revival since the late 2000s in
connection with the NoSQL movement (section~\ref{sec:nosql}).

So called `semantic database models' \cite{Hull1987,Peckham1988} will be 
described as conceptual data models in section~\ref{sec:modelangs}. Other
more specialized databases models not covered here are spatial, temporal,
and spatio-temporal databases, and multidimensional databases. The former add
better support of space, time, and versioning \cite{Chen2001}. The latter are
used to efficiently summarize and analyze large amounts of data
\cite{Vassiliadis1999}.

\subsection{Record databases}
\label{sec:records}
The \Term{record model} dates back to pre-electronic data processing with
\term{punched cards}. A database was a set of punched cards that each stored one
record.\footnote{See \textcite{McGee1981} on database history and the quote about
`files' \cite[p. 3]{Saltzer1965}, also at page~\pageref{quot:files}.} Records are
usually stored in databases or files. Although records are still the most used
form of structuring data, the record model is not often described in database
textbooks and research. A detailed critique is given in \textcite{Kent1979} and
\textcite[ch. 8]{Kent1978}. Internally data may be structured in different
ways, but the most prominent medium to enter, edit, and display data is the
\term{form}, which usually is shaped as a record. 

In its most general sense the term \Term{record} is used for any
collection of related data, storage devices, files, documents and more, equal to
the general term of a digital `\term{document}'. In a more strict sense, a record
is a grouped unit of data elements, that are called its \Term[field!of a
record]{fields} or its \textit{attributes}. \index{attribute|see{field}} The
fields may be ordered in a sequence and identified by \Term[field name]{field
names} or indices. Field names often act as a mnemonic aid to human users. In
most databases they are not part of the record, but included with field
descriptions in the specification of a \Term{record type} or \Term{record
schema} which records conform to. To map data elements of a given record to
fields, the elements must be separated and identifyable. There are three
methods to fulfill this requirement:

\begin{itemize}
 \item the record type defines a fixed set of fields with fixed length and
       position
 \item the record type defines a fixed set of fields with fixed order
 \item field name are included in the record
\end{itemize}

\begin{figure}
\centering
\begin{tikzpicture}[decoration={brace}]
\matrix (dcr) [datamatrix] {
  dc:creator   & =
    & Brian W. Kernighan; Dennis M. Ritchie & |[ucs]| 0A \\
  dc:title     & =
    & The C programming language            & |[ucs]| 0A \\
  dc:publisher & =
    & Englewood Cliffs, NJ : Prentice-Hall  & |[ucs]| 0A \\
  dc:date      & =
    & 1978                                  & |[ucs]| 0A \\
};
\draw[decorate] 
(dcr-4-2.south west) -- node[anchor=north,inner sep=2mm] {field names}
(dcr-4-1.south west); 
\draw[decorate] 
(dcr-4-4.south west) -- node[anchor=north,inner sep=2mm] {field values}
(dcr-4-3.south west);
\draw[decorate] (dcr-1-4.north east) -- node[anchor=west,inner sep=2mm]
 {record} (dcr-4-4.south east);

\draw[<-] (dcr-4-2.south) |- +(1em,-2.6em) node[anchor=west] 
(sf) {separators};
\draw[->] (sf) -| (dcr-4-4.south);
\end{tikzpicture}
\caption{Dublin Core record (field names included)}
\label{fig:dcrecord}
\end{figure}

With the first two methods, the record type strictly defines, which fields must
occur in which order. Fields may further be described by a \term{data type} or
\term{domain}, that values of the field in each record must conform to. This
corresponds to the classical notion of records as described by \textcite[ch.
8]{Kent1978}, and as used in other database models. Both methods impose a
restriction on field values, either on their length, or on the set of symbols
allowed to represent field values. The second method needs at least one special
symbol to separate field values (table \ref{ex:separatorchars}; other popular
symbols are comma, semicolon, and space). If the field separator symbol occurs
in field values, it must be escaped. The second method is favoured in database
theory, because it directly maps to the mathematical concept of a \term{tuple}
and you can list multiple record in a table (figure~\ref{fig:csvexample}).
However single records are not self-describing and do not allow exceptions or
repeated fields. In practice, they lead to the invention of special
\term{NULL}-values to denote `not applicable' or `unknown' field values and to
ad-hoc formatted lists of values, packed in one field.\footnote{See the list of
names, the creator field in figure \ref{fig:dcrecord}.} The third method is
more flexible, but it also needs some separators between field names and field
values. An example of a self-describing\footnote{The term `self-describing' is
used with similar carelessness as the term `semantic'.  In most cases, the only
`description' of self-describing data is a simple mapping of data elements to
their field names.} record is shown in figure~\ref{fig:dcrecord}. It contains a
Dublin Core (\acro{DCMES}) record, encoded with the special characters `='
(\U{3D}) and line break (\U{0A}).  Another example with numbers and characters
as field names is given in appendix~\ref{appendixD}.

The inclusion of field names in records is also known as \term{markup}, which
is described in more detail in section~\ref{sec:markuplanguages}. Markup allows
you to freely omit, repeat, and order fields. Such records do not even need a
record schema but can consist of a simple list of field names and field values,
such as in \acro{INI} files (see section~\ref{sec:csvandini}). These
schema-free records have been avoided in most databases the last decades, but
they are getting popular again with \term{NoSQL} databases. If fields are
unordered and they can only occur once per record, the record model corresponds
to the concepts of \term{associative arrays}, \term{maps}, \term{hashtables},
or dictionaries from type theory and programming languages. If stored or sent
as sequence of bytes, however, the record does not show, whether fields are
ordered and which fields have been omitted. Given the record in
figure~\ref{fig:dcrecord}, you need background knowledge about the record type
\tacro{Dublin Core Metadata Element Set}{DCMES} to know, that the order
\verb|dc:creator|, \verb|dc:title|, \verb|dc:publisher|, \verb|dc:date| is not
relevant, and that there are eleven other possible fields defined in
\acro{DCMES}.  Other interpretations of \acro{DCMES} allow repeating the
\verb|dc:creator| field to express lists of creators, which requires field
order to be preserved.  The limitation of fields of a record to non-repeatable,
atomic values -- also known as \term{first normal form} -- is often assumed
implicitely. But as described by \textcite{Fotache2006}, the notion of
atomicity depends on context.  For instance in figure~\ref{fig:dcrecord} you
could split the \verb|dc:publisher| field value into the publisher's name
(\verb|Prentice-Hall|), place (\verb|Englewood Cliffs|), and state (\verb|NJ|).
\verb|dc:title| could be split in words and characters, and \verb|dc:date| in
century, decade, and year of the decade. The inclusion of field names in
records increases flexibility, but it still selects a specific set of fields,
that may be quite different in another context.

Some record models, for instance the MultiValue/PICK database and \acro{MARC}
records, allow fields to be repeated or split up into subfields. If you allow
arbitrary nesting of records in fields, subfields, sub-subfields and so on, you
end up with a hierarchical database. If you restrict the number of levels to a
fixed value $n$ you can represent an ordered, \term{flat file database}
structure with $n$ special separator elements that must not occur in (sub)field
values. ASCII defines four such control characters that were used in many
binary flat file formats (example~\ref{ex:separatorchars}).

\begin{table}
\begin{tabular}{|c|l|l|l|}
 \hline
 Code & \acro{ASCII} name & Unicode name & \acro{MARC} name \\
 \hline
 \U{1C} & File Separator (FS)   & INF. SEPARATOR FOUR
        & -- \\
 \U{1D} & Group Separator (GS)  & INF. SEPARATOR THREE
        & Record Separator \\
 \U{1E} & Record Separator (RS) & INF. SEPARATOR TWO
        & Field Separator \\
 \U{1F} & Unit Separator (US)   & INF. SEPARATOR ONE
        & Subfield Delimiter \\
 \hline
\end{tabular} 
\caption{Separator control characters in \acrostyle{ASCII}, Unicode, and 
	 \acrostyle{MARC}}
\label{ex:separatorchars}
\end{table}

Records are used as building blocks in many data structuring
methods. The database model of a plain record store is also known as \Term{flat
file database model}. Records in a \term{flat file database} are typically stored
and accessed sequentially, which enforces an order on all records and fields, no
matter if this order is relevant or not. Figure~\ref{fig:flatfilemodel} shows
the model of a flat file database with atomic fields: Each file may have
zero or more records, which each must belong to exactly one file. Each record
may contain zero or more fields, which each must belong to exactly one record,
and have exactly one field value. To refer to a particular record or field, it
must be \Term[index!in a database]{indexed}. A record index is also called
\Term{record identifier}. If records and/or fields are ordered, their position
can be used as one index. Field names are the usual index for fields, but only
for non-repeatable fields. Selected field values, or combinations of
multiple field values, can be used as record identifier; but only if every
record happens to contain the selected fields, and if their values uniquely
identify the records. Such additional constraints are not part of the record
model, but fundamental in the relational model.

\begin{figure}
\centering
\begin{tikzpicture}[orm,lpin/.style={label distance=0mm},
ormpin/.style={label=[lpin]above:\ormind{*}},
nmrole/.style={roles,unique=2,ormpin}]
\entity (file) at (-4,-2.7) {File};
\entity[right=1.5 of file] (record) {Record} 
       edge[mandatory] node[nmrole] {} (file);
\entity[right=1.5 of record] (field) {Field} 
       edge[mandatory] node[nmrole] {} (record);
\value[right=1.5 of field] {Value} 
       edge[both mandatory] node[roles,unique] {} (field);

\node[rule=*,below=1mm of file.south west,anchor=north west]
  {indexed by position (if ordered), identifier and/or field name};
\end{tikzpicture}
\caption{Flat file database model}
\label{fig:flatfilemodel}
\end{figure}

A \Term{table of records} is the usual method to manage files of records, if
all records share the same fixed set of fields (figure~\ref{fig:csvexample}).
The first row contains the record schema by listing its field names, and each
following row contains one record. The table header can be omitted, if fields
are indexed by their position. Single records are also indexed by their
position, so record identifiers are neither part of the record.

\begin{example}
\centering
\begin{tabular}{|l|l|l|}
\hline
\textbf{author} & \textbf{title} & \textbf{year} \\
\hline
Kerninghan and Ritchie & The C programming language & 1978 \\
\hline
Bjarne Stroustrup & The C++ programming language & 1985 \\
\hline
\end{tabular}

\begin{tikzpicture}
\matrix (csv) [datamatrix] {
~ & & & & & & \\
~ & author &,& title &,& year & |[ucs]| 0A \\
~ & Kerninghan and Ritchie &,& The C programming language &,& 1978 & |[ucs]| 0A
\\
~ & Bjarne Stroustrup &,& The C++ programming language &,& 1985 & |[ucs]| 0A \\
};
\draw[<-] (csv-4-3.south) |- +(1em,-1.5em) node[anchor=west] (sf) {separators};
\draw[->] (sf) -| (csv-4-5.south);
\draw[->] (sf) -| (csv-4-7.south);
\draw[-] (csv-4-1.south) |- +(.5em,-1.5em) 
  node[anchor=west] {record schema (first row)};
\draw[->] (csv-4-1.south) |- (csv-2-2.west);
\end{tikzpicture}

\caption{Table of records as formatted table and as \acrostyle{CSV}}
\label{fig:csvexample}
\end{example}

% A general, non-obvious question whether indices are part of the record 
% index is part of the record or not uniqueness 
% indices that span multiple fields.

The lack of a clear definition of record identifiers is one drawback of the
record model. Without record identifiers as link target you cannot express
relations that span multiple records. On the other hand there are implicit
relationships between the fields of one record, but some relationships within
a record cannot be described \cite[ch. 8.4f]{Kent1978}. It even depends on
context, whether a record represent an entity or a relationship. In summary,
records lack a clear method to express relationships while struggling with a 
rich variety of representational alternatives \cite{Kent1988}. Nevertheless 
the record concept is useful in grouping data elements, and can be found on
various levels --- you must only take care which variant of the record model
you deal with in a particular application.

% Note that this general definition of a record contains no restriction on the
% nature of its data elements.

% Such a subdivision of fields can
% result in a multimap, but you must carefully define whether subfields are
% ordered and/or indexed.
% the general model of a record is a tree of fixed depth.

% TODO: type theory:
% In computer science records are stored with the abstract data
% type \textit{associative array}, also known as \textit{map}. 


\subsection{Hierarchical databases}
\label{sec:hierarchicaldatamodel}

\textit{Hierarchical databases} are among the oldest and longest running
database systems. \Term*{hierarchical database model}
The first hierarchical \acro{DBMS} -- \term{IBM}'s
\tacro{Information
Management System}{IMS} -- was developed in 1966-1968 and is still
used today by a large number of banks, insurance companies and similar
organizations.\footnote{See \textcite{Blackman1998},
\url{http://www.ibm.com/software/data/ims/}, and
\cite[appendix E]{Silberschatz2010}.} 
Popular specialized storage systems using the hierarchical model are file
systems (section~\ref{sec:filesystems}), directory services such as 
\aterm{LDAP}{Lightweight Directory Access Protocol}, and the
\term{Domain Name System} (section~\ref{sec:dns}). 

In a hierarchical database the data is organized into mathematical tree structure.
Records may be typed and may contain data in typed field values (attributes).
There is one special type to connect records via 1:m parent-child relationships.
Each record can only have one parent node (unless it is the root
node that has no parent) and may have one or more child nodes. A database 
can be described as a set of trees, each having the same structure (you can also
have multiple parallel tree structures by adding a virtual root node). Managing
data in a hierarchical database is comfortable as long as the information to be
stored is also hierarchic in nature. Other structures require
additional arrangements as shown in the following example:

Figure~\ref{fig:hnetexamples} at the left shows a hierarchical database to
store information about libraries and their publications. Each {\ormtext
library} has one {\ormtext catalog}. A {\ormtext publication} in the catalog may
be assigned to one {\ormtext topic} or more. Topics can be arranged into a
hierarchy. We assume that topics follow the mono-hierarchy of a classification
but not the poly-hierarchy of a thesaurus. Each time the library acquires an
item, the {\ormtext vendor} and the {\ormtext publication} are stored.
As there are only 1:n relationships, the hierarchic database cannot ensure
that each library has only one catalog ({\ormtext\small\textbf 1}).
Relations only span two records, therefore the ternary relationship between
library, vendor, and publication must be expressed as record
({\ormtext\small\textbf 2}). This also requires a virtual hierarchy
(called ``logical'' in contrast to ``physical'' in \acro{IMS}) between vendor
and acquisition and between publication and acquisition. The n:1 relationship
between acquisition and publication also requires a virtual hierarchy
({\ormtext\small\textbf 5}). The n:m relationship between publications and
topics ({\ormtext\small\textbf 4}) and the recursive 1:n relationship between
topics and subtopics ({\ormtext\small\textbf 3}) are also modeled by virtual
hierarchies but the monohierarchy cannot be expressed.

Virtual hierarchy pointers can be used to circumvent some of the hierarchic
model's limitations, similar to \term{symbolic links} in file systems
(section~\ref{sec:symlinks}). Yet in practice they are less efficient and
their integrity is not ensured by the \acro{DBMS}. Other hierarchic
\acro{DBMS}, for instance native \acro{XML} databases share similar
limitations.


\begin{figure}
\begin{tikzpicture}[orm,
level 1/.style={sibling distance=2cm},
pointer/.style={dashed,->,shorten >=2pt,shorten <=2pt,thick},
set/.style={<-,shorten <=2pt,thick}
]
\entity (vendor) at (2.2,0) {vendor};
\entity (library) at (0,0) {library}
  child {
    node[entity] (catalog) {catalog}
    child {
      node[entity] (topic) {topic}
    }
    child {
      node[entity] (publication) {publication}
    }
  }
  child {
    node[entity] (acquisition) {acquisition}
  };
  \draw (acquisition) edge [pointer] (publication);
  \draw (publication) edge [pointer] (topic);
  \draw[pointer] (topic.240) arc (360:90:3mm);
  \draw (acquisition) edge [pointer] (vendor);

  \node[above=0.3 of library] {\textit{hierarchy 1}};
  \node[above=0.3 of vendor] {\textit{hierarchy 2}};

  \node at (-.8,-.7) {\ormind{1}};
  \node at (1.0,-.9) {\ormind{2}};
  \node at (-2.6,-2.7) {\ormind{3}};
  \node at (-1.2,-2.9) {\ormind{4}};
  \node at (.1,-2.3) {\ormind{5}};

  \rules[anchor=north west] at (-3, -4.2) {
    \node {\textbf{Hierarchical database}};\\
    \node[rule=1] {1:1 constraint not expressible};\\
    \node[rule=2] {n-ary relationship requires virtual hierarchy};\\
    \node[rule=3] {recursive relationship requires virtual
hierarchy\textbf{\small{*}}};\\
    \node[rule=4] {m:n relationship requires virtual hierarchy};\\
    \node[rule=5] {n:1 relationship requires virtual hierarchy};\\
    \node[rule=*] {some hierarchical models allow recursion};\\
  };

% \draw [dotted] (4.3,1.4) to (4.3,-7.4);

\entity (vendor) at (9.5,.5) {vendor};
\entity (library) at (7.7,.5) {library}
  child[set] {
    node[entity] (catalog) {catalog}
    child {
      node[entity] (topic) {topic}
      child[bend right] { node[entity] (tlink) {tlink} }
    }
    child {
      node[entity] (publication) {publication}
      child { node[entity] (tplink) {tplink} }
    }
  }
  child[set] {
     node[entity] (acquisition) {acquisition}
  };
  \draw[set] (publication) to (acquisition);
  \draw[set] (topic) to (tplink);
  \draw[set] (topic) to[bend right] (tlink);
  \draw[set] (vendor) to (acquisition) ;

  \draw (library)++(-.7,-.7) node {\ormind{1}};
  \draw (library)++(1,-1) node {\ormind{2}};
  \draw (tlink.west)++(-.2,0) node {\ormind{3}};
  \draw (tplink.west)++(-.2,0) node {\ormind{4}};

  \rules[anchor=north west] at (4.8, -4.9) {
    \node {\textbf{Network database} (link types not shown)};\\
    \node[rule=1] {1:1 constraint not expressible};\\
    \node[rule=2] {n-ary relationship modeled as record};\\
    \node[rule=3] {recursive relationship modeled as record};\\
    \node[rule=4] {m:n relationship modeled as record};\\
  };

\end{tikzpicture}
\caption{Hierarchical and a network database with its limitations}
\label{fig:hnetexamples}
\end{figure}

\subsection{Network databases}
\label{sec:networkdatamodel}

The \Term{network database model} evolved in the 1960s from the
\tacro{Integrated Data Store}{IDS} \acro{DBMS}. Under the guidance of
\Person[Charles]{Bachman} and the \tacro{Database Task Group}{DBTG}
within the \tacro{Conference of Data Systems and Languages}{CODASYL}, 
it resulted in the first database standard specification
\cite{CODASYL1971,CODASYL1978}. Many basic concepts of database
terminology were introduced by the \acro{DBTG}, including the 
difference between a \tacro{data description language}{DDL} to
define the \term{database schema} and a \tacro{data manipulation 
language}{DML} to query and modify the content of a database. The 
\acro{DBTG} model can be described as a partly ordered graph with typed
\Term[record!in a network database]{records} as nodes and typed
\Term[link!in a network database]{links} as edges. Records can have attributes
as described in section~\ref{sec:records}. Similar to the hierarchical model,
links are limited to 1:n relationships. In the \acro{DBTG} model a
relationship is called \Term[set!in a network database]{set} with the
relationship type as \textit{set type}, one record as \textit{owner} and zero
or more records as \textit{member} of the set. The \acro{DDL} statement
``\texttt{insertion is automatic; retention is mandatory}'' can mark a set type as
required for the member record. Other features of the \acro{DBTG} model include
unique key attributes, ordering sets by a selected record attribute, and
singular sets. \index{ordered set!in network databases}

In contrast to the hierarchical model there is no separation between
hierarchical links and virtual pointers. A record can be a member of multiple
owners, as long as each membership takes place in a different set type.
Other limitations of the hierarchical model are also present in the network
model: there is no separation between 1:n and 1:1 relationships
({\ormtext\small\textbf (1)} in fig.~\ref{fig:hnetexamples} on the right) and
relationships with more then two members must be modeled by records
({\ormtext\small\textbf (2)}). Such additional 
\Term[junction record!in a network database]{junction records} are also used to model
recursive relationships ({\ormtext\small\textbf 3}) and
n:m relationships ({\ormtext\small\textbf 4}).

Later extensions of the network database model evolved to the \Term{role data
model}. It was planned as generalization of the network and the relational
model, similar to object databases, but never got really adopted
\cite{Bachman1977,Steimann2007}.


\subsection{Relational databases}
\label{sec:rdbms}
The \Term{relational database model} was introduced by \textcite{Codd1970} as
superior alternative to hierarchical and network database models. It overcomes
data dependency on ordering, indexing, and access paths, and it highlights 
the separation between logical level and physical level. Together with the
relational model, \person[Edgar F.]{Codd} introduced the idea of 
\term{database normalization} to avoid redundancy and inconsistencies.
Based on set theory and predicate logic, the relational
model gave database research a highly stimulating, mathematical foundation.
However its properties are often confused with those of the 
\tacro{structured query language}{SQL}, and the 
conceptual \tacro{entity-relationship model}{ERM}. Both are influenced
from the relational model, but with deviation from its original design, and both 
had far more impact on implementations. These implementations are now known as 
\Tacro{relational database management system}{RDBMS}. 

In the \term{relational database model}, you declare data and queries as 
logic predicates in form of  $n$-ary \Term[relation!database]{relations}, while 
the \acro{DBMS} takes care of storing and retrieving the data. Similar to the
mathematical sense of a \term{relation}, database relations are defined as 
subset of the cartesian product $S_1 \times \cdots \times S_n$ over $n$ sets
$S_1, \ldots, S_n$. The sets $S_i$ ($i=1,\ldots,n$; the sets need not be distinct)
are indexed by unique names for each relation, and called its 
\Term[column of a database]{columns}. In contrast to mathematical relations, the
rows of a database relation have no order. The relation's tuples, which are 
called its \Term[row of a database]{rows} are also unordered. In short, a
relation in the relational model is a set of distinct \term{record}s, like in
the record database model, with a fixed set of fields, which are indexed by 
names. Each fields can be restricted to a given \term{data type}, that is 
called its \Term{domain}. The collection of fields, field names, and domains
of a relation are sometimes called its \Term[record type]{(record) type}. 
The relational model defines a \Term{relational algebra} based on relations
and types, with the basic operators selection, projection, cross join, 
relational set union, relational set difference, and rename. 

% The exclusion of other possible relations, for instance 
% \term{transitive closure} is one of the limitations of the 
% relational model. (TODO: where is transitive close described?)

Originally, \person[Edgar F.]{Codd} did not specifiy a \term{query language}
for the relational model, but required that any relational query language must
be based on relational algebra. \textcite{Chamberlin1974} presented such a query 
language, that later evolved to \acro{SQL}. Current dialects of \acro{SQL} 
include numerous extensions, and only share basic ideas with the original
relational model. As explained by \textcite{Darwen1995}, \acro{SQL} violates
the relational model in several ways, especially by using simple tables of
records, which allow duplicated rows instead of set-based relations, and by 
inclusion of \term{NULL} values. \textcite{Atkinson1989} point to
\acro{SQL} as the ``classical, and unfortunate, pattern in the computer field 
that an early product becomes the \emph{de facto} standard and never disappears''.
At the same time there is no consensus on what \acro{SQL} really is.
Only between 1987 and 1996, the \tacro{National Institute of Standards and 
Technology}{NIST} provided a test suite for \acro{SQL}, which \acro{RDBMS} 
vendors had to conform to, to get used in governmental funded projects.%
\footnote{The latest test suite from 1996, testing \acro{SQL}, as specified
in ANSI X3.135-1992 (SQL-92) is available at %
\url{http://www.itl.nist.gov/div897/ctg/sql_form.htm}.}
Meanwhile each \acro{RDBMS} has its own restrictions and extensions. 
The \acro{SQL} standard, as specified by \acro{ISO}, has grown in complexity 
and size with each new version, and no product fully implements each detail.
Moreover, the specification is not freely available, which makes it hard to 
check, whether a specific language construct or implementation conforms to
standard \acro{SQL}. Despite its divergence from the relational model, and 
its lack of a reliable, vendor-independent specification, which deserves that 
appellation, \acro{SQL} is perceived as lingua franca of \acro{DBMS} in 
general. This dominance also takes into account for the dominance of the 
relational database model.\footnote{In short, the relation model had a
lot of impact, especially spoiled and misinterpreted by
\acro{SQL}, which also had a lot of impact, especially spoiled and 
misinterpreted by its differing implementations.}
Critics from database research, such as \textcite{Stonebraker2007} and
\textcite{Darwen1995} therefore often argue against \acro{SQL} and 
\acro{RDBMS}, but less against the relational model as introduced by 
\person[Edgar F.]{Codd}.

The concept of \Term{database normalization} \index{normalization!in relational databases} 
was a basic part of the relational model from the beginning.
%introduced by \textcite{Codd1970} as basic part of the relational model. 
Similar to relational model in its original form, normalization is not 
fully applied in practice \cite{Fotache2006}. Beginning with the 
\tacro{first normal form}{1NF}, normalization techniques were soon 
extended by the \tacro{second normal form}{2NF} and the \tacro{third normal form}{3NF}
\cite{Codd1971}, the \tacro{Boyce-Codd normal form}{BCNF} \cite{Codd1974},
the \tacro{fourth normal form}{4NF} \cite{Fagin1977}, and more forms. The 
general objective of normalization is to map relations and dependencies
information to database relations, without introducing possible redundancy
and inconsistencies. Single facts should only be stored once, and facts, that
can be derived from other facts, should not be stored at all. As a result,
no queries are favoured at the expense of others. Any complex query can 
be build by joining multiple tables, that share fields of the same domain.
To speed up specific queries, indexes and views, that act
as caches of query results, can be created. In practice, some parts of the database are
not normalized on purpose for performance reasons --- in this case, the
database user must take care of database integrity.%
\footnote{A popular example of denormalized databases are data cubes in
\tacro{Online Analytical Processing}{OLAP}.}

Leaving performance issues aside, normalization is also useful independent
from relational databases. \Acro{1NF} deals with \term{uniformity} and 
\term{atomicity}: all \term{row}s of a \term{record type} must contain
the same number of fields, and field values must not be decomposable into
smaller data items (\term{subfield}s). First normal form is often assumed
implicitly, but silently ignored at the same time. For instance a table
of publications and authors must contain exactly one author for each 
single publication to fulfill \acro{1NF}. To express publications without
author, we must either introduce a new table of all publications, or
a virtual `anonymous' author as \term{NULL} value.  Lists of
authors (like ``Kerningham and Ritchie'' in figure~\ref{fig:csvexample})
can be allowed by either adding another table with author-list
and list-member, or by allowing lists as atomic data types. The latter 
solution has been favoured by advocates of object-relational models
such as \textcite{Darwen1995}, as the notion of atomicity depends
on context \cite{Fotache2006}. However, the ad-hoc introduction of
\term{NULL} values and lists without any strict definition of their
meaning does neither align with \acro{1NF} nor with any other database 
formalism.%
\footnote{In practice, \term{NULL} values often cover a fuzzy bunch of
meanings, and lists are encoded by introduction of separators (for instance 
the string ` and '), that are not part of any database definition 
known to the \acro{DBMS}.}

Second and third normal forms\Acro*{2NF}\Acro*{3NF} ground on the concepts
of \Term{database key}s and \Term[functional dependency]{functional dependencies}.
Both can also be applied to other types of databases. A key is a set of
one or more fields (or table columns), which in their relation (or table) uniquely
identify a record. If relations are sets, every table has an implicit key, build 
of all their columns. Generally, keys should be short, to be used as reference
(also known as \term{foreign keys}) in other tables. Under \acro{2NF} and 
\acro{3NF}, all non-key fields must functionally depend on the combination of all
key fields. That means, the mathematical binary relations between all key-fields 
and any non-key field must be a total function (totality is enforced by exclusion
of NULL values with \acro{1NF}). In example~\ref{ex:dbnorm}, publications are
listed in table {\ormtext a} with country, author, year, title, publisher,
and place. If we choose the set $\{${\ormtext author}, {\ormtext year}$\}$ as
database key (red uniqueness overline in \ref{ex:dbnorm}), each combination 
of author and year must uniquely identify a 
publication's country, title, publisher, and place. Obviously, an author can 
create multiple titles per year, so we must modify the key. For instance, each
publication can get a {\ormtext letter} suffix between `a` to `z`.%
\footnote{Note that this only extends the number of publications per author 
and year from one to 26.} Such additions of
\term[artificial identifier]{artificial identifiers} are common,
although they do not represent a given fact about the publication.

The extended table {\ormtext b} now allows multiple publications per author
and year, but it may violate \acro{2NF}. This is the case, if a non-key field
depends on a subset of a key. If the non-key field {\ormtext country}
denotes the country, an author origins from,\footnote{For some reasons, 
nationalists and library classifications try to uniquely group authors under
countries.} it should  be split up in another table, with $\{${\ormtext author}, 
{\ormtext country}$\}$ as key (example~\ref{ex:dbnorm}, {\ormtext c}).

\Acro{3NF} is violated when a non-key field depends on another non-key field. 
For instance, if {\ormtext place} denotes the place of a publisher, and if places
do not change over the years, the table should be decomposed as shown in 
example~\ref{ex:dbnorm} {\ormtext d}.

\begin{example}
\begin{tikzpicture}[orm,b/.style={rectangle,draw},every orm line,
 ma/.style={matrix of nodes,text height=0.8em, text depth=0.25ex,anchor=west}]
\matrix(m1)[ma]{
a &|[b]| country   &|[b]| author    &|[b]| year 
  &|[b]| title     &|[b]| publisher &|[b]| place \\
};
\draw[constraint] ([yshift=1mm]m1-1-3.north west) to ([yshift=1mm]m1-1-4.north east);

\matrix(m2)[below=6mm of m1.south west,ma]{
b &|[b]| country   &|[b]| author    &|[b]| year  &|[b]| letter 
  &|[b]| title     &|[b]| publisher &|[b]| place \\
};
\draw[constraint] ([yshift=1mm]m2-1-3.north west) to ([yshift=1mm]m2-1-5.north east);

\matrix(m)[below=6mm of m2.south west,ma]{
c &|[b]| author      &|[b]| country
  &[4mm]|[b]| author &|[b]| year  &|[b]| letter 
  &|[b]| title       &|[b]| publisher &|[b]| place \\
};
\draw[constraint] ([yshift=1mm]m-1-2.north west) to ([yshift=1mm]m-1-3.north east);
\draw[constraint] ([yshift=1mm]m-1-4.north west) to ([yshift=1mm]m-1-6.north east);

\matrix(m)[below=6mm of m.south west,ma]{
d &|[b]| author      &|[b]| country 
  &[4mm]|[b]| author &|[b]| year  &|[b]| letter 
  &|[b]| title       &|[b]| publisher 
  &[4mm]|[b]| publisher   &|[b]| place \\
};
\draw[constraint] ([yshift=1mm]m-1-2.north west) to ([yshift=1mm]m-1-3.north east);
\draw[constraint] ([yshift=1mm]m-1-4.north west) to ([yshift=1mm]m-1-6.north east);
\draw[constraint] ([yshift=1mm]m-1-9.north west) to ([yshift=1mm]m-1-10.north east);

\matrix(m)[below=6mm of m.south west,ma]{
e &|[b]| author &|[b]| language &|[b]| award \\
};
\draw[constraint] ([yshift=1mm]m-1-2.north west) to ([yshift=1mm]m-1-4.north east);

\matrix(m)[below=6mm of m.south west,ma]{
f &|[b]| author &|[b]| language &[4mm]|[b]| author &|[b]| award \\
};
\draw[constraint] ([yshift=1mm]m-1-2.north west) to ([yshift=1mm]m-1-3.north east);
\draw[constraint] ([yshift=1mm]m-1-4.north west) to ([yshift=1mm]m-1-5.north east);

%\matrix(m)[below=6mm of m.south west,ma]{
%g &|[b]| author &|[b]| year &|[b]| letter &|[b]| title &|[b]| publisher &|[b]| language
%  &[4mm]|[b]| author &|[b]| year  &|[b]| letter &|[b]| award \\
%};
%\draw[constraint] ([yshift=1mm]m-1-2.north west) to ([yshift=1mm]m-1-4.north east);
%\draw[constraint] ([yshift=1mm]m-1-8.north west) to ([yshift=1mm]m-1-10.north east);
\end{tikzpicture}

\caption{Normalization of relational database tables}
\label{ex:dbnorm}
\end{example}

\acro{4NF} and additional normal forms attempt to minimize the number of fields
involved in a key. Let us assume we want to store information about the popularity
of authors. Table {\ormtext e} in example~\ref{ex:dbnorm} list languages, that
authors have been published in, and awards, that authors have received. Under
forth normal form, the implicit key $\{${\ormtext author}, {\ormtext language}, 
{\ormtext award}$\}$ should be split in two independent keys (tables~{\ormtext f}),
because languages and awards are independent from each other. If the awards and
languages are properties of publications, a normalized database should not
contain tables {\ormtext e} and {\ormtext f} at all, because their content can
be derived by \term{relational algebra} as view from other tables.

Despite the theoretical importance of database normalization, it has failed to
become an ultimate aid for database designers. At most, they use normalization
for validation after creation of databases based on intuition.
\textcite{Fotache2006} gives some reasons for this gap: popular textbooks on
database design often describe and exemplify normalization poorly or even
incorrectly. Most literature on normalization focuses on rigor and sober
mathematics and uses artificial examples, instead of real world applications.
There is a lack of graphical diagramming tools for functional dependencies, and
other graphical modeling languages such as \acro{ERM} (section~\ref{sec:erm})
base on different philosophies. Another problem, also raised by
\textcite{Kent1983}, is the dependency on uniquely identified entities. If
fields contain different values for the same object, for instance different
spellings of names, speaking about keys and dependencies is futile. However
normalization only reveals the difference between rigor data and fuzzy reality,
that would also exist without it.

% Even with fully normalized relations, you do not necesarrily get unique
% representations \cite{Kent1988} (or 83?)

% ``Semantic'' interpretation of the relational model: 
% each tuple represents a particular proposition based on that predicate
% constraints (types are constraints)

% relationships are generally m:n and generated on the fly by join operations.
% relational tables similar to kett-records

\subsection{Object database}
\label{sec:oodbms}

\Tacro{Object Orientation}{OO} was introduced during the 1960s by
\Person[Ole-Johan]{Dahl} and \Person[Kristen]{Nygaard} with creation 
of the \term[Simula (programming language)]{Simula programming language} 
\cite{Holmevik1994}. \Tacro{Object Oriented Programming}{OOP} was
further popularized by \Person[Alan]{Kay}'s \term[Smalltalk (programming
language)]{Smalltalk} and had large impact on many following programming
languages. Core concepts of different \acro{OO} systems have been identified
retrospectively by \textcite{Armstrong2006}. The main idea is to bundle data 
and interactions in \term{object}s. An \Term{object} is a structure with data
fields, attributes, or properties; and methods to invoke a specific
behaviour. Both are combined in a \Term{class} that, like a record type, acts
as blueprint to create objects, which then are \Term{instance}s of the class.
In addition, the properties and methods of a class may be included or used as
basis for another class via \Term{inheritance}. Details of implementation are
concealed by the object via \Term{encapsulation}.

When \acro{OOP} became widespread in the 1980s, the mismatch between
object-oriented programs and relational databases led to the development
of object(-oriented) databases. The classical definition of an \Tacro{object
oriented database management systems}{OODBMS} in the `Object-Oriented Database
System Manifesto' \cite{Atkinson1989}.  It identifies thirteen \acro{OODBMS}
requirements, five of which hold for \acro{DBMS} in general (persistence,
secondary storage management, concurrency, recovery, ad-hoc query facility),
and eight of which are specific for object-oriented systems. 

\begin{itemize}

\item objects must be identifiable independent of their values. This implies
two samenesses: same valued objects (\term{equivalence}), and same
objects (\term{identity}).

\item objects are defined only by which operations they can be modified and
interact with. Internal representations are hidden (\term{encapsulation}).

\item objects must have classes or types that define their characteristics.

\item objects can be build from some basic types (integers, floats, characters,
byte strings, booleans, etc.) and some object constructors (at least set, list,
and tuple), that can be applied to any object, independent from its type.

\item there is no distinction in usage between predefined basic types and
newly constructed types.

\item there are at least four types of inheritance for a class $T$
and a subclass $S$: substitution (instances of $S$ can substitute instances
of $T$),\footnote{This type of inheritance has formally been refined by
\textcite{Liskov1987} and is known as \term{Liskov substitution principle}. It
states that all provable properties of instances of $T$ must also be true for
instances of $S$. Thus any algorithm designed for $T$ instances of will
behave exactly the same if used with $S$ instances.} specialization (instances
of $S$ contain more information than instances of $T$), inclusion, and
constraint (every instance of $S$ is also instance of $T$ if
it satisfies a given constraint).

\item names can denote different operations for different object types.

\item the \term{data manipulation language} must be a Turing-complete computational
programming language.

\end{itemize}

In summary, the characteristics of object databases are
\textit{i)} object identity, \textit{ii)} classes
and types, and \textit{iii)} inheritance. It must be said that, `object
oriented', like many terms in computing, is also used as marketing buzzword,
or to indicate a rough direction, rather then specifying a final list of
features. The list of requirements given above is rarely fulfilled in existing
\acro{OODBMS}. \textcite{Stonebraker1990}, in response to
\textcite{Atkinson1989}, published the `Third-Generation Database System
Manifesto' and argued that \acro{OO}
can be added to relational systems with keeping \acro{SQL} as common database
language. Some \acro{OODBMS} concepts were incorporated into \acro{RDBMS},
which led to \Tacro{object-relational databases}{ORDBMS}. However the
predominance of relational databases persisted and therefore developers created
\Term{Object-relational mappings} to combine \acro{OOP} and \acro{RDBMS}. 
This error-prone task has been described by \textcite{Neward2006} as 
`Vietnam of Computer Science': it ``starts well, gets more complicated as time
passes, and before long entraps its users in a commitment that has no clear
demarcation point, no clear win conditions, and no clear exit strategy.''
Above all, migration to and from object databases is costly because the
dominance of \acro{SQL} and the lack of an accepted query language for object
databases, independent from type systems of specific programming languages.
A unification of the relational model and the object orientation has been
proposed by \textcite{Darwen1995} in a `Third Manifesto'. They made clear
that progress is hindered by adherence to the \acro{SQL} database language,
which is not even truly relational, but their proposed alternative
\Term{Tutorial D} \cite{Date2006} had little impact on concrete implementations.
Today the \term{NoSQL} movement shares some of the critics on \acro{SQL} and
object databases are sometimes subsumed under the NoSQL approach. Maybe
the most effective impact of \acro{OO} to data modeling is the
\tacro{Unified Modeling Language}{UML} (see section \ref{sec:uml}). It is
mostly used to model object oriented software but also for databases.

% Examples of object databses: db4o, Perst, Caché ...

\subsection{NoSQL databases}
\label{sec:nosql}
During the first decade of 21st century several providers of large web
applications (\term{Google}, \term{Amazon}, \term{Facebook}, \term{Yahoo},
etc.) had started to develop their own non-relational data stores. Database
researcher \textcite{Stonebraker2007} declared the relational \acro{DBMS}
obsolete, because of changed hardware limitations and data processing needs.

When more new non-relational open source \acro{DBMS}es 
were available in 2009 \term{last.fm} employee \Person[Johan]{Oskarsson}
organized an event called NOSQL (for `not SQL'), which can be seen as baptism of
the following NoSQL movement. NoSQL has brought a new momentum into the development
of database systems by questioning some general properties of \acro{RDBMS},
such as secondary storage management, concurrency, recovery, schemas, and ad-hoc
query facility. The term refers to no common database model but compasses all
non-relational or `structured storage' data stores, which may also subsume file
systems. Beside object databases the following types can be distinguished:

\begin{description}

\item[key-value stores] provide a simple dictionary where keys are mapped
to arbitrary values. These systems are more comparable with flat file
systems that have no data types and may only put restrictions on keys.
An exception is \term{redis},\footnote{\url{http://code.google.com/p/redis/}}
a \acro{DBMS} that supports strings (sequences of bytes), sequences,
sets, and ordered sets of strings. Other examples of key-value stores are 
\term{Amazon~S3} \cite{S3DevGuide}, 
\term{Berkeley DB}\footnote{\url{http://www.oracle.com/database/berkeley-db/}},
and \term{Project Voldemort}.\footnote{\url{http://project-voldemort.com/}}

\item[document databases] or document stores manage values as `documents' in a
specific data structuring language (mostly \acro{JSON} or \acro{XML}). In
general, documents are not constrained by a schema, and they may be versioned.
Popular examples of document databases are
\term{CouchDB},\footnote{\url{http://couchdb.apache.org/}}
\term{MongoDB},\footnote{\url{http://www.mongodb.org/}} and
\term{RavenDB}.\footnote{\url{http://ravendb.net/}} Native
\acro{XML} databases, such as
\term{eXist}\footnote{\url{http://www.exist-db.org/}} are less 
often mentioned, but also fall into this category.

\item[graph databases] allow storing arbitrary graph structures with
nodes, edges, and properties. In most cases, graphs are schema-free without
distinction between different structural kinds of relationships (1:n, 1:1, m:n,
mandatory, recursive, unique, etc.). The specific graph model depends on the
particular graph \acro{DBMS} \cite{Angles2008,Rodriguez2010}. Popular instances
are based on the \acro{RDF}~graph~model for \term{triple store}s (see~\ref{sec:rdf})
or on \term{property graph}s, but there also exist other
models, for instance hypergraphs.\footnote{See
\url{http://www.kobrix.com/hgdb.jsp} for a Hypergraph \acro{DBMS}.} Network
databases can be seen as a restricted subset of graph databases. Examples of
general graph databases include
\term{Neo4J},\footnote{\url{http://www.neo4j.org/}} and
\term{InfoGrid}.\footnote{\url{http://www.infogrid.org/}}
\Term*{graph database model}\index{graph!database model}

\item[column databases] are table-based databases, which seperately store 
each column of all relation (or each field of all records), instead of
keeping together rows or records. Column databases provide good performance
especially for processing of sparse data. The most prominent column 
\acro{DBMS} is Google's \term{BigTable} \cite{Chang2006} (also supporting
versioning), another instance is \term{Cassandra} (Facebook, now Apache)%
\footnote{\url{http://cassandra.apache.org/}}. Similar column data-structures
are also used in databases and other software for statistical analysis of
large data sets.

\end{description}

