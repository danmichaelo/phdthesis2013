\section{Library and information science}
\label{sec:lis}

%\begin{quotation}%
%It takes a good system and a lot of work to keep track of books. [\ldots] \\
%Books hold our universe, past, present, and future, and other universes, too.
%\\ \quotationsource \Person[William]{Kent} \cite{Kent1980}
%\end{quotation}

\noindent The following section gives a short overview of \term{library and
information science} and some of its core concepts relevant to this thesis.  The
two disciplines library science and information science are deeply connected
and often combined. A major reason for their split up is a lack of attention
to technical aspects by librarians and library scientists --- a blind spot that
only begins to diminish since the 1970s \cite{Buckland1998b}.  Actually both
disciplines deal with digital documents and metadata from slightly different
viewpoints, so I will treat them as one discipline.

\subsection{Background of the discipline}

One could argue that library science has been practiced since the first great
ancient libraries, but as serious disciplines it originates in the 19th
century. The first explicit \Term{library science} textbook was published by
\Person[Martin]{Schrettinger} (\citeyear{Schrettinger1808}), followed by others
during the next decades.  At the turn of the century, the development of
library science was most advanced by two men: \Person[Melvil]{Dewey} created
the decimal classification system (\citeyear{Dewey1876}), and promoted the card
catalog.  \Person[Shiyali Ramamrita]{Ranganathan} is best known for his five
laws of library science (\citeyear{Ranganathan1931}) and his invention of
\term{facetted classification}.\footnote{Ideas of facets can also be found
earlier, for instance Schrettinger consideres a theoretical system of parallel
classifications in his work's second volume \cite[p. 85]{Schrettinger1829}.}
\Term{Information science} evolved in parallel with the creation of scientific
indexes. Up to the 18th century, bibliographies and catalogues mostly included
single books, but no scientific journals or articles. As described by
\textcite{Kronick1962} the primary function of scientific journals was that of
providing a vehicle for the dissemination of information rather than a
repository for the storage of new scientific ideas. When libraries concerned
themselves with preservation, organization and access to physical holdings,
scientists organized the overview of research on their own.\footnote{This
schema repeats regularly: for instance the first digital repositories were not
created by librarians.} As in the 19th century the number of articles in
scientific and technical journals increased, abstracts journals were created to
summarize and review new articles and facts.  Starting with the
`Pharmaceutisches Centralblatt' in 1830, dozens of periodical indexes cataloged
scientific literature in their subfields.  In contrast to library catalogues,
these indexes not only included single articles, but they also collected
references to documents independent from physical access and ownership. The
most ambitious indexing project was the \Term{Universal Bibliographic
Repertory}, founded in 1895 by \Person[Paul]{Otlet} and \Person[Henri]{La
Fontaine}.  Influenced by works of \person[Melvil]{Dewey}, they created the
first universal bibliographic database, and founded modern information science,
which was then called \Term{documentation} \cite{Rayward1997}. The works and
ideas of \person[Paul]{Otlet} are remarkable in several ways: As described by
\textcite{Rayward1994}, Otlet anticipated later ideas of \term{knowledge
organization}, such as those of \textcite{Wells1938}, \textcite{Bush1945},
\textcite{Engelbart1963}, \textcite{Nelson1965,Nelson1981}, and
\person[Tim]{Berners-Lee} (\citeyear{BernersLee1989,BernersLee2001a}). After
World War I, and with rise of English as predominant scientific language, most
of the works of Otlet, \person[Wilhelm]{Ostwald} and other researchers
fell into oblivion. Two aspects of Otlet's work have stayed
characteristic for documentation and later information science: the proactive
use of technology, and the central role of document concepts.

While libraries regularly hesitate to use new technologies, information science
at best is actively involved in development, and propagation of technology to
ease and automize the organization of information. Otlet systematically used
reproducible index cards (predating digital records), pioneered the use of
microfilm, and envisioned networked multi-media machines decades before their
electronic realization \cite{Otlet1990,Otlet1934}. In the 1950s and 1960s
\Term{information retrieval} evolved as major branch of information science
with important contributions from \Person[Calvin]{Mooers},
\Person[Eugene]{Garfield}, and \Person[Gerard]{Salton}, among others.  The
development was driven by the exponential growth of publications
\cite{Price1963} and motivated by computerized automatization, which promises
to speed up and improve the process of finding relevant information.  However,
automatic systems tend to ignore human factors, such as motivation and
ethics\footnote{This can be exemplified by \term{Mooers' law}, in which
	\Person[Calvin]{Mooers} (\citeyear{Mooers1960}) observes that ``many people
	may not want information, and they will avoid using a system precisely
	because it gives them information [\ldots] not having and not using
	information can often lead to less trouble and pain than having and using
	it''.  For other neglected aspects see the works of
\Person[Joseph]{Weizenbaum} (e.g.  \citeyear{Weizenbaum1976}).} A closer look
on computer science paradigmas, especially artificial intelligence and its
recurrences, also reveals suprisingly positivist images of knowledge.
Meanwhile information science has lead to more specialized but
interdisciplinary sub-disciplines like \term{information systems},
\term{information architecture}, and \term{information ethics}. The core
concept of information --- see \textcite{Capurro2003} and \textcite{Ma2012} for
analysis --- still has impact on the transformation of library and information
science. For instance in 2006 the newest faculty at University of Berkeley,
originating in library and information science, was simply renamed the `School
of Information'.\footnote{See \url{http://www.ischool.berkeley.edu/about/}.}

\subsection{Documents}
\label{sub:documents}

Despite the trend on information, the roots of library and information science
are located in the organization of collected \Term[document]{documents}. Since
the 1990s one can identify a renaissance of the document approach with
independent schools of though. The Kopenhagen school of document theory can be
found in contributions by \textcite{Lund2009}, \textcite{Hjorland2007},
\textcite{Orom2007} and other papers collected by
\textcite{Skare2007}.\footnote{See also \url{http://thedocumentacademy.org}.}
The French school of thought is most visible by publications of \Person[Roger
T.]{Pédauque}
(\citeyear{Pedauque2003,Pedauque2006,Pedauque2007,Pedauque2011}), a
group of scholars publishing under common pseudonym. English introductions to
their discource have been given by \textcite{Truex2007} and
\textcite{Gradmann2008}.  Despite the importance of documents as core concept
of library and information science, there is no commonly agreed upon
definition. While libraries tend to define documents based on physical
entities --- the most prominent instance of a document is a book ---
information science tends to abstract documents from their form. This focus
results from research on aspects of preservation and access to original
documents on the one hand compared to research on aspects of document
descriptions and connections on the other. With the shift to digital documents
it is more difficult to use form as defining criterion because traditional
concepts such as `page` and `edition` loose meaning. For this reason
\textcite{Buckland1997,Buckland1998} argues to define documents in terms of
function rather than form. This idea had already been brough up by
\textcite{Briet1951} before the advent of digital documents.  Eventually any
entity --- that is any sequence of bits in the digital world --- can act as
document. To be a document it must be ''conserv\'e ou enregistr\'e, aux fins de
repr\'esenter, de reconstituer ou de prouver un ph\'enom\`ene ou physique ou
intellectuel'' \cite[p. 7]{Briet1951}.\footnote{Translated by
\textcite{Buckland1997} as ``preserved or recorded, intended to represent, to
reconstruct, or to demonstrate a physical or conceptual phenomenon''.} This
implies two important properties of documents: first document must be recorded,
and second they must refer to something. The document's referent is also called
its \Term{content}.\footnote{From a semiotic point of view the relation between
a document and its referent is more complex. See section~\ref{sec:semiotics}
and \textcite{Brier2008,Brier2006} for details.} As described by
\textcite{Yeo2010}, the content is not necessarily fixed and known, but it
highly depends on context.  The property of being recorded distinguishes
digital documents in library and information science from more general data
objects, for instance databases: digital documents do not change. Even
``dynamic documents'' are fixed as soon as you package them in some form
suitable for storage. \textcite{Renear2009} have carried this argument to the
extreme: either a change constitutes a new document or it is not relevant
enough to be recorded.  A document is created to persist as fixed snapshot,
while other data objects can also be created to capture the current state of a
dynamic system.  This distinction can be exemplified by a library information
system that manages both, stable digital documents, and dynamic data about
users and access to documents.

The traditional role of a library is the collection of separated documents.
With increase in aggregated documents which combine independent smaller parts,
such as encyclopaedia and journals that hold single articles, library
institutions need to divide documents into separate conceptual units. For this
purpose Paul Otlet introduced the \Term{monographic principle} in 1918 and
applied this new document concept in the \term{Universal Bibliographic
Repertory} \cite{Rayward1994,Otlet1990}:\footnote{As summarized by
\textcite{Hapke1999} the monographic principle can also be traced back to the
project ``Die Br\"ucke'' founded in 1911 by \Person[Wilhelm]{Ostwald}.} the
idea was to ``detach what the book amalgamates, to reduce all that is complex
to its elements and to devote a page to each.`` \cite[cited and translated by
Rayward]{Otlet1918}.  The monographic principle requires methods to extract
individual pieces of information from documents which can then be used to
create new documents. With hypertext and new document types like blog articles
and tweets, the creation of monographic documents experiences a revival. In the
Semantic Web community the idea is beeing reinvented as `nano-publications'
\cite{Groth2010,Mons2009}.

\subsection{Metadata}
\label{sub:metadata}

More than documents as such, library and information science is interested in
their description, that is bibliography (from the Greek $\beta \imath \beta
\lambda \imath o \gamma \rho \alpha \varphi \acute{\imath} \alpha$ for
`[de]scription of books'). Applied to digital documents, all bibliographic data
is \Term{metadata}. This term became popular in library and information
science, during the 1990s. Meanwhile, metadata subsumes any information about
digital and non-digital content, including traditional library
catalogs.\footnote{\textcite{Shelley1995} according to \textcite{Caplan2003}
state NASA's \tacro{Directory Interchange Format}{DIF} as the first standard
that defines metadata in its modern sense \cite[F-10]{DIF1988}. The definition
in this standard includes any information describing a data set, including user
guides and schemas. One of the first specifications named metadata was the
\tacro{Content Standard for Digital Spatial Metadata}{CSDSM} \cite{CSDSM1994}.}
Before this, the term metadata had been introduced in computing by
\textcite{Bagley1968}\footnote{\textcite{Solntseff1974} refer to ''The notion
of 'metadata` introduced by Bagley``, citing \textcite{Bagley1968}, but I could
not get a copy of his report because of copyright restrictions. Philip Bagley
is among the forgotten forefathers of library and information science, also
because he created the very first analysis of possibles uses of an existing
computer, the \Term{Whirlwind I}, for document retrieval \cite{Bagley1951}.}
but it was only used casually for management data in databases and programming
languages.  With rise of the Web, its meaning shifted from data about data sets
and computer files to data about online resources. Finally, metadata became
popular with the creation and promotion of the \tacro{Dublin Core Metadata
Element Set}{DCMES}.

Similar to documents, metadata can best be defined based on its function.
\textcite{Coyle2010} describes metadata as something constructed, constructive
and actionable. As a result, there is no strict distinction between data and
metadata but the use of data as metadata depends on context: a digital record
can both be a plain data object, a digital document, and a piece of metadata,
even with different content in different usage scenarios. The relevance of
usage for metadata distinguishes metadata from traditional cataloguing, as
described by \textcite{Gradmann1998}: traditional bibliographic records were
created mainly to describe a document with a very limited context of usage.
Gradmann argues that metadata ``are intended to be part of a usage context
different than that of cataloguing records, and that they are technically
linked to this context to a very high degree.'' The important role of a
technical infrastructure which metadata is used in, requires an analysis of
the infrastructure as carried out in chapter~\ref{ch:methods} of this thesis.

A major application of metadata and a growing branch of library and information
science is the (long-term) preservation of digital documents. Long-term
preservation provides two general strategies to cope with the rapid change and
decay of technologies: either you need to emulate the environment of digital
objects or you must regularly migrate them to other environments and formats.
Both strategies require good descriptions of the data and environment to be
archived.  When time passes, the descriptions themselves become subject of
preservation. By this, digital documents may get buried in nested layers of
metadata or they may become migrations of migrations as shadows of the original
documents. Knowledge of general patterns in data and metadata could help to
reveal data by ``data archaeology'' also when long-term preservation has failed
(see section \ref{sec:dataarchaeology} in the outlook). In other applications
but preservation, metadata is difficult to work with, because it is aggregated
from heterogeneous sources with different structures than expected
\cite{Tennant2004,Thomale2010}.  Nevertheless, existing metadata research
provides useful some guidelines and tools to achieve interoperability even
among applicatons with different usage context: Metadata registries collect and
describe standards, metadata crosswalks provide mappings, and metadata
application profiles allow for customization without loosing a general
consensus how data should be structured.  The vast diversity of metadata
standards and formats, which are defined and evaluated in library and
information science, shows both the need for metadata and its complexity.
A broad overview of the large number of metadata formats and specifications 
in the cultural heritage sector is provided by \textcite{Riley2010}.

%for instance \acro{METS}, \acro{PREMIS} and \acro{LMER} for preservation
%TODO: metadata is also constitutional!

% However, in practice a lot of
% manual work is needed to make use of metadata, because context and function 
% are not fully known or creators of data just do not comply to assumed standards.

%the description of digital documents in form
%of other digital documents, that is the relation of data to other data. 
%...This
%highlights the importance of descriptive metadata to put data in context,
%but it does not eliminate the need to actually look at data at some level
%of description. 

%library focused formats and models like \term{Dublin Core}, \acro{MARC},
%\acro{OAI-ORE} and \acro{CIDOC-CRM} and some additional motivation
%why my topic is located in the information sciences.

%As any piece of data can act as document, the definition of a document is 
%either passed on to the eye of the beholder or to the level of metadata 
%about the document

% \item What is a document?
% \item What is metadata?
% \item How is metadata created and managed in practice?
% \item Which metadata standards and formats exist?

% Weaver: statistisc ``it is very tempting to say that a book written in
% Chinese is simply a book written in English which was coded into the `Chinese
% code.''' (Weaver) 

% early theory of metadata, that shaped RDF: \cite{Guha1997}

\ignore{
\subsection{Additional concepts [+?]}

\Term{information} and \Term{knowledge} are also core concepts in library
and information science, but their definition is out of the scope of this
work. Reviews of discussions about the relation between data, information, 
and knowledge are provided by \textcite{Zins2007} and \textcite{Gray2003}. 


facets, aspects or 

% TODO: "sorting things out"
As shown by \textcite{Bowker1999} on classifications, the creation of
\term{knowledge organization} systems is no neutral act, but these systems are
artificial and inherently discriminating because of hidden social assumptions.


\TODO{
I am not quite sure whether some additional concepts from library and 
information science should be explained. At least classification theory
and knowledge organization might provide some relevant ideas such as
descriptor, authority control, classification, facets, and 
(general and syntactic) indexing.
}
% subject indexing and authority control reinvented: \cite{Voss2007}.
}


