% I bet in 40 years
% we will complain about some ancient XML or RDF data that contains strange
% artifacts. Of course MARC is much worse because it is so old, designed for
% cataloging records. But try to get information out of other decade-old
% databases that have been designed for a specific use-case that is not yours,
% and you will stumble upon similar problems.

\section{Deconstruction of a MARC record}
\label{appendixD}

An application of the results of this thesis shall briefly be illustrated with
a fragment of a bibliographic record (figure~\ref{fig:marcrecord}). Similar
analyses of \acro{MARC} have been given by \textcite{Thomale2010} and by
\textcite{Coyle2011}.  The \Tacro{Machine-Readable Cataloging}{MARC} standard
was developed during the 1960s to support library automation in general and to
exchange bibliographic descriptions in particular
\cite{McCallum2009,McCallum2002,Avram1975}.  \acro{MARC} origins in pre-digital
data in form of physical catalog cards --- the format is also criticized for
being suitable only for printing these cards
\cite{Tennant2002,Coyle2005}.\footnote{Sure today's formats will be criticized
in 40 years for not being suitable then.} Nevertheless \acro{MARC} is still
used widely among library systems today .  The brief analysis of the sample
record consists of three steps: first, one needs to clarify the main purpose of
\acro{MARC} to find out what a record actually is.  This is done by means of
the prototype categorization identified in section~\ref{sec:categorization}.
Second, one should ask which basic paradigms have influenced the record
(section~\ref{sec:paradigms}).  And third, one can identify data patterns in
the record (chapter~\ref{ch:patterns}).
% and table~\ref{tab:patternclassification}).

%According to the official \acro{MARC} website \cite{LoC2012}, the ``MARC 21
%Format for Bibliographic Data'' is a standard ``widely used standard{[}s{]}
%for the representation and exchange of bibliographic {[}\ldots{]} information
%data in machine-readable form'' and ``a carrier for bibliographic
%information''.  

On a closer look, \acro{MARC} consists of three methods \cite{LoC2012}: its
\emph{record structure} is used as general data structuring and markup
language, the \emph{content designation} is based on a rough conceptual model
of bibliographic entities (e.g. titles and physical properties), and the actual
\emph{content} of data elements is constrained by cataloging rules
(\acro{ISBD}, \acro{AACR}, \acro{RAK}, \ldots). As neither model nor rules
are defined in a formal language, and many different \acro{MARC} variants and 
interpretations exist, the main use of \acro{MARC} is limited to a
basic record structure (section~\ref{sec:records}), similar to methods
described in section~\ref{sec:dsl} and \ref{sec:markuplanguages}.\footnote{Even
the basic structure cannot be taken for granted: in 2004 German and
Austrian libraries decided to adopt MARC, but they introduced an invalid
subfield code (\texttt{A}), making some of their records broken \acro{MARC}. Such
violating interpretations also occur at schema and conceptual levels.}
Figure~\ref{fig:marcflatmodel} shows a possible model of this structure:
parts may be ordered (\pattern{sequence} pattern) or indexed
(\pattern{identifier} pattern).

\begin{figure}[ht]
\flushright
\begin{tikzpicture}[decoration={brace}]
\matrix (marc) [datamatrix] {
  100 1\#  & |[ucs]| 1F & a & Kernighan, Brian W.         & |[ucs]| 1E ~~ \\
  245 14   & |[ucs]| 1F & a & The C programming language. & |[ucs]| 1E \\
  260 \#\# & |[ucs]| 1F & a & Englewood Cliffs, NJ :      & ~~ ~~\\
           & |[ucs]| 1F & b & Prentice-Hall,              & \\
           & |[ucs]| 1F & c & 1978.                       & |[ucs]| 1E ~~ \\
  700 1\#  & |[ucs]| 1F & a & Ritchie, Dennis M.          & |[ucs]| 1E 1D \\
};
\draw[decorate] 
(marc-6-1.south east) -- node[anchor=north,inner sep=2mm,align=center]
  {field names\\(``tags'')}
(marc-6-1.south west); 
\draw[decorate]
(marc-6-5.south west) -- node[anchor=north,inner sep=2mm,align=center]
  {subfield values} (marc-6-4.south west); 
\draw[decorate] (marc-3-5.north east) -- node[anchor=west,inner sep=2mm]
 {field} (marc-5-5.south east);
\draw[decorate] ($(marc-1-5.north east)+(1cm,0)$) -- 
  node[anchor=west,inner sep=2mm] {record} ($(marc-6-5.south east)+(1cm,0)$);
\draw[<-] (marc-6-3.south) |- +(1em,-2.6em) node[anchor=west] 
{subfield codes as (repeatable) subfield indices};

\end{tikzpicture}
\caption{MARC record and flat file database model with subfields}
% TODO: arrow to "record separator" and "field separator"
\label{fig:marcrecord}
\end{figure}

The governing paradigm of \acro{MARC} is the paradigm of standards and rules
(section~\ref{sec:standardsrules}), so this paradigm can reveal most defects of
the format. It is worth remarking that \acro{MARC} is neither specified by a
formal language nor does it come with a schema language to express subsets and
applications of \acro{MARC} (\acro{MARCXML}, an encoding of \acro{MARC} in
\acro{XML}, only defines a schema for the basic record structure but not for
particular data elements). Furthermore there is no official validator to check
whether records conform to (a specific dialect of) \acro{MARC}.

The lack of formal specifications and automatic tools for validation increase
the importance of intellectual analysis of \acro{MARC} records. Many actual
data patterns do not simply follow the basic record structure of MARC. In
particular \textcite{Thomale2010} found that ``the underlying structure is
based on linguistics rather than a format that was designed to be
machine-readable'', so \acro{MARC} should better be treated like textual
markup. The interpretation of records as markup, which is normally based on
element order (\pattern{sequence}) contrasts with the requirement select data
elements based on the field-subfield-structure (\pattern{identifier} pattern).
For instance one could combine tag, indicator, and subfield code to a
normalized pointer, such as \texttt{245 14 a} for the title in
figure~\ref{fig:marcrecord}.  Within \acro{MARC} fields, \textcite{Coyle2011}
identified three pattern structures: first, subfields can indepedently and
directly describe a resource (so they can be used as part of a pointer).
Second, subfields can qualify or modify other subfields (\pattern{dependence}
pattern or \pattern{flag} pattern), and third, multiple subfields can together
form a resource description (\pattern{sequence}, \pattern{container}, or
\pattern{embedding} pattern). 

An in-depth analysis of \acro{MARC} in particular is out of the scope of this
work, so this appendix ends with some additional pattern instances from the
sample record:

% \texttt{1001\#$\ddagger$a} (identifier): Kernighan, Brian W.

\begin{itemize}
\item The fields 100 and 700 form a \pattern{sequence} of authors.
\item Author names are structured by an \pattern{embedding} with comma as 
  \pattern{separator} (surname, given). Second given names are further abbreviated
  (\pattern{etcetera}).
\item Several instances of punctuation are irrelevant (\pattern{garbage} pattern).
\item \texttt{NJ} in `Englewood Cliffs, NJ' is an \pattern{identifier} that refers 
	to New Jersey.
\item Core elements (`Brian', `Prentice-Hall', \ldots) are instances of
	the \pattern{label} pattern.
\end{itemize}


% time-consuming, intellectual analysis

% it can be encoded in other forms without
% adding conceptual value (for instance in \acro{XML} as \acro{MARCXML}. See
% figure~\ref{fig:jsonrdfxml} for an illustration of similar encodings). A
% \acro{MARC} record is an instance of ISO~2709 (\citeyear{ISO2709:2008}),
% similar to \acro{PICA} (example~\ref{ex:picafieldids}).  In short, a record
% consists of a header (not included in this example) and a list of data fields.
% Each field consists of a field tag, which consists of three digits, an
% indicator, which consists of two lowercase alphanumerical characters or spaces,
% and a list of subfields, each with an alphanumeric subfield code character and
% the actual subfield value.


% conceptual background is rather low, contains schema but many different
% variants. loosely coupled

% Few people have actually read the specification

% Even if there is a standard, people will prefer not to read it, but learn a 
% structure from examples and from the effect of data, when used with a specific
% application --- for instance bibliographic data is primarily shaped not 
% to describe documents, but to nicely print cataloging cards or to trigger 
% the display in electronic catalogs \cite{Tennant2002,Coyle2005,McCallum2009}.
% In practice these applications act as validators of their own interpretation
% of the standard they seem to implement. 

% For practical application: \cite{Thomale2010}
% General LIS: Encyclopedia of Library and Information Sciences


% The record includes punctation from the \tacro{International Standard Bibliographic Description}{ISBD}.

% Schemas (SQL, XML Schema, OWL...) only describe what you
% called "explicit" structure, but as people create data, they add "implicit"
% structure, based on additional conventions or ad-hoc rules.

\begin{figure}
\centering
\begin{tikzpicture}[orm,lpin/.style={label distance=0mm},
  nmrole/.style={roles,unique=2,label=[lpin]above:\ormind{*}}]
\entity (file) at (0,-1) {File};
\entity[right=1.5 of file] (record) {Record} 
       edge[mandatory] node[nmrole] {} (file);
\entity[right=1.5 of record] (field) {Field} 
       edge[mandatory] node[nmrole] {} (record);
\entity[right=1.5 of field] (subfield) {Subfield} 
       edge[mandatory] node[nmrole] {} (field);
\value[right=1.5 of subfield] {Value} 
       edge[mandatory] node[roles,unique] {} (subfield);

\node[rule=*,below=1mm of file.south west,anchor=north west]
  {indexed and/or ordered};
\end{tikzpicture}
\caption{Flat file record model of MARC}
\label{fig:marcflatmodel}
\end{figure}
