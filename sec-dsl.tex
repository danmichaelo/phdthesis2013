\section{Data structuring languages}
\label{sec:dsl}

\Tacro[data structuring language]{Data structuring languages}{DSL}
or \Term[data serialization language|see{data structuring language}]{data
serialization languages} are used to express, exchange, and store
data structured in general forms such as records, lists, sets, and
tables. Similar to most file systems (\ref{sec:filesystems}) and 
databases (\ref{sec:databases}), and unlike specific markup languages 
(\ref{sec:markuplanguages}) the elements of a \acro{DSL} do not
hold special semantics but general patterns and constraints. These
constraints may further be tightened by schemas (\ref{sec:schemas})
that define concrete formats based on a particular \acro{DSL}.
Data that is only structured by a \acro{DSL}, but not by a more
specific schema is often denoted as \Term{semi-structured data}.

Each \acro{DSL} defines a simple \term{type system} and at least one syntax to
serialize data in form of a stream of characters or bytes. The type system can
be seen as (conceptual) data model of the \acro{DSL} and the syntax as logical
model of the \acro{DSL}. Some \acro{DSL}s provide a syntax and a clear
definition of its data model (\acro{XML}, \acro{RDF}, \acro{YAML}).  Others
only define a syntax, that implies a model (\acro{JSON}) or they do not define
a strict standard at all (\acro{INI}, \acro{CSV},
\acro{S-EXP}%,\acro{ZZSTRUCT}). This section will describe some popular data
structuring languages with focus on their underlying data model: \acro{CSV} and
\acro{INI} (\ref{sec:csvandini}), \acro{JSON} (\ref{sec:json}), \acro{YAML}
(\ref{sec:yaml}), and \acro{XML} (\ref{sec:xml}) all provide a syntax that is
also human-readable to some degree. The focus of \acro{RDF} (\ref{sec:rdf}) is
more communication between machines. Depending on what one considers as core
part of \acro{RDF}, it can also be seen as simple conceptual modeling language
(section~\ref{sec:modelangs}).
% \acro{ZZSTRUCT} (\ref{sec:zzstructure}) is a less known, theoretical
% \acro{DSL}, developed for the \term{Xanadu} hypertext project. 
If one removes all executable parts from a \term{programming language}, its
\term{type system} can also be seen as \acro{DSL} -- a popular example is
\acro{JSON} that evolved as subset of JavaScript. Data binding languages
(\ref{sec:databinding}) provide a compact and abstract form of a type system
independent from a specific programming language . Some programming languages
even structure data and programs in the same way: that means every program is
semi-structured data in the programming language's own type system.  Rules of
the programming language act like a schema that restricts the \acro{DSL} to
valid, executable code. A typical example of such a data-oriented programming
language is \term{Lisp}, which is purely based on S-Expressions (\acro{S-EXP}).


\subsection{Data binding languages}
\label{sec:databinding}

A special form of data structuring languages are language-specific
\Term[serialization format]{serialization formats}. These are used to convert
data structures in programming languages into byte streams and vice versa, a
process that is also called marshalling or deflating (structures to bytes); and
unmarshalling, deserialization, or inflating (bytes to structures). The general
application of a serialization format is also called \Term{data binding}
because several application can be `bound together' by exchanging data in a
common serialization format. Table~\ref{tab:dslrpcformats} lists several
languages that have been developed for data binding. Some binding languages
come with a more general \term{interface description language} to specify
\acro{API}s and with \tacro[data definition language]{data definition
languages}{DDL} to specify more concrete formats (see
section~\ref{sec:schemas}). The absence of a \acro{DDL} does not mean one
cannot specify concrete formats based on the particular \acro{DSL}, but there
is no common and defined language to express these formats.

\begin{table}[h]
\centering
\begin{tabularx}{\textwidth}{|X|X|X|}
\hline
\textbf{\acro{DSL}} & \textbf{first defined} & \textbf{\acro{DDL}} \\
\hline
\tacro{Abstract Syntax Notation One}{ASN.1} & 1984 by ISO 
& \tacro{Encoding Control Notation}{ECN}
\\ \hline
\tacro{External Data Representation}{XDR} &  1987 by Sun (RFC~1014)
& -- 
\\ \hline
\tacro{CORBA Common Data Representation}{CDR} & 1991 by \acro{OMG}
& \tacro{Interface Description Language}{IDL} 
\\ \hline
\tacro{Structured Data eXchange Format}{SDXF} & 2001 as RFC~3072
& -- % (?) 
\\ \hline
\term{Hessian} & 2004 by Caucho
& -- % (?) 
\\ \hline
\term{Fast Infoset} & 2007 by ISO 
& same as for \acro{XML} (see~\ref{sec:xmlschemas})
\\ \hline
\term{Thrift} & 2007 by Facebook
& -- % (?) 
\\ \hline
\term{Protocol Buffers} & 2008 by Google 
& .proto files \\
\hline
\term{Etch} & 2008 by Cisco 
& -- % (?) 
\\ \hline
\term{MGraph} & 2008 by Microsoft 
& \term{MSchema}/\term{MGrammar} 
\\ \hline 
\acro{BSON} & 2010 by MongoDB 
& -- 
\\ \hline
\end{tabularx}
\caption{Data structuring languages developed for data binding}
\label{tab:dslrpcformats}
\end{table}

% For ``M`` from Microsoft (aka Oslo) see
% http://startbigthinksmall.wordpress.com/2008/12/10/mgraph-the-next-xml/
% http://dvanderboom.wordpress.com/2009/01/17/why-oslo-is-important/
% This includes MSchema, MGraph, MGrammar

% Good critizism
% http://blog.jclark.com/2008/11/some-thoughts-on-oslo-modeling-language.html
% http://dewpoint.snagdata.com/2008/10/21/google-protocol-buffers/

\subsectionexample{Protocol Buffers}
\label{ex:proto}

\term{Protocol Buffers} is a serialization format with associated schema language
developed by Google. It was first introduced for remote procedure
calls and now is used for storing and interchanging all kinds of structured
data (the Protocol Buffers developer Guide names it as ``Google's lingua franca
for data'') \cite{Varda2008}. The format's serialization is binary and thereby
much smaller and quicker to parse then \acro{XML}.  Schemas (see
section~\ref{sec:otherschemas}) are defined in \verb|.proto| files that can be
used to automatically generate parsers and serializers in many programming
languages. The underlying data model is hierarchical: The basic data type of
Protocol Buffers is the ``message'', that is a multimap with unique, unsorted
keys, and repeatable, sorted values. Values can be other messages or instances
of 16 scalar core data types (table~\ref{tab:protocolbuffersdatatypes}). An
earlier version of Protocol Buffers also included a group data type which is
now deprecated. Some types only differ in the way they are serialized (for
instance int32 and sint32) but encode the same values.

\begin{table}[h]
  \centering
  \begin{tabularx}{\linewidth}{|l|l|l|X|}
    \hline
    \textbf{Type(s)} & \textbf{in XML} & \textbf{in Java} & \textbf{Content} \\
    \hline
    int32, sint32 & int32 & int & signed 32-bit integer \\
    \hline
    uint32, fixed32 & uint32 & int & unsigned 32-bit integer \\
    \hline
    int64, sint64 & int64  & long & signed 64-bit integer \\
    \hline
    uint64, fixed64 & uint64 & long & unsigned 64-bit integer \\
    \hline
    float & float & float & 32 bit floating point (IEEE 754) \\
    \hline
    double & double & double & 64 bit floating point (IEEE 754) \\
    \hline
    bool & bool & boolean & true of false \\
    \hline
    string & string & String & Unicode or 7-bit ASCII string \\
    \hline
    bytes & string & ByteString & sequence of bytes \\
    \hline
    enum & enum & enum & choice from a set of given values \\
    \hline
    message & class & class & multimap with unique, unsorted keys
       repeatable, sorted fields (possibly constraint by a schema) \\
    \hline
\end{tabularx}
\caption{Core data types of Protocol Buffers}
\label{tab:protocolbuffersdatatypes}
\end{table}


\subsection{INI, CSV, and S-Expressions}
\label{sec:csvandini}

\Tacro{Comma-separated values}{CSV}, \Tacro{initialization files}{INI}, 
and \Tacro{S-expressions}{S-EXP} exist as \acro{DSL} in several variants.
Despite the lack of a strict and commonly agreed specification, these 
languages are used because of their simplicity in a wide range of
applications. Descriptions of the most used variants of each language
can be found in \textcite{RFC4180} and \textcite{Repici2010} for \acro{CSV},
in \textcite{WP:INI} for \acro{INI}, and in \textcite{Rivest1997} for
\acro{S-EXP}. Each language uses a tiny set of data types with strings or 
byte sequences as the only atomic type. Syntaxes of \acro{INI}, \acro{CSV},
and \acro{S-EXP} are mainly defined as \term{context-free language} in 
\term{Backus-Naur Form} with some additional constraints.

We will now show underlying models for each of these languages. \acro{INI} 
is primarily used for configuration files. In its most basic form, it is
just a \term{key-value} structure with field names (\format{Field}) and 
values (\format{Value}). Some \acro{INI} files may have a second level 
(\format{Section}). Section names should be unique per file and field 
names should be unique per section, but both constraints depend on the
particular variant of \acro{INI}. A general model is shown in 
figure~\ref{fig:inimodel}. In summary, \acro{INI} files are a special 
instance of the record database model as described in 
section~\ref{sec:records} (see see flat file database model in
figure~\ref{fig:flatfilemodel}).

\acro{CSV} is popular to exchange simple lists of database records. 
An example is given in figure~\ref{fig:csvexample}. \acro{CSV}
is based on a  tabular model (figure~\ref{fig:csvmodel}) where
data is stored in cells (\format{Cell}) that form a grid of rows
(\format{Row}) and columns (\format{Column}). In general, all rows
must have the same set of columns, or they are automatically unified
by adding missing cells with a default value. 

\acro{S-EXP} originates in the \term{Lisp} programming languages and
it is also used in some data exchange protocols. The model of
\acro{S-EXP} is a rooted, ordered tree with strings or empty lists
as leafs  (figure~\ref{fig:sexpmodel}). There is not one standard
but several dialects. A canonical subset of \acro{S-EXP} with binary
form has been proposed by \textcite{Rivest1997}.

\begin{figure}
\centering
\begin{tikzpicture}[orm]
 \entity (section) (section) {Section\\(.name)};
 \entity[right=1.8 of section] (field) {Field}
        edge node[roles=3,unique=1-2,unique=3] (r) {} (section);
 \value[right=1.4 of field] (value) {Value}
        edge[required by] node[roles,unique] {} (field); 
 %\value[below=of r.south] {FieldName} edge (r.south);
 \binary[below=1.0 of r.two split south,unique=1:-1,unique=2:-1] (rfn) {};
 \value[left=of rfn] (fname) {Name} edge (r.south);

 \plays (rfn.east) -| (field) (rfn) to (fname); 

 \limits (r.two split south) to node[constraint] (c) {*} (rfn.north);
 
 \node[constraint=partition,below=3mm of r.south east] (d) {};
 \limits (r.three south) to (d) (d) to (rfn.two north);

 \node[rule=*,align=left,anchor=north west] at (6.2,0.5) {
  Fields may be required to be in a Section\\
  Fields may be ordered and/or repeatable\\
  within a Section. Sections may also be\\
  ordered and/or repeatable (not shown).
 };
\end{tikzpicture}
\acro*{INI}%
\caption{Model of \acrostyle{INI} with variants}
\label{fig:inimodel}
\end{figure}
%
\begin{figure}
\centering
\begin{tikzpicture}[orm]
 \entity (cell) {Cell};
 \value[right=1.4 of cell] (value) {Value}
      edge[required by] node[roles,unique=1] {} (cell); 
 \ternary[left=of cell,unique=1-2,index=1:r,index=2:c] (r) {};
 \plays[mandatory] (cell) to (r);
 \entity[below=of r.south west] (row) {Row\\(.nr)};
 \entity[below=of r.south east] (col) {Column\\(.name)};
 \plays[mandatory] (row) to (r.one south);
 \plays[mandatory] (col) to (r.two south);
 \limits (r.one split north) to +(0,0.4) node[constraint] (c) {*};
 \node[rule=*,anchor=north west,align=left] at (3.2,0.7) {
   Rows and Columns should form a complete grid:\\
   each Row that plays r must do so with every\\
   Column that plays c (and vice versa). In most\\
   cases Rows and/or Columns are ordered.
   Columns\\may also have no names but only numbers.
 };
\end{tikzpicture}
\acro*{CSV}
\caption{Model of \acrostyle{CSV} with variants}
\label{fig:csvmodel}
\end{figure}
%
\begin{figure}
\centering
\begin{tikzpicture}[orm]
\entity (e) {Element};
\entity[below=5mm of e.south west,anchor=north east,xshift=-4mm] (l) {List}   
  edge[suptype] node[pos=0.3] (a) {} (e);
\entity[below=5mm of e.south east,anchor=north west,xshift=4mm] (s) {String} 
  edge[suptype] node[pos=0.3] (b) {} (e);

%\node[constraint=xor,below=of e] (c) {};
\limits (a) edge node[constraint=xor] {} (b);

\binary[above=6mm of l,xshift=2mm,unique=2] (r) {};
\limits (r.north) to +(0,0.3) node[constraint] (c) {*};
\plays (l) -- (r.one south) (r) -- (e);

\value[right=1.4 of s] {Value} edge[required by] node[roles,unique] {} (s);
\node[rule=*,align=left,anchor=north west,right=of e] {
  Elements together must form\\an ordered, rooted tree.
};
\end{tikzpicture}\acro*{S-EXP}
\caption{Model of \acrostyle{S-EXP}}
\label{fig:sexpmodel}
\end{figure}

The \format{Value} of a \format{Field}, \format{Cell}, or \format{String} in 
\acro{INI}, \acro{CSV}, or \acro{S-EXP} respectively can hold arbitrary 
byte sequences or character strings, depending on the specific language variant. 
Some byte or character sequences may be disallowed, especially for names of 
sections, fields, and columns in \acro{INI} and in \acro{CSV}.


\subsection{JSON}
\label{sec:json}

\Tacro{JavaScript Object Notation}{JSON} is based on notations of the
\term{JavaScript} programming language. First specified by
\textcite{Crockford2002} and later standardized as \acro{RFC} \cite{RFC4627} it
soon became a widespread language to exchange structured data between web
applications, serving as an alternative to \acro{XML}. \acro{JSON} was first
published in form of a \term{railroad diagram} (see section~\ref{sec:bnf}) and
later expressed in a variant of \term{Backus-Naur Form}.
Figure~\ref{fig:jsonbnf} shows a full \acro{BNF} grammar of \acro{JSON}. In
a nutshell \acro{JSON} is based on data model with five atomic value types
(\format{String}, \format{Number}, \format{Boolean}, and \format{Null}), and
two composite types \format{Array} and \format{Object}. Strings can hold any
\term{Unicode} codepoint, but most application will limit codepoints to allowed
Unicode characters.  Numbers include integer values and floating point values
without limit in length and precision.\footnote{Special numbers like
\texttt{-0}, \texttt{NaN}, and \texttt{Inf} are not allowed.} An \format{Array}
holds a (possibly empty) list of values, and a \format{Object} holds a
(possibly empty) map from strings as keys to data elements as member values.

\begin{figure}[h]
\centering
\begin{lstlisting}[language=BNF,tabsize=11]
Composite	= s* ( Object | Array ) s*
Object	= "{" ( Member ( "," Member )* )? "}"
Array	= "[" ( Value  ( s* "," s* Value )* )? "]"
Member	= Key ":" s* Value s*
Key	= s* String s*
Value	= Composite | String | Number | Boolean | Null
String	= '"' ( char - ( '"' | '\' ) | charref )*  '"'
 charref	= '\' ( ["\/bfnrt] | [0-9A-F][0-9A-F][0-9A-F][0-9A-F] )+
Number	= "-"? ( "0" | [1-9] [0-9]* ) ( "." [0-9]+ )? 
	  ( ( "e" | "E" ) ( "+" | "-" )? [0-9]+ )?
Boolean	= "true" | "false"
Null	= "null"
s	= ( #x20 | #x9 | #xA | #xD )
\end{lstlisting}
\caption{Formal grammar of \acrostyle{JSON}}
\label{fig:jsonbnf}
\end{figure}

The definition of \acro{JSON} syntax as context-free language imposes 
the mathematical structure of a partly-ordered tree on models of
\acro{JSON}. In such a model, nodes are values but atomic types
must be leaf nodes and the root node must be a composite.
Similar structures to \acro{JSON} are found in many programming languages, 
for instance \Term{JavaScript} and \Term{Perl} but they may contain
pointers that go beyond the tree structure. In addition, virtually all
implementations add uniqueness constraint on objects keys,\footnote{repeated 
object keys (like \texttt{\{"a":1,"a":2\}}) are allowed in theory.}
limit maximum size of text, numbers, and nesting level, and restrict 
\format{String} to the Unicode character set.\footnote{Unicode codepoints
outside of \acro{UCS} are allowed but not supported by all
implementations.} With the rise of \term{NoSQL} (see~\ref{sec:nosql})
\acro{JSON} is also used more and more to store data in databases. Most 
\acro{JSON} databases put additional restrictions on special object keys
(\texttt{""}, \texttt{\_id}, \texttt{id}, \texttt{\$ref}\ldots) that are 
used for uniquely identifying and linking \acro{JSON} documents or parts 
of it. Other extensions such as \tacro{Binary JSON}{BSON} restrict atomic 
types and/or add data types that are not part of the \acro{JSON} 
specification.\footnote{\acro{BSON} extends some parts of \acro{JSON}
but is does not support numbers of arbitrary length}
There are some proposals for schema languages for \acro{JSON}
(JSON Schema,\footnote{see \url{http://json-schema.org}},
\term{Kwalify},\footnote{see \url{http://www.kuwata-lab.com/kwalify/}}
JSONR\footnote{see \url{http://web.archive.org/web/20070824050006/http://laurentszyster.be/jsonr/}}\ldots), and for 
query languages to select a subsets of a given \acro{JSON} document
(JPath,\footnote{see \url{http://bluelinecity.com/software/jpath/} and
\url{http://bitcheese.net/wiki/code/hjpath} for two different \acro{JSON}
path languages}
JSONPath\footnote{see \url{http://goessner.net/articles/JsonPath/}}\ldots) but
none of them is widely accepted. Manipulation of \acro{JSON} data is usually
done directly in programming languages or via custom database APIs.

The clear and simple definition of \acro{JSON} has made it a popular data 
structuring language not only for web applications but also for ad-hoc
tasks in structuring, storing, and exchanging data. Proplems may result
from differences in compatibility of atomic types (especially keys and numbers)
and from data that does not fit into the tree-model of \acro{JSON}.

\subsection{YAML}
\label{sec:yaml}

\Aterm{YAML}{YAML Ain't Markup Language} was developed as human-readable 
alternative to \acro{XML} and first published by \Person[Clark]{Evans} 
in 2001 \cite{YAML2009}. Unlike most other \acro{DSL} it can natively 
express hierarchical and non-hierarchical structures. In contrast to most
other data serialization languages, the \acro{YAML} specification defines
in one document: a syntax, a conceptual model, and an abstract serialization
to map between syntax in model.

\acro{YAML} syntax is very flexible: it allows multiple alternatives to
express complex structures in a simple, human readable way as stream of
Unicode characters. Some examples of language constructs are given in
figure~\ref{fig:yamlsyntax}. Apart from repeatable object keys and Unicode 
\format{Surrogate} codepoints, which are not allowed in \acro{YAML}, the 
syntax is a superset of \acro{JSON} syntax. Other similarities exist with
the \tacro{semistructured data expression syntax}{ssd} used by 
\textcite{Abiteboul2000}. The abstract serialization of \acro{YAML} is 
called its \Term[serialization (YAML)]{serialization tree}. The
serialization tree can be be traversed as sequence of parsing/serializing 
events, similar to the event-driven \tacro{Simple API for XML}{SAX}
(see page~\ref{note:sax}). The conceptual model of \acro{YAML} is called
its \Term[representation graph (YAML)]{representation graph}.
%
% \term[representation graph (YAML)]{representation graph}, the specification
% defines an intermediate ordered \Term[serialization tree (YAML)]{serialization tree}
% http://robotlibrarian.billdueber.com/data-structures-and-serializations/
% No standard path and schema languages (but proposals)
It is defined by the specification as ``rooted, connected, directed graph
of tagged nodes''. Eventually this is a special multi-property graph with 
possible \term{loop}s and three disjoint kinds of nodes. Figure~\ref{fig:yamlmodel}
gives a partial model of the representation graph in ORM2 notation:\footnote{Roots,
scalar values, (local) tag names and URIs are not included.}
\format{Sequence}
nodes impose on order on outgoing edges, and \format{Mapping} nodes have
their outgoing edges indexed by node values, as described below. Nodes
of \term{outdegree} zero can also be of the \format{Scalar} kind, which
each holds a \term{Unicode} string as value. Mapping keys can be arbitrary 
nodes, which makes the structure rather complex
-- but in practice most \acro{YAML} instances represent simple hierarchies. 
Each node in a \acro{YAML} representation graph has exactly one \format{Tag}
as node type.

% TODO: clarify the following:
Tags can be either identified by an \acro{URI} 
(\format{GlobalTag}) or by a simple string (\format{LocalTag}).
A \Term[schema!in YAML]{YAML schema} is a set of tags. Each tag is defined
by an URI, an expected node kind (scalar, sequence, or mapping) and a 
mechanism for converting its node's values to a canonical form.\footnote{See
\url{http://yaml.org/type/} for a registry of known tags.} Furthermore,
a tag may provide additional information such as the set of allowed values
for validation and the schema may provide a mechanism for automatically
resolving values to tags. For instance a schema could automatically 
tag the string \texttt{true} as boolean value instead of a literal string.
Normalization of node values to their canonical form is important for
node comparision. Keys of a mapping node must not only be different but 
unequal. Two nodes are equal if they have the same tag and the same 
canonical content. Equality of sequences and mappings is defined
recursively.\footnote{Note that recursive equality checks may require
determining whether the subgraphs used as keys are isomorphic -- a
problem that is not solvable in polynomial time in worst case.} 
The \acro{YAML} specification lists some possible types and schemas
but their support depends on particular implementations of \acro{YAML}
parsers. \acro{YAML} neither defines a standard how to express types
and schemas in a machine-readable way so their defintion is only adressed
to implementors and users. Support of additional collection types such as 
sets and ordered mappings also depends on additional conventions.

% TODO: graph ismomorphism complexity
% TODO: summary?

In summary the data structuring philosophies behind \acro{YAML} are very
sophisticated but too complex for most applications. Especially the support
of arbitrary nodes as array keys has little practical value but complicates
the construction of a full \acro{YAML} model. 

\begin{figure}
\centering
\begin{tikzpicture}[orm]
\matrix (m) [column sep=1.2cm,row sep=4mm,matrix of nodes,nodes in empty cells]
{
           &|[entity]|Sequence     & |[entity]|SequenceTag \\
 |[entity]|Node &[18mm]|[entity]|Scalar & |[entity]|ScalarTag 
  & |[entity]|Tag \\ %& |[value]|TagName \\
                &|[entity]|Mapping      & |[entity]|MappingTag \\
};
\draw[subtype] (m-2-1) to (m-1-2);
\draw[subtype] (m-2-1) to (m-3-2);
\draw[subtype] (m-2-1) to (m-2-2);
\limits ($(m-2-1)!.5!(m-1-2)$) to node[constraint=partition](c){}
        ($(m-2-1)!.5!(m-3-2)$);
\draw[subtype] (m-2-4) to (m-1-3);
\draw[subtype] (m-2-4) to (m-3-3);
\draw[subtype] (m-2-4) to (m-2-3);
\limits ($(m-2-4)!.5!(m-1-3)$) to node[constraint=partition](c){}
        ($(m-2-4)!.5!(m-3-3)$);
\draw[mandatory] (m-1-2) to node[roles,unique]{} (m-1-3);
\draw[mandatory] (m-2-2) to node[roles,unique]{} (m-2-3);
\draw[mandatory] (m-3-2) to node[roles,unique]{} (m-3-3);

\draw (m-1-2) -| node(r)[roles,xshift=9mm]{} (m-2-1);
\limits (r.one north) to +(0,4mm) node[constraint] {$<$};

\draw (m-3-2) to (m-3-1) 
  node(s)[roles=3,xshift=2mm,unique=2-3]{};
\draw (s.one north) to node[role name,anchor=east]{[value]} (m-2-1);
\draw (s.two north) to node[role name,anchor=west]{[key]} (m-2-1);

\limits (s.two split south)  to +(0,-3mm) node(c)[constraint] {};

\node[anchor=north west,xshift=-3mm] at (c) 
  {keys must be recursively unequal per mapping};

\end{tikzpicture}
\caption{\acrostyle{YAML} data model (partial)}
\label{fig:yamlmodel}
\end{figure}

\begin{table}
\begin{tabularx}{\textwidth}{rX}
 \verb|&x foo| & 
    Scalar node with link anchor \texttt{x} and value \texttt{"foo"} \\
 \verb|[ *x, bar ]| & Sequence node with previously defined node
    \texttt{x} and another scalar node with value \texttt{"bar"} as members \\
 \verb|{ key1: foo, key2: bar }| & Mapping node with two key-value pairs \\
 \verb|{!!str 42}|               & 
    \texttt{"42"} tagged as string (instead of number) \\
 \verb|!point {x: 12, y: 4}|     & Mapping node with local tag \texttt{point} \\
 \verb|? [ a, b ] : [ 1, 2, 3 ]| & Mappping node with one sequence as
    key and another sequence as value \\
 \verb|&n [ *n, *n ]| & Sequence node that contains itself twice \\
 \verb|&m { *m : *m }| & Mapping node that maps itself to itself \\
\end{tabularx}
\acro*{YAML}
\caption{Examples of \acrostyle{YAML} syntax, including some edge cases}
\label{fig:yamlsyntax}
\end{table}

\subsection{XML}
\label{sec:xml}

\begin{quotation}%
XML has succeeded beyond the wildest expectations as a convenient format
for encoding information in an open and easily computable fashion. But it 
is just a format, and the difficult work of analysis and modeling 
information has not and will never go away.
\\\quotationsource \textcite{Wilde2008}
\end{quotation}

\noindent
The \Tacro{Extensible Markup Language}{XML} was designed between 1996
and 1998 as simplified subset of the \tacro{Standard Generalized 
Markup Language}{SGML} for the Web \cite{Bray1998}. Its origin in 
\acro{SGML} (see section~\ref{sec:markuplanguages} about \acro{SGML} and 
markup languages in general) gave \acro{XML} strong support for marked
up text documents, but also some features, that for most applications 
only add unnecessary complexity. Beginning from the late 1990s, more and 
more domain specific data formats were created based on \acro{XML}, or they
migrated to \acro{XML} from \acro{SGML}. \acro{XML}~1.0 was first published
as \acro{W3C} recommendation in February 1998. Soon it was accompanied 
by numerous extensions and revisions, such as the
\tacro{Document Object Model}{DOM} in late 1998,
% http://www.w3.org/TR/1998/REC-DOM-Level-1-19981001/
\term{XML Namespaces} (1999),
% http://www.w3.org/TR/1999/REC-xml-names-19990114/
\term{XPath} (1999),
%, see example~\ref{ex:xpath} at \pageref{ex:xpath}), % http://www.w3.org/TR/1999/REC-xpath-19991116
\acro{XSLT} \cite{Clark1999x},  % http://www.w3.org/TR/1999/REC-xslt-19991116
\tacro{XML Schema}{XSD} (2001), % http://www.w3.org/TR/2001/REC-xmlschema-0-20010502/
\term{Canonical XML} (2001), % http://www.w3.org/TR/2001/REC-xml-c14n-20010315
\term{XML Base} (2001), % http://www.w3.org/TR/2001/REC-xmlbase-20010627/
\term{XML Infoset} (2001), % http://www.w3.org/TR/2001/REC-xml-infoset-20011024/
and \term{XInclude} (2004). % http://www.w3.org/TR/2004/REC-xinclude-20041220/
% W3C. XML Linking Language (XLink) Version 1.0, June 2001.
% http://www.w3.org/TR/2000/REC-xlink-20010627/.
% W3C. XPointer xpointer() Scheme, December 2002.
% http://www.w3.org/TR/2002/WD-xptr-xpointer-20021219/.
\acro{XML}~1.1 was introduced in 2004 as successor to \acro{XML}~1.0
\cite{Bray2004}, but it never got widely adopted.
% because it broke compatibility 
% and introduced no compelling new features.% see "Xml 1.1 failed"
The listed extensions define slightly different models of \acro{XML},
and the degree of their support varies among applications, what 
complicates an exact definition of \acro{XML} documents \cite{Dodds2002}.
However, all definitions share a common subset, that can be described as 
an ordered tree with Unicode strings and key-value-pairs as node-properties.
Beginning with \acro{XML}~1.0, we will first describe the most common parts of 
\acro{XML} syntax, then discuss aspects of \acro{XML} processing and differences 
between models of the \acro{XML} family of standards, and finally give an 
overview and review of the most common \acro{XML} structures.

\acro{XML}~1.0 is defined based on a context-free grammar over a sequence of
\term{Unicode} characters with some additional \Term[well-formed]{well-formedness}
constraints. The grammar is given in a variant of \term{Backus-Naur-Form}. 
Figure~\ref{fig:xmlbnf} shows a slightly adopted subset of the grammar rules: 
A \bnf{document} starts with an optional \bnf{prolog}, followed by a mandatory 
root \bnf{element}, and optional \bnf{comment}, processing-instructions (\bnf{pi}),
and whitespaces (\bnf{s}). The \bnf{prolog} usually contains an 
\acro{XML} declaration, that among other information can specify the 
character encoding, a standalone flag, and a \tacro{document type definition}{DTD}.
An \bnf{element} in \acro{XML} syntax either consist of a \bnf{starttag} and
an \bnf{endtag} with the same \bnf{name}\footnote{The same name requirement
that is one of the constraints that cannot be expressed in \acro{BNF}.} and some
\bnf{content} in between, or it is an \bnf{emptytag}.
Start tags and empty tags can have a list of \bnf{attribute}, which are
key-value-pairs with unique \bnf{name} per attribute list.%
\footnote{The uniqueness requirement of attribute names is another 
additional well-formedness constraint.} A \bnf{content} may contain
other \bnf{elements}, resulting in the general tree of \acro{XML} documents 
(see example~\ref{ex:xmlmods} for a document).

\begin{figure}[h]
\centering
\begin{lstlisting}[language=BNF]
document  = prolog element misc*
misc      = comment | pi | s
s         = ( #x20 | #x9 | #xA | #xD )+
element   = starttag content endtag | emptytag
starttag  = "<" name (s attribute)* s? ">" 
endtag    = "</" name s? "/>"
emptytag  = "<" name (s attribute)* s? "/>"
content   = text? ((element | reference | cdata | pi | comment) text?)*
text      = chars - (chars ("<" | "&" | "]]>") chars)
reference = charref | entityref
charref   = "&#" [0-9]+ ";" | "&#x" [0-9a-fA-F]+ ";"
entityref = "&" name ";"
value     = (text | reference)*
cdata     = "<![CDATA[" (chars - (chars "]]>" chars)) "]]>"
comment   = "<!--" (chars - (chars "--" chars | chars "-") "-->"
pi        = "<?" pitarget s (chars - (chars "?>" chars)) "?>"
attribute = name s? "=" s? ( '"' (value - (value '"' value)) '"' 
                            | "'" (value - (value "'" value ) "'" )
\end{lstlisting}
\caption{Subset of the formal grammar of \acrostyle{XML}}
\label{fig:xmlbnf}
\end{figure}

Textual data (\bnf{text}) in \acro{XML} can be any \term{Unicode} 
string, except some codepoints below \U{0020}, \U{FFFE} and \U{FFFF}. 
Furthermore the characters `\verb|<|' and `\verb|&|', and in \bnf{content}
the sequence `\verb|]]>|' is not allowed. To include these characters
in an \acro{XML} document, you can use character references (\bnf{charref})
which can refer to an allowed Unicode character by its \acro{UCS} codepoint.
In addition there are predefined named entities (\bnf{entityref}): `\verb|&lt;|'
for `\verb|<|', `\verb|&gt;|' for `\verb|>|', `\verb|&amp;|' for `\verb|&|', 
`\verb|&apos;| ``for `\verb|'|', and `\verb|&quot;|' for `\verb|"|'.
\acro{XML} is further complicated by the possibility to define named entities
in a \acro{DTD}. These entities can either stand for an arbitrary piece 
of \bnf{content} (\Term{internal entity}) or as placeholder for some other data
that is referenced by an \acro{URI} (\Term{external entity}).

Most entities are replaced by their content, when an \acro{XML} document is
read by an \Term{XML processor} (a piece of software that parses the syntax
of an \acro{XML} document and provides access to its content and structure). 
However, some named entities can remain as unparsed artifacts because they 
are external or because the \acro{DTD} is not taken into account by the
processor. In practice the \tacro{Simple API for XML}{SAX} \cite{Megginson2004}
\label{note:sax}
is a common abstraction in \acro{XML} processors, especially for the \term{Java} 
programming language. \acro{SAX} is not a formal specification but it originates
in an implementation of an \acro{XML} parser that was first discussed in early
1998. The \acro{API} of \acro{SAX} provides a stream of parsing events that can
be used to construct an \acro{XML} document, if the stream of events follows the
well-formedness constrain of \acro{XML} (every \acro{XML} document can be mapped
to a stream of \acro{SAX} events but not vice versa).

\acro{XML} 1.0 defines two types of \acro{XML} processors:
validating and non-validating processors. Non-validating processors must
only check whether a document is well-formed, but they do not need to
process all aspects of a \acro{DTD}.\footnote{Some simple \acro{XML}
processors just ignore the \acro{DTD} although this is against the 
specification. Removal of \acro{DTD} is one of the most common request
in discussions about a future ``\acro{XML} 2.0'', as most \acro{XML}
documents have no \acro{DTD}, and validating is mostly done by
using other schema languages.}
Validating parsers must analyze the entire \acro{DTD}, including other 
documents referenced from the \acro{DTD}, and they must check whether
the document matches the additional rules from its schema (see 
section~\ref{sec:xmlschemas}). A processor may even change the content of an
\acro{XML} document by normalizing strings and by adding default values.

\begin{figure}[h]
\centering
\begin{tikzpicture}
\node[text width=2.5cm,align=center] (d) {\textbf{\acrostyle{XML} document}\\(syntax)};
\node[right=1.8cm of d,text width=3cm,align=center](m){\textbf{document model}\\(structure)};
\draw[->] (d) to node[yshift=3mm] {parsing} (m);
\node[right=0 of m,text width=5cm,yshift=-9mm] (l) {%
parsed document can be:
\begin{itemize}
 \item not well-formed\\(syntax error, no model)
 \item of some model type\\ (\acrostyle{DOM}, \term{Infoset}, \term{Canonical} \ldots)
 \item modified by validation
 \item invalid by validation
\end{itemize}
};
\node[below=3mm of d,text width=3cm,xshift=4mm]{\texttt{<a b:c="d">\\%
~<e f='g'>h</e>\\
~\&i;<?j k?>\\</a>}};
\node[below=2mm of m.south west,xshift=6mm] (a0) {};
\begin{scope}[orm];
\node at (a0) (a) {a};
\node[right=2mm of a,yshift=0mm] (bc) {b:c};
\node[right=2mm of bc] (d) {d};
\draw (a) -- (bc) -- (d);
\node[below=9mm of a] (e) {e};
\node[right=2mm of a,yshift=-5mm] (j) {j};
\node[right=0mm of a,yshift=-9mm] (i) {i};
\node[right=1mm of e,yshift=-5mm] (h) {h};
\draw (a) -- (e);
\draw (e) -- (h);
\draw (a) -- (i);
\node[right=2mm of e] (f) {f};
\node[right=2mm of f] (g) {g};
\draw (e) -- (f) -- (g);
\draw (e) -- (h);
\node[right=2mm of j] (k) {k};
\draw (a) -- (j) -- (k);
\end{scope}

\draw[decoration={brace},decorate,yshift=2mm] (l.south west) to (l.north west);
\end{tikzpicture}
\caption{\acrostyle{XML} document and \acrostyle{XML} document models}
\label{fig:xmlparsing}
\end{figure}

Parsing \acro{XML} can best be understood as a process that converts \acro{XML}
syntax, given as sequence of characters, to another data structure (figure%
~\ref{fig:xmlparsing}). In general the act of parsing an \acro{XML} document 
is not reversible, because some aspects of \acro{XML} syntax are considered
as irrelevant (figure~\ref{fig:irrelevantxmlparts}). The resulting data 
structure is a model not only of the parsed document, but of all other 
``logically equivalent'' documents that result in the same model. Parsing
\acro{XML} can result in different structures. If the 
original data was not well-formed,
there is no model, and the document is no \acro{XML} by definition.\footnote{%
In practice you sometimes have to deal with not-well-formed documents that were 
intended to be \acro{XML}. You can call this documents `broken' \acro{XML} if
there is a chance to recover well-formedness.} The specific type of model
defines, which parts of syntax are translated to which parts of a model and 
which parts are omitted as irrelevant to the given model 
(figure~\ref{fig:irrelevantxmlparts}). A processor may also modify the 
document to some degree or it may mark the document as invalid.

\label{p:xmlmodel}
The most prominent models of \acro{XML} are the \Tacro{Document Object Model}{DOM}
and \Term{XML Infoset}. \acro{DOM} evolved parallel to \acro{XML} in the late 1990s.
It was created to harmonize existing \term{JavaScript}-Interfaces that had been created
by Web browser makers for manipulating \acro{HTML} documents. The part of
\acro{DOM} that deals with \acro{XML} documents is `\acro{DOM} Core'.
Actually there are three variants: Level 1 is based on the
tree structure of \acro{XML} 1.0, Level 2 expresses the structure of \acro{XML}
with Namespaces, and Level 3 expresses a model compatible with XML Infoset
\cite{Cowan2004}. Another model of \acro{XML} is shared by XPath 1.0 and 
Canonical \acro{XML} \cite{Boyer2008}, XPath 2.0 and XQuery define yet 
another model \cite{Berglund2010}. A given model may also be expressed in
other languages but \acro{XML} syntax. For instance \Term{Fast Infoset} 
\cite{FastInfoset2005} is a binary representation of Infoset based on
\acro{ASN.1} and \textcite{Tobin2001} defines an \acro{RDF} Schema to serialize
\acro{XML} document models as \acro{RDF} instances.

\begin{figure}
\begin{multicols}{2}
\begin{itemize}
 \item type of attribute delimiters ("/')
 \item type of character entities
 \item original character encoding
 \item CDATA sections
 \item standalone flag
 \item all entity references
 \item specified schemas
 \item whitespaces
 \item position of namespace declarations
 \item namespace prefixes
 \item attribute types (e.g. \dtd{ID}, \dtd{IDREF}\ldots)
 \item explicit default attributes
 \item original form of normalized attributes
 \item original form of normalized Unicode
 \item comments
 \item processing instructions
\end{itemize}
\end{multicols}
\caption{Some properties of \acrostyle{XML} considered as irrelevant by some processors}
\label{fig:irrelevantxmlparts}
\end{figure}

Despite all minor differences, all document and processing models of \acro{XML}
share a basic structure, that can be described as ordered tree with nodes of 
different types. Basically, there are element nodes with
exactly one element as root, attribute nodes, and text nodes. Other node
types (processing-instructions, comments, external entity references etc.) are 
much less used to hold relevant information, and they more depend on the 
particular document model. 

Each element node has a (possibly empty) set of unordered attribute
nodes with unique attribute names, and an ordered (possibly empty) list
of text and/or element nodes as child nodes. Attribute nodes cannot hold
nested structures but only one text node each, and text nodes are Unicode
strings with some code points excluded.

Each attribute and each element node has a name. The exact definition of a
\bnf{name} from figure~\ref{fig:xmlbnf} depends on the specific 
\acro{XML} model: in \acro{XML} 1.0 a name is just a Unicode
string that not contains some disallowed characters. The dependence on a particular version 
of Unicode was lifted with the fifth edition \cite{Bray2008}. The most 
important (and often confusing) extension to \acro{XML}~1.0 is 
\acro{XML} Namespaces \cite{Bray2009}: namespaces allow names of elements 
and attributes to be qualified by an \acro{URI}. This way names can be grouped
together in vocabularies
and elements from different vocabularies can be mixed in one document.
In the model of \acro{XML} with namespaces (and in other techniques that 
build upon namespaces, such as \acro{DOM} Level 2 and 3, \term{Infoset} etc.)
a name is triple consisting of the namespace \acro{URI}, a local name, and a
namespace prefix. In \acro{XML} syntax namespaces are declared by 
special attributes that start with \verb|xmlns| (in example~\ref{ex:xmlmods}
the namespace is declared at the root element so it applies to the whole
document).
Example~\ref{ex:xmlns} shows three \acro{XML} elements that make use of
a namespace declaration. In most cases only the namespace 
\acro{URI} and the local name matter, so the first two examples should
be treated as equivalent. The prefix is also included in most
models, and some applications rely on it.\footnote{
See \url{http://www.w3.org/TR/xml-c14n\#NoNSPrefixRewriting} for details.}
The third example \ref{ex:xmlns} is always 
different from the two above: in contrast to \acro{RDF} Turtle syntax
(see section~\ref{sec:rdf}), namespaces and local names cannot be used to 
construct a canonical name, but they must be used together to identify the
full name of an \acro{XML} element or attribute.\footnote{Some vocabularies
may specify \emph{additional} identifiers for \acro{XML} elements, for
example in \term{XML Schema} each element has an \acro{URI} that happens to
be constructable by appending local name to namespace \acro{URI}. However
there is no general rule to do so in other vocabularies.}

\begin{example}
\begin{tabular}{ll}
\textbf{element in \acrostyle{XML} syntax} & \textbf{namespace, local name, prefix} \\
\hline
\lstinline[language=XML]|<x:zz xmlns:x="http://example.org/"/>|   & ( http://example.org/, zz, x ) \\
\lstinline[language=XML]|<y:zz xmlns:y="http://example.org/"/>|   & ( http://example.org/, zz, y ) \\
\lstinline[language=XML]|<xz:z xmlns:xz="http://example.org/z"/>| & ( http://example.org/z, z, xz )\\
\end{tabular}
\caption{Namespaces in \acrostyle{XML}}
\label{ex:xmlns}
\end{example}

To allow more complex graph structures, there are several techniques to extend
the basic tree model of \acro{XML} with links: attributes can be defined to
only hold unique \dtd{ID} values or references to other identifiers
(\dtd{IDREF} in \acro{DTD} or keyref constraints in \term{XML Schema}).
\term{XLink} \cite{DeRose2010} and \term{XPointer} \cite{Grosso2003} describe
other extensions to \acro{XML} to create links to portions of \acro{XML}
documents. However, like other extensions to \acro{XML} 1.0, this adds another
layer of complexity and another model that first must be agreed on to achieve
interoperability. To reduce complexity within the family of \acro{XML}
specification, simplified subsets have been proposed by \textcite{Bray2002},
\textcite{Clark2010} and others, but none of them has widely been adopted yet.
Nevertheless, \acro{XML} is successfully being used to encode and exchange data
on the Web and in other areas from markup languages such as \acro{TEI} to
structured metadata formats such as \acro{METS}, \acro{MODS}, and \acro{EAD}.
Furthermore several serialization forms of other formats in \acro{XML} exist,
for instance \acro{RDF/XML} for \acro{RDF} and \acro{MARCXML} for \acro{MARC}.
As described by \textcite{Wilde2008}, many problems with \acro{XML} arose from
overbroad claims for \acro{XML}, which in the end is just a format.  It still
suits best for marked-up textual data and other records that can be modeled
well as ordered tree, but less for data with arbitrary order and links.


\begin{example}
\centering
\begin{lstlisting}[language=XML,morekeywords={xmlns}]
<?xml version="1.0" encoding="UTF-8"?>
<mods xmlns="http://www.loc.gov/mods/v3" version="3.4">
  <titleInfo>
    <nonSort>The </nonSort>
    <title>C programming language</title>
  </titleInfo>
  <name type="personal">
    <namePart>Kernighan, Brian W.</namePart>
  </name>
  <name type="personal">
    <namePart>Ritchie, Dennis M.</namePart>
  </name>
  <originInfo>
    <place>
      <placeTerm type="text">Englewood Cliffs, NJ</placeTerm>
    </place>
    <publisher>Prentice-Hall</publisher>
    <dateIssued>1978</dateIssued>
  </originInfo>
</mods>
\end{lstlisting}
\acro*{MODS}
\caption{\acrostyle{MODS} record in \acrostyle{XML}}
\label{ex:xmlmods}
\end{example}




\subsection{RDF}
\label{sec:rdf}

The \Tacro{Resource Description Framework}{RDF} dates back to the
\tacro{Meta Content Framework}{MFC} which \Person[Ramanathan]{Guha}
had created in the 1990s at \term{Apple} \cite{Andreessen1999,Guha1996}.
\textcite{Guha1997} submitted the idea of \acro{MFC} to the \acro{W3C}, where 
it evolved to a general graph-based metadata framework, first released as 
\acro{RDF} specified by \textcite{Lassila1999}. Merged with ideas of 
\textcite{TBL1997}, the focus had widened to metadata about any objects for
creating a ``\term{Semantic Web}'', that can express knowledge
about the world itself \cite{BernersLee2001a}. The ambitious aim of \acro{RDF}
can be traced back further to the artificial intelligence project \term{Cyc},
which \person[Ramanathan]{Guha} was co-leader of in 1987--1994, and to
the original proposal of the \acro{WWW} \cite{BernersLee1989}.
We will first describe the basic components of \acro{RDF}, show how its 
structure can be described and extended, and list several serialization forms.
Afterwards we will discuss how semantics is brought to \acro{RDF} via an
algebra over RDF graphs.

\subsubsection{RDF components}

In its most basic form --- that is without additional techniques such as 
\acro{RDFS}, \acro{OWL}, \acro{SKOS}, etc. --- \acro{RDF} is just a, graph-based 
data structuring language. The \acro{RDF} data model is defined as abstract 
syntax by \textcite{Klyne2004} as follows: an \Term[graph!in RDF]{RDF graph} 
is a set of \Term[triple (RDF)]{RDF triples} (also known as \Termstyle{statement}s)
each consisting of a \Term[subject (RDF)]{subject}, a \Term[predicate (RDF)]{predicate}
(also known as \Term[property (RDF)]{property}), and an \Term[object (RDF)]{object}.
The nodes of an RDF graph are its subjects and objects, and the edges are labeled by
its predicates. Because RDF graphs are defined on mathematical sets, each particular 
combination of subject, predicate, and object is only counted once in a graph.
In summary, an RDF graph can be described as multigraph with labeled edges, possible
loops (triples where subject and object coincide), and partly labeled nodes. This 
multigraph may contain two kinds of graph labels: First an \Term{URI reference} 
(or \Termstyle{URIref}) is an absolute, percent-encoded \acro{URI} or \acro{IRI}. 
Two URIrefs are equal if and only if they compare as equal as encoded strings.%
\label{term:uriref}%
\footnote{This definition of equality is not based on normalized \acro{IRI}s, but 
on Unicode character string comparision \cite[6.4]{Klyne2004}. This makes 
\url{http://example.com/\%41} and \url{http://example.com/A} two distinct URIrefs, 
although in most applications, after normalization of percent-encoding, the former 
results in the latter. The ambiguity cannot be solved in general, because there
is no general canonicalization algorithm for all types of \acro{IRI}s. It is 
possible but strongly discouraged to have two different URIrefs that 
percent-encode the same \acro{IRI}.} URIrefs are treated as identifiers
for \Term[resource (RDF)]{RDF resources}, which beside triples are the central 
part of the Resource Description Framework. No assumptions are made about the 
nature of resources, but the same URIref always refers to the same resource
\cite[section 1.2]{Hayes2004}. \acro{RDF} allows different URIrefs to refer to
the same resource (synonyms), but unlike natural langugages it is assumed to
have no homonym URIrefs just by definition.\footnote{By this, \acro{RDF}
can be seen as one of many attempts to create a `perfect language', in which same
words always refer to same objects. The history of other attempts and their
failures have been illustrated vivid by \textcite{Eco1995}.}
It should be noted that RDF graphs cannot contain resources which are not linked
to other resources: you state that some resource plays a specific role in at least
one \acro{RDF} triple, but you cannot state that a selected resource only `exists`.

The second type of graph labels are \Term[literal (RDF)]{RDF literals}. A
literal is a Unicode string, which should be in \term{Normalization Form C}.
Optionally it is combined with either a lowercase \Term{language tag} as
defined by \textcite{RFC4646}, or with a \Term{datatype URI} being an URIref.
Literals with datatype are called \Term{typed literals} in contrast to
\Term{plain literals}. Two literals are equal if they hold the same Unicode
string (also called its \Term[lexical form (RDF literal)]{lexical form}), and
(if given) the same language tag datatype \acro{URI}. Datatypes may enforce
restrictions and normalization rules on lexical forms, but the details of this
rules are out of the scope of basic \acro{RDF}.\footnote{To give an example,
the \term{XML Schema} specification \cite{Biron2004} defines the datatype
\rdf{xs:boolean} with four allowed lexical values (``\texttt{true}'',
``\texttt{1}``, ``\texttt{false}'', ``\texttt{0}'') that map to the canonical
literal values ``\texttt{true}'' and ``\texttt{false}''.}

In addition to labeled nodes, at least subjects and objects can be unlabeled
\Term[blank node]{blank nodes}. Blank nodes are treated as variables for 
unknown URI references in one particular RDF graph. You can state that two
blank nodes in one graph refer to the same resource, but blank nodes from
different graphs are disjoint, unless you replace them with URIrefs. In
practice, blank nodes are identfied by arbitrary identifiers, that are not
shared among different graphs. As laid out by \textcite{Carroll2003}, blank
nodes can make it hard to check, whether two graphs are equal, to calculate 
a canonical representation of an RDF graph, and to remove all infereable
tuples from a graph.\footnote{\label{fn:gi}It is assumed
that the underlying \tacro{graph isomorphism problem}{GI} is strictly harder 
than \tacro{polynomial time}{P}, and strictly easier than nondeterministic 
\tacro{non-deterministic polynomial time}{NP} \cite{Kobler2006}.}

\begin{figure}
\centering
\begin{tikzpicture}[orm,arrow/.style={->,thick},box/.style={draw,rectangle,thick}]
\node[box,inner sep=4pt] (r) {info:oclcnum:318301686};

\node[right=15mm of r,yshift=8mm] (t) {"The C programming language"@en};
\draw[arrow,out=35,in=180] (r) to (t);
\node[left=1mm of t,yshift=2mm] {dc:title};

\node[right=15mm of r] (y) {"1978"\textasciicircum\textasciicircum};
\draw[arrow] (r) to (y);
\node[left=1mm of y,yshift=2mm] {dc:date};
\node[right=0mm of y,box,inner sep=2pt,yshift=-0.5mm] 
    {http://www.w3.org/2001/XMLSchema\#gYear};

\node[draw,thick,circle,right=16mm of r,yshift=-8mm] (b1) {};
\node[draw,thick,circle,right=16mm of r,yshift=-15mm] (b2) {};

\draw[arrow,out=-25,in=180,shorten >=1mm] (r) to (b1);
\node[left=1mm of b1,yshift=3mm] {dc:creator};
\draw[arrow,out=-75,in=180,shorten >=1mm] (r) to (b2);
\node[left=1mm of b2,yshift=2mm] {dc:creator};

\node[right=18mm of b1] (v1) {"Kernighan"};
\draw[arrow] (b1) to node[yshift=2mm] {foaf:name} (v1);
\node[right=18mm of b2] (v2) {"Ritchie"};
\draw[arrow] (b2) to node[yshift=2mm] {foaf:name} (v2);

\matrix[row sep=2mm] (m) at (6,-3.2) {
 \node[box,anchor=west,label=left:{dc:title =},inner sep=2pt]  
     {http://purl.org/dc/elements/1.1/title}; \\
 \node[box,anchor=west,label=left:{dc:creator =},inner sep=2pt] 
    {http://purl.org/dc/elements/1.1/creator}; \\
 \node[box,anchor=west,label=left:{foaf:name =},inner sep=2pt] 
   {http://xmlns.com/foaf/0.1/name}; \\
};
\draw[decorate,decoration={brace},xshift=-2cm] 
  (m.south west) -- node[anchor=east,inner sep=2mm]
  {abbreviations} (m.north west);

\end{tikzpicture}
\caption{Example of a simple RDF graph}
\label{fig:rdfgraphexample}
\end{figure}

Figure \ref{fig:rdfgraphexample} shows a simple \acro{RDF} graph consisting
of six triples. URIrefs are depicted as rectangles, blank nodes as circles,
and predicates are labeled arcs. Literals are shown in quotation marks,
optionally followed by ``\verb|@|'' and a language tag, or by
``\verb|^^|'' and a datatype \acro{URI}. The same graph in
Turtle syntax is shown in figure~\ref{fig:rdfgraphexturtle}.

\begin{table}
\centering
\begin{tabular}{|l|c|c|c|c|c|c|}
\hline
\textbf{\acrostyle{RDF} extension} & \textbf{subject} & \textbf{predicate} & 
\textbf{object} & \textbf{datatype} & \textbf{lang.} & \textbf{graph} \\
\hline
standard          & $U \cup B$        & $U$        & $U \cup B \cup L$ & $U$ & $T$ & -- \\
\hline
symmetric         & $U \cup B \cup L$ & $U$        & $U \cup B \cup L$ & & & \\
\hline
generalized       & $U \cup B \cup L$ & $U\cup B\cup L$ & $U\cup B \cup L$ & & & \\
\hline
full blanks       & & $U\cup B$ & & $U\cup B$ & & $U\cup B$ \\
\hline
named graphs      &  &  &  & & & $U$ \\
\hline
language URIs     & & & & & $U [\cup B]$ & \\
\hline
\end{tabular}

\begin{tabular}{l}
$U$: URIref, $B$: blank node, $L$: Literal,
$S$: Unicode string, $T$: language tag \\
$L= S \cup (S \times U) \cup (S \cup T)$ 
and $U, B, L, S, T$ are pairwise disjoint. \\
The set of blank nodes $B$ is partitioned into disjoint sets
for different \acro{RDF} graphs.
\end{tabular}

\caption{Definitions of the \acrostyle{RDF} data model and its extensions}
\label{tab:rdfvariants}
\end{table}

\subsubsection{The model and its extensions}
\label{sec:rdfmodels}

An \acro{RDF} graph can formally be defined as subset of the set 
$(U \cup B) \times U \times (U \cup B \cup L)$ of all triples, as laid
out in table~\ref{tab:rdfvariants}. You can think of several
useful extensions of the standard \acro{RDF} data model.
First literals can also be allowed as subject of a triple. This 
extension to `symmetric' \acro{RDF} allows reversing the direction of 
any triple by switching subject and object. In standard \acro{RDF} you
cannot state, that a given literal is the `name of' a given resource, but 
only that a resource is `named as' a literal. Symmetric \acro{RDF} is 
allowed at least internally in many \acro{RDF} applications, for instance
in \acro{SPARQL} \cite[sec. 12.1.4.]{Prudhommeaux2008}.
Second you could allow literals and blank nodes at any part of a triple.
This extension to \Term{generalized RDF} is allowed for instance in
\acro{OWL2} \cite[sec. 2.1]{Schneider2009}. An \acro{ORM} model of
generalized RDF is given in figure~\ref{fig:genrdfmodel}.
Third you could allow blank nodes at every place where URIrefs are allowed,
that means also as predicates and/or data types.

A different popular extension
allows labeling whole RDF graphs by URIrefs. This extension was 
introduced as \Term{named graphs} by \textcite{Carroll2005}. It has been 
adopted for instance in \term{triple store}s (in this case also known as 
\term{quadstore}s) that deal with multiple RDF graphs.
Finally the
replacement of language tags by URIrefs or blank nodes would repair 
another design failure of standard \acro{RDF}.\footnote{In standard \acro{RDF}
you cannot refer to language tags in statements because language tags are 
disjoint with URIrefs and blank nodes.}

\subsubsection{Serializations}

The \acro{RDF} data model is not bound to a specific syntax, but there a several
serialization formats. The most common format is \acro{RDF/XML} which uses 
\acro{XML} \cite{Beckett2004}. Suprisingly, some allowed RDF graphs cannot be 
expressed in \acro{RDF/XML} because they contain literals or predicate \acro{URI}s
with Unicode codepoints forbidden in \acro{XML}~1.0. There are also numerous alternative
ways to describe the same RDF graph in \acro{RDF/XML}, which makes it hard 
to use generic \acro{XML} tools to process general \acro{RDF} data.
\Term{TriX} (RDF Triples in XML) is an alternative \acro{XML} based syntax for
\acro{RDF}, that additionally provides for serializing several (named) graphs in
a single document \cite{Carroll2004}. \acro{RDF/JSON} and \acro{JSON-LD} are serializations of
\acro{RDF} in \tacro{JavaScript Object Notation}{JSON}, which is popular
in several scripting languages \cite{Alexander2008,Sporny2012}.
\Term{N-Triples} is a simple, plain text
serialization that was created for test cases \cite{Grant2004}. RDF graphs in
N-Triples are written one triple per line and the character set is 7-bit 
US-ASCII, but still the format is capable of encoding all \acro{RDF}. 
\Term{Turtle} (Terse RDF Triple Language) was created by \Person[David]{Beckett}
as more flexible and readable syntax extension of N-Triples \cite{Beckett2007}.
Turtle is probably the most popular \acro{RDF} serialization format next to 
\acro{RDF/XML}; an example is shown in figure~\ref{fig:rdfgraphexturtle}.
\Term{TriG} syntax \cite{Bizer2007,Carroll2005} extends Turtle by using curly 
brackets to group triples into multiple graphs, and to precede each graph by an
URIref as its name. Apart from minor syntax variants, that can be added 
automatically (an equal sign before and a dot after each graph), TriG is also 
compatible with the syntax of \Tacro{Notation3}{N3},\footnote{However both use
different models of \acro{RDF}: Trig is based on named graphs and N3 on standard
RDF with a custom \rdf{rei:} reification vocabulary.} which is another superset of 
Turtle. \acro{N3} extends Turtle with features such as variables, formulae, logical 
implications, and functional predicates, that can be used to abbreviate common 
URIrefs and patterns of \acro{RDF} statements \cite{BernersLee2008}. A summary 
of syntax elements of \term{Turtle} and \term{Notation3} is given in 
table~\ref{tab:n3syntax}.

\begin{figure}
\centering
\begin{minipage}{9cm}
\begin{lstlisting}[language=turtle]
@prefix xs: <http://www.w3.org/2001/XMLSchema#> .
@prefix dc: <http://purl.org/dc/elements/1.1/> .
@prefix foaf: <http://xmlns.com/foaf/0.1/> .

<info:oclcnum:318301686>
  dc:title "The C programming language"@en ;
  dc:date "1978"^^xs:gYear ;
  dc:creator [ foaf:name "Kernighan" ] ,
             [ foaf:name "Ritchie"   ] .
\end{lstlisting}
\end{minipage}
\caption{Simple RDF graph in Turtle syntax}
\label{fig:rdfgraphexturtle}
\end{figure}

\begin{table}
\centering
\begin{tabular}{|l|l|}
\hline
\textbf{Syntax element(s)} & \textbf{purpose} \\
\hline
\rdf{@prefix} & defines a namespace shortcut to abbreviate URIrefs \\
\rdf{@base}   & defines a standard prefix for URIrefs \\
\rdf{<X>} & URIref with URI \url{X} \\
\rdf{"..."}, \rdf{"""..."""} & literals \\ 
\rdf{"..."@X} & literal with language tag \rdf{X} \\ 
\rdf{"..."\^\^<X>} & literal with datatype URIref \url{X} \\ 
\rdf{.} & marks the end of a statement \\
\rdf{;} & following statement(s) have same subject \\
\rdf{,} & following statement(s) have same subject and predicate \\
\rdf{a} & shortcut for \rdf{rdf:type} \\
\rdf{_:X}, \rdf{[ ]} & blank node with local id \rdf{X} or without specific id \\
numeric literals & \rdf{xs:integer} and \rdf{xs:float} as datatype \\
\rdf{( )} & \rdf{rdf:List}, \rdf{rdf:first}, \rdf{rdf:rest}, and \rdf{rdf:nil}.\\
\rdf{#...} & comment \\
\hline
\rdf{=} & \rdf{owl:sameAs} \\
\rdf{!}, \rdf{^}, \rdf{@forSome} & statements with blank nodes \\
\rdf{=>}, \texttt{<=} & \rdf{log:implies} \\
\rdf{\{  \}} & statements with \rdf{rei:} reification and formulae \\
\rdf{@forAll} & \rdf{rei:universals} \\
\rdf{?x}, \rdf{:y} & variables in formulae \\
\hline
\end{tabular}
\begin{tabular}{l}
 \rdf{rdf:} is a shortcut for \url{http://www.w3.org/1999/02/22-rdf-syntax-ns#} \\
 \rdf{owl:} is a shortcut for \url{http://www.w3.org/2002/07/owl#} \\
 \rdf{xs:} is a shortcut for \url{http://www.w3.org/2001/XMLSchema#} \\
 \rdf{log:} is a shortcut for \url{http://www.w3.org/2000/10/swap/log#} \\
 \rdf{rei:} is a shortcut for \url{http://www.w3.org/2004/06/rei#} \\
\end{tabular}
\caption{Syntax elements of Turtle (above) and Notation3 (additionally below)}
\label{tab:n3syntax}
\end{table}

\subsubsection{Vocabularies}

A common technique used in all syntaxes (except \term{N-Triples}) is the abbreviation
of URIrefs with \term{namespace} prefixes. In practice it is often assumed, that all
resources, under one namespace prefix share same properties and that they belong to 
one common \Term{RDF vocabulary}. An \term{RDF vocabulary} is a set of URIrefs and
statements, that are created, described, and maintained for a specific use-case. 
If the vocabulary makes use of an RDF schema language like \acro{RDFS} or \acro{OWL}
(see section~\ref{sec:rdfschemas}), or if it implies other logical inference rules, 
the vocabulary can also be called an \term{ontology}.

The basic \acro{RDF} data model as described by \textcite{Klyne2004} includes 
only the predefined datatype URIref \rdf{rdf:XMLLiteral} for 
embedding \acro{XML} in \acro{RDF} literals. The standard further recommends 
to use datatypes from the \term{XML Schema} vocabulary
(see section \ref{sec:xmlschemas}). Other parts of the \acro{RDF} specification
\cite{Manola2004,Hayes2004,Brickley2004} provide a basic RDF vocabulary to
collect resources in classes (\rdf{rdf:type}), and resources that are used
as properties (\rdf{rdf:Property}). In addition the RDF vocabulary contains
resources to express containers (\rdf{rdf:Seq}, \rdf{rdf:Bag}, \rdf{rdf:Alt},
\rdf{rdf:_1}, \rdf{rdf:_2} \ldots), collections (\rdf{rdf:List}, 
\rdf{rdf:first}, \rdf{rdf:rest}, \rdf{rdf:nil}), primary values 
(\rdf{rdf:value}), and \Term{reification} (\rdf{rdf:Statement},
\rdf{rdf:subject}, \rdf{rdf:predicate}, \rdf{rdf:object}). Reification
is the description of \acro{RDF} triples using other \acro{RDF} triples.
This technique can be used for instance to express provenance and 
$n$-ary relationships, but it increases complexity and there are
several competing reification ontologies.\footnote{See 
\url{http://www.w3.org/TR/rdf-mt/\#Reif}, 
\url{http://www.w3.org/DesignIssues/Reify.html},
and \url{http://purl.org/ontology/prv/core\#} for other reification
ontologies.}

\subsubsection{Semantics}

A common misconception of \acro{RDF} is, that \acro{RDF} data automatically 
adheres to some semantics. The \acro{RDF} data model imposes no conditions
on the use of \acro{RDF} vocabularies to only create `meaningful' or 
`well-formed' \acro{RDF} graphs. On the contrary, an important principle 
of \acro{RDF} is that ``anyone can say anything about anything''.%
\footnote{This wording from the first \acro{RDF} concepts document
draft \cite{Klyne2004} was later modified to ``Anyone Can Make Statements About
Any Resource'' without changing the general declaration.} This means
``\acro{RDF} does not prevent anyone from making assertions that 
are nonsensical or inconsistent with other statements, or the world as 
people see it'' and ``it is not assumed that complete information about 
any resource is  available.'' \cite{Klyne2004}. The latter important 
principle is also known as \Tacro{Open World Assumption}{OWA}: the absence
of a particular statement from an \acro{RDF} graph does not mean that the 
statement is false. We illustrate this by the first triple of the graph 
in figure~\ref{fig:rdfgraphexample} and figure~\ref{fig:rdfgraphexturtle}.
The triple can be read as '`\url{info:oclcnum:318301686} is titled
`The C programming language' in English''. More precisely, it says
``the resource identified by URIref \url{info:oclcnum:318301686} has 
the English title `The C programming language', assuming the concept of 
having-a-title as identified by URIref \url{http://purl.org/dc/elements/1.1/}.''
However this statement does not imply the absence of parallel titles. The
resource may also have more than the two authors from the example --- 
we just only know that there are two authors named at least ``Kernighan'' 
and ``Ritchie'' respectively. Once we start inferencing, it could also
turn out to be one author with two names as shown in
figure~\ref{fig:rdfgraphinstances}.

The semantics that is usually associated with \acro{RDF}, does not origin 
from the \acro{RDF} data model, but from an algebra, that can be defined on 
\acro{RDF} graphs.\footnote{Some may argue, that the algebra is an
inherent part of \acro{RDF}. But this would neglect all \acro{RDF} 
applications, which do not fully implement all aspects of the \acro{RDF} 
algebra.} The algebra allows you to freely merge and 
intersect \acro{RDF} data based on simple set algebra. This is not possible 
in most other data structuring languages, for instance tree-based languages, 
which must have exactly one root element.%
\footnote{You could define union, intersection, and relative complement 
also for \acro{INI} files and some other record based data formats, but as described
in section~\ref{sec:csvandini}, these formats lack a precise definition.}
As blank nodes are always disjoint for different graphs, the simple set 
intersection of \acro{RDF} graphs cannot contain blank nodes. The same applies
to two \acro{RDF} graphs $A$ and $B$ with $A\equiv B$. The \acro{RDF} specification
uses the word `equivalent' in a different way, so we better call $A$ and $B$
`set-equivalent' or `identical' if $A\equiv B$. 

The \acro{RDF} specification defines another kind of equivalence and 
two additional relationships between \acro{RDF} graphs: \Term{graph equivalence}, 
\Term{graph instance}, and \Term{graph entailment}. Basically, two \acro{RDF} 
graphs $A$ and $B$ are \Term[equivalence (RDF)]{equivalent}, written as $A \cong B$,
if there is a bijection $M$ that maps literals to equivalent literals, URIrefs 
to equivalent URIrefs, blank nodes to blank nodes. In addition the triple
$(s,p,o)$ is in $A$ if and only if $(M(s),M(p),M(o))$ is in 
$B$ \cite[sec. 6.3]{Klyne2004}. It is recommended, but not required to 
apply Unicode Normalization before comparing graphs for equivalence.

An \acro{RDF} graph $H$ is called \Term[instance (RDF)]{instance of} an 
\acro{RDF} graph $G$, if $H$ can be obtained from $G$ by replacing zero 
or more of its blank nodes by literals, URIrefs, or other blank nodes.%
\footnote{The term `instance' is used also in
\term{RDFS} and \term{OWL} for the \rdf{rdf:type} URIref.} 
$H$ is a \Term{proper instance} of $G$, if at least one of its blank
nodes has been replaced by a non-blank node, or if at least two blank
nodes have been mapped to one. Two graphs $A$ and $B$ are equivalent 
if and only if both are instances of each other but neither a proper 
instance. Figure~\ref{fig:rdfgraphinstances} shows two possible
non-equivalent instances of the graph from figure~\ref{fig:rdfgraphexturtle}.

\begin{figure}
\begin{tabular}{ m{0.5\linewidth} m{0.5\linewidth} }
%Blank nodes replaced by URIrefs: &
%Two blank nodes merged to one: \\
\begin{lstlisting}[language=turtle]
@base <http://viaf.org/viaf/>.

<info:oclcnum:318301686> dc:title 
  "The C programming language"@en;
 dc:date "1978"^^xs:gYear;
 dc:creator <108136058>, <616522>.

<108136058> foaf:name "Kernighan".
<616522>    foaf:name "Ritchie".
\end{lstlisting}
&
\begin{lstlisting}[language=turtle]


<info:oclcnum:318301686> dc:title 
  "The C programming language"@en;
 dc:date "1978"^^xs:gYear;
 dc:creator _:b1.

 _:b1 foaf:name 
    "Kernighan", "Ritchie".
\end{lstlisting}
\\
blank nodes replaced by URIrefs &
two blank nodes merged to one
\\
\end{tabular}
\caption{Two examples of RDF graph instances}
\label{fig:rdfgraphinstances}
\end{figure}

\label{page:entailment} \Term{Entailment} is a relationship between two
\acro{RDF} graphs $A$ and $B$, that holds, if a graph equivalent to $B$ can be
created from $A$ by adding triples, based on specific inference rules. There is
not only one kind of entailment but a variety of \Term{entailment regimes} with
different sets of inference rules.\footnote{See
\url{http://www.w3.org/ns/entailment/} for a non-exclusive list of common
entailment regimes.} A specific entailment regime can be defined by an
ontology, by a set of inference rules, or by some application, that creates
entailments. If the application can handle arbitrary inference rules in some
rule language, it is also called a \Term{reasoner}. To distinguish different
entailment regimes, we say that $A$ \emph{x-entails} $B$, if $B$ is an
entailment of $A$ in regime $x$.  If some graph $G$ is not $x$-entailed by any
other graph, then $G$ can be called $x$-\Term{lean}.\footnote{The \acro{RDF}
specification only defines `lean' for simple entailment, but it can also be
defined for other entailment regimes. Finding lean graphs for a given regime is
an area of ongoing research \cite{Pichler2010}.}

The \acro{RDF} specification describes rules for simple-entailment,
rdf-entailment, and rdfs-entailment \cite[sec. 7]{Hayes2004}.
Simple entailment adds copies of existing triples by replacing
URIrefs of subject and/or object with blank nodes.\footnote{In Notation3:
\rdf{\{ ?s ?p ?o \} => \{ ?s ?p [ ] . [ ] ?p ?o \}}.} 
You can understand simple-entailment as generalization: an 
\acro{RDF} graph is always simple-entailment by all of its 
instances. \Term{rdf-entailment} extends this rule to literals. The
regime furthermore adds a rule that connect all URIrefs to 
\rdf{rdf:Property}, if they are used as property in some triple.% 
\footnote{In Notation3: \rdf{\{ ?s ?p ?o \} => \{ ?s rdf:type rdf:Property \}}.}

Entailment is an important aspect of \acro{RDF}, but it is not a feature of
\acro{RDF} data. The \acro{RDF} specification only says how to apply
entailment, but not whether and when to apply it.  In many cases inference is
expensive to calculate and would lead to a massive expansion of graphs. Some
regimes have infinite entailments also for simple graphs. Even the general
problem of determining simple-entailment between arbitrary RDF graphs is
NP-complete \cite{Hayes2004}, so most applications do not fully implement
entailment unless it is explicitly required. Testing graphs for equivalence
and instantiation is more common, but it also depends on entailment.
Entailment is also used to detect inconsistencies in \acro{RDF} data with
respect to some regime. For instance in \acro{OWL}, the triple 
\rdf{\{ ?x owl:differentFrom ?x \}} is a contradiction, that entails any 
possible triple, if \term{description logic} is applied. Reasoners for these 
entailment regimes usually detect such inconsistencies instead of infinitely 
adding triples.

The vision of the \term{Semantic Web} includes the idea of 
``intelligent agents'' that can aggregate information from distributed
sources and that can draw conclusions based on inferencing 
\cite{BernersLee2001a}. This idea requires decisions about which 
sets of \acro{RDF} data to combine, which entailment regimes to 
apply, and which URIrefs to rely on as non-ambiguous identifiers.
All these agreements are out of the scope of \acro{RDF}, which
alone is just another method of structuring data.

\begin{figure}
\centering
\begin{tikzpicture}[orm]
\entity[minimum width=1.8cm] (t) {};
\node[above=0 of t] {Triple};
\node[roles=3] (r) {};
\entity[below=6mm of r] (n) {Node};
\plays (n) to (r.one south) (n) to (r.two south) (n) to (r.three south);

\node[constraint=partition,below=of n] (nc) {} edge[inheritance] (n);

\entity[right=11.5mm of nc]  (u) {URI} edge[subtype] (nc);
\entity[below=10mm of nc] (l) {Literal} edge[subtype] (nc);
\entity[left=9.5mm of nc] (b) {Blank}  edge[subtype] (nc);

\node[roles=3,unique=1,right=6mm of l,yshift=6mm] (ld) {}; 
\node[roles=2,unique=1,right=8mm of l] (lv) {};
\node[roles=3,unique=1,right=6mm of l,yshift=-6mm] (ll) {};
\plays (ld.west) to (l) (lv.west) to (l) (ll.west) to (l);

\node[constraint=partition,below=of l.south east] (lp) {};
\limits (lp) to (ll.west) (lp) to (ld.west) (lp) to (lv.west);

\plays (ld.two north) to (u);

\value[right=10mm of lv] (s) {String};
\plays (s) to (ld.east) (s) to (lv.east) (s) to (ll.east);

%\value[right=6mm of ll.south east,yshift=-3mm] (lang) {LanguageTag};
\value[below=of ll,anchor=north west,xshift=-4mm] (lang) {LanguageTag};
\plays (ll.south) to ++(0,-4mm);

\value[right=14mm of u] (uv) {URIref};
%\draw[supinterface] (uv) to (s);
%\draw[supinterface] (lang) to (s);

\plays (uv) edge[both required] node (c) {} (u);
\binary[unique=1,unique=2] at (c) {};

%\draw[limits] (ld.one south) to node[constraint=x]{} (ll.one north);
\end{tikzpicture}
\caption{Model of generalized RDF}
\label{fig:genrdfmodel}
\end{figure}




% \input{ssec-zzstructures}

