\chapter{Patterns in data structuring}
\label{ch:patterns}

\begin{quotation}%
For they say the real is differentiated only by `rhythm' and `inter-contact'
and `turning'; and of this rhythm is shape, inter-contact is order, and
turning is position; for A differs from N in shape, AN from NA in order,
M from W in position.\\
\quotationsource \person{Aristotle}: \textit{Metaphysics}, Book I, part 4
(around 350 B.C.)% translated by W. D. Ross, written 350 B.C.E.
\end{quotation}

\noindent This chapter contains a language of fundamental patterns in data
structuring (for pattern languages in general see
section~\ref{sec:patterntheory}). Twenty fundamental patterns were identified
based on analysis from chapter~\ref{ch:methods} and findings from
chapter~\ref{ch:findings}. The pattern language is introduced with its
organization in section~\ref{sec:organization} before the actual patterns are
listed in four groups section~\ref{sec:basic-patterns} to
\ref{sec:continuing-patterns}.  Finally the language is evaluated by
comparison with related collections in section~\ref{sec:evaluation}.

\section{Organization}
\label{sec:organization}

In short, a \term{pattern} is a named description of ``a problem which occurs
over and over again in our environment'' \cite{Alexander1977}.  The pattern
guides to ways of solving the problem, independent from particular solutions.
Each pattern from this pattern language of data structuring highlights a
specific problem that occurs when data is actually organized. As mentioned
before, problems of data structuring should not be confused with particular
\term[data structure]{data structures} as technical solutions. A pattern refers
to particular methods of data structuring (see chapter~\ref{ch:methods}), but
it does not adhere to concrete implementations. Instead, each pattern shows
general strategies of solutions with its benefits, consequences, and pitfalls.

Each of the patterns is structured similarly to the design patterns presented by
\textcite{Gamma1994}, \textcite{Cunnigham1995}, and other pattern
languages.\footnote{An informal collection of pattern templates can be found in
WikiWikiWeb at \url{http://c2.com/cgi/wiki?PatternForms}. See also
\textcite{Meszaros1997} for patterns in pattern languages.} Each pattern
consists of four essential elements that imply a set of uniform sections:

\begin{itemize}

  \item The \emph{name} is a short label for referencing and describing
    the pattern. A good name should be easy to recognize and communicate
    the pattern. Additional well-known or helpful names of the pattern may
    be listed as \textbf{alias}. Note that a pattern name may have different 
    meanings and connotations in other context, so it should be taken as 
    technical term with its pattern language as controlled vocabulary. 

  \item The \emph{problem} describes when to apply the pattern. This
    description consists of the core \textbf{idea} with the 
    pattern's rationale and intent, the    \textbf{context} in which the 
    pattern can be applied and which imposes constraints on solutions,
    and a \textbf{motivation} to illustrate the problem.

  \item The \emph{solution} lists possible \textbf{implementations}
    as strategies to solve the problem. The solution can be illustrated
    by \textbf{examples} and \textbf{counter examples}. Note that there
    is rarely a single or best solution but each implementation has 
    its strengths and weaknesses.

 \item The \emph{consequences} are the results and trade-offs of
   applying the pattern. This includes possible \textbf{difficulties}
   and forces that must be taken into account when choosing a solution.
   Additional cross-references to \textbf{related patterns} include other 
   patterns that describe similar problems or patterns that may coexist
   unnoticed. Among related patterns there can be \textbf{implied patterns} 
   that appear together with a pattern, and 
   \textbf{specialized patterns} that solve a more specific problem.
\end{itemize}

\noindent
The pattern language is structured in four groups of patterns:

\begin{description}
\item[Basic patterns] include the most fundamental strategies of
	describing and structuring data elements (section~\ref{sec:basic-patterns}).
	These patterns are named \pattern{label}, \pattern{atomicity}, 
	\pattern{size}, \pattern{optionality}, and \pattern{prohibition}.

\item[Combining patterns] connect multiple data elements to larger structures
	(section~\ref{sec:combining-patterns}). These patterns are named 
	\pattern{sequence}, \pattern{graph}, \pattern{container}, 
	\pattern{dependence}, and \pattern{embedding}.

\item[Relationing patterns] relate data elements to each other to solve
	typical types of problems
	(section~\ref{sec:relationing-patterns}). These patterns are named
	\pattern{flag}, \pattern{derivation}, \pattern{encoding}, \pattern{identifier},
	\pattern{normalization}, and \pattern{schema}.

\item[Continuing patterns] indicate that more data exists
	(section~\ref{sec:continuing-patterns}). These patterns are named
	\pattern{separator}, \pattern{etcetera}, \pattern{garbage}, and \pattern{void}.
\end{description}

\noindent A classification of all patterns is summarized in the last chapter in
table~\ref{tab:patternclassification} and appendix~\ref{appendixC} includes a
graph to depict connections between patterns.


%\vfill
%\pagebreak
\section{Basic patterns}
\label{sec:basic-patterns}

Basic patterns highlight the the most fundamental strategies of describing and
structuring data elements. The patterns can be found anywhere at single data
elements. The patterns \pattern{label} and \pattern{atomicity} take an element
as such without further inspection. The patterns \pattern{size},
\pattern{optionality}, and \pattern{prohibition} also include the idea of
content which a data element is build of and which can be shaped in a specific
way for each of the patterns.

\input{datapatterns/basic}

\clearpage
\section{Combining patterns}
\label{sec:combining-patterns}

The primarily purpose of combining patterns is to connect multiple data
elements to larger structures. This combination can be done by several methods.
Combining patterns include \pattern{sequence} and \pattern{graph} which
structure multiple elements on the same level, and \pattern{container},
\pattern{dependence}, and \pattern{embedding} which include an idea of
subsuming elements.  As shown in figure~\ref{fig:compatterns}, combining
patterns are hierarchically connected to each other by general implications and
by the context they occur in (both connections shown by arrows).

The most fundamental and most abstract patterns are \pattern{embedding} and
\pattern{dependence}. More visible data combinations can be found as general
collections (\pattern{container}), such as sets of files and records, and in
form of ordered data (\pattern{sequence}) and graph structures
(\pattern{graph}). Together with these combining patterns one often finds two
basic patterns and two continuing patterns, \pattern{size} and
\pattern{atomicity}, which reflect a number of (possibly combined) elements and
the indivisible items to be connected, respectively. The continuing patterns
\pattern{separator} and \pattern{etcetera} are needed to indicate borders and
connections between elements and to indicate that a combination is incomplete.


\vfill
\begin{figure}[h]
\centering
\begin{tikzpicture}
\matrix[every node/.style={draw,ellipse,},column sep=3mm,row sep=5mm,
 ] {
  & & & \node (separator) {\pattern{separator}}; \\
  & \node (graph) {\textbf{\pattern{graph}}}; &
	\node (sequence) {\textbf{\pattern{sequence}}}; &
	\node (etcetera) {\pattern{etcetera}}; \\
  & \node (atomicity) {\pattern{atomicity}}; &
	\node (container) {\textbf{\pattern{container}}}; & \\
	\node (dependence) {\textbf{\pattern{dependence}}}; &
  \node (size) {\pattern{size}}; &
  \node (embedding) {\textbf{\pattern{embedding}}}; & \\
};
\draw[garc] (graph) to (container);
\draw[garc] (graph) to[out=-160,in=90] (dependence);
\draw[garc] (atomicity) to (size);
\draw[garc] (container) to (size);
\draw[garc] (container) to (embedding);
\draw[garc] (etcetera) to[out=-100,in=30] (embedding);
\draw[garc] (etcetera) to (container);
\draw[garc] (sequence) to (container);
\draw[garc] (separator) to[out=-20,in=0] (embedding);
\draw[garc] (separator) to (sequence);
\end{tikzpicture}
\caption{Connections between patterns (combining patterns in bold)}
\label{fig:compatterns}
\end{figure}
\vfill


\clearpage
\input{datapatterns/combining}

\clearpage
\section{Relationing patterns}
\label{sec:relationing-patterns} 

The following data patterns primarily relate elements to each other. In
contrast to combining patterns (section~\ref{sec:combining-patterns}), which
primarily group data to larger structures, the connections established by
relationing patterns are more between data elements which serve different
purposes. Each pattern solves a general problem in data structuring and
description, for instance complexity (solved by \pattern{encoding}) and
redundancy (solved by \pattern{normalization} and \pattern{derivation}).
Figure~\ref{fig:relpatterns} groups relationing patterns by connections of
implication or context in three levels: the most fundamental patterns include
\pattern{identifier} and \pattern{derivation}. Based on these patterns one can
find instances of \pattern{encoding} and \pattern{flag}, among other
structures. Finally instances of \pattern{normalization} and \pattern{schema}
are based on patterns of the second level. Figure~\ref{fig:relpatterns}
includes more patterns connected to relational patterns by informal implication
and context. A full diagram of connections is given in appendix
\ref{appendixC}.

\vfill
\begin{figure}[h]
\centering
\begin{tikzpicture}
\matrix[every node/.style={draw,ellipse,},column sep=3mm,row sep=5mm,
 ] {
 \node (optionality) 
 	{\pattern{optionality}}; &
 \node (garbage) 
 	{\pattern{garbage}}; &
 \node (prohibition) 
 	{\pattern{prohibition}}; \\
 \node (normalization) 
 	{\pattern{normalization}}; &
 \node (schema) 
 	{\textbf{\pattern{schema}}}; \\
 \node (encoding) 
 	{\textbf{\pattern{encoding}}}; &
 \node (flag) 
 	{\textbf{\pattern{flag}}}; &
 \node (void) 
 	{\pattern{void}}; \\
 \node (identifier) 
 	{\textbf{\pattern{identifier}}}; &
 \node (derivation) 
 	{\textbf{\pattern{derivation}}}; \\
}; % end of matrix
\draw[garc] (optionality) to (schema);
\draw[garc] (garbage) to (schema);
\draw[garc] (prohibition) to (schema);
\draw[garc] (normalization) to (encoding);
\draw[garc] (schema) to (flag);
\draw[garc] (flag) to (derivation);
\draw[garc] (encoding) to (identifier);
\draw[garc] (encoding) to (derivation);
\draw[garc] (void) to (derivation);
%\draw[garc,dashed] (flag) to (void);
%\draw[garc,dashed] (garbage) to (identifier);
\end{tikzpicture}
\caption{Connections between patterns (relationing patterns in bold)}
\label{fig:relpatterns}
\end{figure}
\vfill

\clearpage
\input{datapatterns/relational}

\clearpage
\section{Continuing patterns}
\label{sec:continuing-patterns}

Continuing patterns are easily overlooked because they don't show explicit
encoding. Instead these patterns primarily refer to data that is continued
elsewhere, possibly even at another level or another realm of description. The
most prominent continuing pattern is the \pattern{separator} which indicates a
border between data elements. The ``punctuation of data'' is often visible in
form of brackets, delimiters, and whitespace. The \pattern{etcetera} pattern is
less welcome because it shows that data is rarely complete. Continuation
markers such as `\verb|et al.|' still have their use because gaps and limits
would be hidden without them. In the end all data refers to something in the
realm of reality which is never fully encoded in data. The \pattern{garbage}
pattern can indicate missing data as well but in this case there is nothing
more to be encoded. Garbage values such as `\verb|n/a|` and `\verb|NULL|' make
irrelevant or inapplicable values explicit instead of just omitting them.
Omission on the other hand is the basic idea of the \pattern{void} pattern.  If
patterns would be given more colloquial nicknames, the continuing patterns
could also be named ``the glue`` (\pattern{separator}), ``the hint''
(\pattern{etcetera}), ``the ugly'' (\pattern{garbage}), and ``the mystery''
(\pattern{void}).

%\clearpage
\input{datapatterns/continuing}
\clearpage



\section{Evaluation}
\label{sec:evaluation}

Despite their popularity, there is little research on evaluation of pattern
languages and the pattern language paradigm itself
\cite{Petter2010,Dearden2006}.  A pattern language is meant to be a tool of
communication to describe typical solutions to common problems.  Evaluation can
therefore concentrate on: first whether and how well the language describes
typical solutions to common problems, and second whether and how well the
language communicates these problems and solutions. The second question can
best be answered with user studies which would go beyond the scope of this
thesis. The first question can be answered by direct examination of the
patterns and their problems and solutions.

To check whether the pattern language developed in this thesis reflects
experience in data description, the pattern language is compared to similar
collections and models. If similar approaches have led to similar solutions,
this is a strong indicator that the pattern language actually describes common
problems and solutions in data structuring and description.  Most existing
pattern languages such as \textcite{Gamma1994} are not comparable because they
do not refer to static digital documents but to dynamic behavior of information
systems. Literature review led to three models of data that are similar enough
to allow for comparison. These models will briefly be compared with the pattern
language in the following. Evaluation in greater depth will require more
feedback.  Evaluation of patterns, as suggested by \textcite{Petter2010}, is a
conscious continuous improvement activity, which should be applied to the
entire life-cycle of a pattern language. The current language should therefore
be taken as valid starting point for further improvement.

\subsection{Honig's analysis model of data structures}
\label{sec:honig}

Based on a review of 21 programming languages and data base management systems,
\textcite{Honig1975} in his thesis developed a general analysis model of data
structures. The model distilled the major differences out of existing data
types as possible axis of variation. Similar to my thesis, Honig's model
describes only ``the static, unchanging nature of data structures'' and it
refers to the \term{logical level}, independent from particular implementations
and naming. In contrast to the broad collection of data structuring methods in
chapter~\ref{ch:methods}, elements of the conceptual realm and data structures
from other domains, such as markup languages, conceptual modeling languages,
and conceptual diagrams are not explicitly included in Honig's model. The final
analysis model is a faceted classification with three major classes aggregate,
association, and file data structures. Each class has a number of facets,
represented as questions in appendix \ref{appendixA}. Table
\ref{tab:honigpatterns} compares Honig's classification with the patterns in
data structuring. As the pattern language is no classification, there is no
1-to-1 relationship but a loose mapping between both systems.  Their different
aims can best be summarized following. While Honig classified data structures
to tell out what they actually are, the pattern language of data structuring
tells what typical kinds of data structures actually mean.

Both aggregates and files from Honig's model are examples of the
\pattern{container} pattern as their basic idea is to combine a number of
elements to a larger structure. The first facet asks whether an aggregate
contains data elements of exactly one kind or type (homogeneous elements). The
kind or type can be given by a \pattern{flag}, a \pattern{schema} or a
\pattern{label}. The second facet tells whether elements are atomic and
indivisible, which is expressed by the \pattern{atomicity} pattern. The idea of
ordering elements corresponds to the \pattern{sequence} pattern (third facet)
and constraints on the number of elements are examples of the \pattern{size}
and the \pattern{schema} patterns (fourth pattern). Finally member elements can
be identified by different means: if the number or position is used, the
\pattern{sequence} pattern can be observed and identification by name is an
example of the \pattern{identifier} pattern. Element identification by pointer
is only mentioned briefly by \textcite[p. 146]{Honig1975} without giving more
examples. On a closer look it can be subsumed to identification by name.

The classification of file data types in Honig's model includes three facets,
the first of which having a subclass: file selection asks about what method is
used to pick one or more entry instances. This facet corresponds to element
identification and the \pattern{identifier} pattern. If unique entries are
specified by file selection, this is eventually based on
\pattern{normalization}. The second facet (sequential files) corresponds to the
\pattern{sequence} pattern and the third facet (kinds of entries) best matches
to the \pattern{flag} pattern.

Associations as third class of data structures basically express instances of
the \pattern{graph} pattern with some data elements as nodes and connecting
data elements as vertices. Cardinalities of connections (first facet) are
expressed by a \pattern{schema} that puts constraints on a \pattern{size}.
Kinds of end in an association (second facet) refer to the types of node
elements: For instance a kinship association may only exist between people
elements. Like the kinds of entries in file data types this question
can best be answered with the \pattern{flag} pattern but the
\pattern{derivation} may also be given if one element can have multiple types.
Associations with loops allowed correspond to general \pattern{graph}
structures which may be constrained by a \pattern{schema}. Finally complete
associations (fourth facet) and exclusive associations (sixth facet) impose
mandatory and/or uniqueness constraints on their member elements which 
maps to the \pattern{optionality} and to the \pattern{flag} pattern respectively.


\begin{table}[ht]
\centering
\begin{tabular}{lll}
& \textbf{dimension/axis} & \textbf{patterns} \\
\hline
\multicolumn{2}{l}{\textbf{aggregates}} & \pattern{container} \\
\hline
& homogeneous elements          & \pattern{flag}, \pattern{schema}, \pattern{label} \\
& basic item elements           & \pattern{atomicity} \\
& ordered elements              & \pattern{sequence} \\
& number of elements (fixed, limited, unbounded) & \pattern{size}, \pattern{schema} \\
& element identified by number  & \pattern{sequence} \\
& element identified by name    & \pattern{identifier} \\
& element identified by pointer & \pattern{identifier} \\
\hline
\multicolumn{2}{l}{\textbf{files}} & \pattern{container} \\
\hline
& file selection & \pattern{identifier} \\
& unique entries  & \pattern{normalization} \\
& sequential file & \pattern{sequence} \\
& kinds of entries & \pattern{flag} \\
\hline
\multicolumn{2}{l}{\textbf{associations}} & \pattern{graph} \\
\hline
& cardinality (1-1, 1-n, n-m) & \pattern{schema}, \pattern{size} \\
& kinds of ends & \pattern{flag}, \pattern{derivation} \\
& loops allowed & \pattern{graph}, \pattern{schema} \\
& complete associations  & \pattern{optionality} \\
& exclusive associations & \pattern{flag} \\
\hline  
\end{tabular}
\caption{Comparison of patterns with Honig's model (\citeyear{Honig1975})}
\label{tab:honigpatterns}
\end{table}


\subsection{Quarks of Object-Orientation}

By literature analysis in the field of \term{Object Orientation}
\textcite{Armstrong2006} identified fundamental concepts that define the
\term{Object Oriented} development approach. She found eight concepts that are
mentioned in more then half of the sources, and put them in a simplified Object
Orientation taxonomy. The most mentioned concepts are: abstraction, classes,
encapsulation, inheritance, objects, message passing, methods, and
polymorphism. Table~\ref{tab:ooquarkpatterns} shows the structural part of the
taxonomy and maps the \acro{OO} concepts to similar patterns:

Abstraction is used on object orientation to simplify something by
concentrating on relevant distinctions only. This methods is found in the
\pattern{normalization} pattern. A class in object orientation combines a
\pattern{label} to be referenced and interpreted by humans and a
\pattern{schema} with possibilities and constraints. Encapsulation hides
details of implementation and provides a simplified form of access.  This
concept maps to both the \pattern{atomicity} pattern and the
\pattern{encoding} pattern. Inheritance connects classes by extension or
restriction, which imply an application of the \pattern{derivation} pattern.
The object concept finally conveys the idea of identifying a single instance,
which is expressed by the \pattern{identifier} pattern.  The OO concepts of
message passing, methods, and polymorphism refer to behavior instead of
structure so they are left out in the comparison. It should only be noted that
data patterns are inherently polymorph: a single patterns rarely describes the
nature of one data object but often there are multiple patterns as possible
viewpoints to the same artifact.

%\clearpage
\begin{table}[p]
\begin{tabularx}{\textwidth}{lXX}
\hline
\textbf{concept} & \textbf{definition} & \textbf{patterns} \\
\hline
Abstraction & Creating classes to simplify aspects of reality using 
distinctions inherent to the problem.
& \pattern{normalization}
\\
Class & A description of the organization and actions shared by one 
or more similar objects. 
& \pattern{label}, \pattern{schema}
\\
Encapsulation & Designing classes and objects to restricts access to 
the data and behavior by defining a limited set of messages that an 
object of that class can receive.
& \pattern{atomicity}, \pattern{encoding}
\\
Inheritance & 
The data and behavior of one class is included in or used as the basis 
for another class.
& \pattern{derivation}
\\
Object & 
an individual, identifiable item,
either real or abstract, which contains data about itself 
and descriptions of its manipulations of the data.
& \pattern{identifier}
\\
Message Passing & the process by which an object sends data
to another object or asks the other object to invoke a
method. & --- \\
Method & A way to access, set, or manipulate an object's information & --- \\
Polymorphism & Different classes may respond to the same message and 
each implement it appropriately. & --- \\
\hline
\end{tabularx}
\caption{Quarks of Object-Orientation compared to patterns}
\label{tab:ooquarkpatterns}
\end{table}
\clearpage

\subsection{ISO 11404}

ISO~11404 (\citeyear{ISO11404}) is one of the rare standards that deal with
\term{data type}s independent from particular (programming) languages. Lessons
learned during its specification have been summarized by
\textcite{Meek1995,Meek1996}.  An overview of ISO~11404 is given in
section~\ref{sec:datatypes}:  the standard defines three notions of data types,
a set of general datatype properties, a collection of common primitive data
types, and some derivation methods to create new types based on existing ones.
By this means ISO~11404 defines a set of abstract data type classes of which
the actual datatypes used in programming languages are derived. A comparison
of these abstract classes and data description patterns is given in
table~\ref{tab:iso11404patterns}. The comparison is explained in the
following.

First, ISO~11404 distinguishes the fundamental notions \term{value space},
\term{value representation}, and \term{computational model} of a data type. The
latter is not relevant to this thesis because it refers to dynamic properties.
The relation between value space and value representation is found with the
\pattern{encoding} pattern as basic semiotic relation between signifier and
signified. The equality property is an instance of the \pattern{normalization}
pattern while order is an applications of the \pattern{sequence} pattern.
Upper and lower bounds restrict an ordered set with constraints
(\pattern{schema} pattern).  Cardinality in ISO~11404 refers to the property of
having either a finite value space, to a denumerable infinite value space
($\aleph_0$) or to an approximate finite or innumerably infinite value space (such
as floating point numbers to represent $\mathbb{R}$). Finite and innumerably
infinite data types must have some known or derivable \pattern{size} while
approximate types require a \pattern{normalization}. Numeric types combine
cardinality with an ordered set, so at least a \pattern{sequence} can be found.

Primitive data types from the standard have been summarized in
table~\ref{iso11404primitives}. A comparison with data patterns shows that
these primitive data types share a set of basic ideas, expressible by a small
number of patterns: some basic data types are direct examples of one data
patterns, such as Boolean for \pattern{flag}, Character for \pattern{encoding},
Integer for \pattern{size} pattern and Void (also known as unit type) for
\pattern{void}. Other types share the same fundamental pattern like Enumerated
and Ordinal with the \pattern{sequence} pattern. The remaining primitive data
types combine multiple patterns, the combination established by
\pattern{embedding} or \pattern{encoding}.

Finally, the general derivation methods, identified in ISO~11404 and further
described in section~\ref{sec:datatypes} can be mapped to patterns or to
combinations of patterns: pointer types are kind of identifiers as secondary
elements (\pattern{identifier} and \pattern{dependence}). Choice types give a
direct example of the \pattern{flag} pattern. Aggregation types combine
multiple elements to one (\pattern{container}), possibly with additional
structure (\pattern{embedding}). Subtypes, like inheritance from the quarks of
Object-Orientation, implement a form of \pattern{derivation} (properties of a
subtype can be derived from properties of supertypes).

\begin{table}[ht]
\centering
\begin{tabular}{ll}
\hline
\textbf{notions} & \textbf{patterns} \\
\hline
value space/representation & \pattern{encoding} \\
\hline
\textbf{properties} & \textbf{patterns} \\
\hline
equality          & \pattern{normalization} \\
order             & \pattern{sequence} \\
upper/lower bound & \pattern{schema} \\
cardinality       & \pattern{size}, \pattern{normalization} \\
numeric types     & \pattern{sequence} \\
\hline
\textbf{primitive data types} & \textbf{patterns} \\
\hline
Boolean    & \pattern{flag} \\
State      & \pattern{flag}, \pattern{encoding} \\
Enumerated & \pattern{sequence} \\
Character  & \pattern{encoding} \\
Ordinal    & \pattern{sequence} \\
Date-and-Time & \pattern{sequence}, \pattern{embedding} \\
Integer    & \pattern{sequence} \\
Scaled (fixed point) & \pattern{size}, \pattern{embedding} \\
Real       & \pattern{size}, \pattern{embedding} \\
Complex    & \pattern{size}, \pattern{embedding} \\
Void       & \pattern{void} \\
\hline
\textbf{derivation methods} & \textbf{patterns} \\
\hline
    pointer types & \pattern{identifier}, \pattern{dependence} \\
     choice types & \pattern{flag} \\
aggregation types & \pattern{container}, \pattern{embedding} \\
         subtypes & \pattern{derivation} \\
\hline
\end{tabular}
\caption{ISO~11404 concepts and data patterns}
\label{tab:iso11404patterns}
\end{table}


