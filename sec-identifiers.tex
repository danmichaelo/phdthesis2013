\section{Identifiers}
\label{sec:identifiers}

\begin{quotation}%
What's in a name? that which we call a rose by any other name would smell as sweet.\\
\quotationsource{\Person[William]{Shakespeare}: Romeo and Juliet}
\end{quotation}

% Identifiers are used throughout all other encoding methods. They are used to
% solve the problem of identification in a very limited context. In a broader
% context this problem is much harder to tackle as shown in
% section~\ref{sec:identityandequality}.

\noindent While character and number encodings are used as base of data
structuring, \term{identifier}s virtually pervade systems to structure and
describe data on all levels. This section first introduces basic identifier
principles (section~\ref{sec:idbasics}) followed by properties of namespaces and
qualifiers (section~\ref{sec:qualifiers}) which identify the context of
association between identifier and referred thing. Identifier systems
(section~\ref{sec:idsystems}) provide an infrastructure in which identifiers are
assigned, managed, and used. Part~\ref{sec:descrids} on descriptive identifiers
and section~\ref{sec:orderedids} on ordered identifiers highlight two important
but often overlooked properties of identifiers on a more theoretical level.
Finally hash codes as special kind of distributed identifier systems are
explained in section~\ref{sec:hashes}. A summarized overview of designated
identifier properties is given in table~\ref{tab:idprops}.

\subsection{Basic principles}
\label{sec:idbasics}

In its most general form, a digital identifier is a piece of data (string,
number, letter, symbol, etc.) that refers to an object. This makes
identifiers a special type of metadata which more generally describe objects.
In contrast to general metadata, an identifier should be unique (no homonyms), 
persistent and short, at least in some context. Distinct objects must have 
distinct identifiers to avoid ambiguity, and the number of identifiers that 
refer to the same object (synonyms) should be low for practical reasons. The 
following analysis is limited to digital identifiers in their general form as 
finite sequences of characters. Examples of identifiers from the previous 
section include number encodings that refer to the mathematical concepts of 
numbers, byte sequences that refer to Unicode code points, and Unicode code 
points that refer to characters. It is shown that this forms -- also known as
names, labels, locators, codes, or pointers -- only make the visible part of
an identifier. The concluding example of Pica format field identifiers will 
then illustrate some basic properties of data identifiers. 

Most literature on identifiers deals with selected types of identifiers
or with special aspects, such as identifier persistence. More general 
discussions are provided by 
\textcite{Eriksson2010}, % Rethinking the Meaning of Identifiers in...
\textcite{Campbell2007} % Identifying the identifiers
\textcite{Coyle2006}, 
\textcite{Vitiello2004}, 
\textcite{Lynch1997},
and \textcite{Kent1991}. The authors define identifiers as data objects that
``refer to'', ``reference'', ``represent'', or ``serve as surrogate for'' 
other objects, but the general idea is the same: a relatively short
piece of data is associated with another (data) object. 
\textcite{Campbell2007} 
provides a more detailed deconstructions of identifiers in six parts:

\begin{enumerate}
\item a ``thing'' that is referenced
\item a ``symbol'' that references the thing. Unless otherwise indicated,
  this symbol in particular is meant by the word ``identifier''
\item an ``association'' between the symbol and the thing
\item a ``context'' that the association occurs within
\item an ``agent'' that states the association and context
\item a ``record'' of the association, context, and ideally also the agent
\end{enumerate}

In the following the same terminology will be used. Symbol, thing, and
association are typically found as ``symbol'', ``referent'', and ``thought'' or
under different names in the \term{semiotic triangle} \cite{Ogden1923} as
described in section~\ref{sec:semiotics}.  The important aspect for the
analysis of identifiers is that there is no direct connection between symbol
(signifier) and thing (signified).  It requires an association that always
depends on some context, is established by an agent, and may or may not be
recorded.

Following the philosophical position of radical \term{constructivism}
\cite{Glasersfeld1990}, one can even neglect the thing, as there is no direct
access to real-world objects by language. This also applies to digital
identifiers about things in the reality realm (in terms of data modeling).
However, digital identifiers about digital objects can be compared with their
referents, because both are recorded in data. In fact digital records of
identifiers and their referents are common practice in many data structures and
known as \Term[lookup table]{lookup tables}. In some cases one can even swap
symbol and referent, because both are unique on their side of the table (see
table~\ref{tab:pica3pp} for an example).

Unambiguity (each identifier must refer to only one object) and uniqueness
(for each object there should be only one identifier) of often combined as
\Term{uniqueness} as most important requirement for an identifier.
Other properties frequently cited as important qualities are \Term{persistence}
(identifiers should not change over time), \Term{scope} (the context of an
identifier should be broad or even `global'), \Term{readability} (identifiers
should be easy to remember or contain information), and \Term{actionability}
(given an identifier one should be able to do something with it,
for instance access the identified object). A summary of these and more
designated identifier properties is given in table~ \ref{tab:idprops}.
To ensure the required
properties it needs an \term{identifier system} as controlled mechanism 
or convention for creating, managing, and using identifiers 
(see part~\ref{sec:idsystems}). When we analyze general data, identifier 
properties can also indicate whether some piece of data is actually used 
as identifier or not: being an identifier is nothing inherently inscribed 
in data, but it is an example of a popular data pattern, that is used all
all over systems to structure and describe data. 

\subsectionexample{Field identifiers in PICA format}
\label{ex:picafieldids}

\begin{table}[h]
\centering
\begin{tabular}{r|l|l|l|l|l}
description & year        & title       & edition     &  place and publisher & DDC \\
\hline
Pica3 tag   & \verb|1100| & \verb|4000| & \verb|4020| & \verb|4030| & \verb|5010| \\
Pica+ tag   & \verb|011@| & \verb|021A| & \verb|032@| & \verb|033A| & \verb|045F| \\
repeatable  & no & no & no & yes & yes \\
\end{tabular}
\term*{Pica}
\caption{Some fields of Pica record format}
\label{tab:pica3pp}
\end{table}

The bibliographic record format \Term{Pica} consists of a simple 
field-subfield-structure, similar to \term{MARC} (compare with figure 
\ref{fig:marcrecord}). Each field can be identified by a so called
Pica3 tag or by a Pica+ tag. Figure~\ref{tab:pica3pp} lists a lookup table 
for some fields with their tags and a repeatability flag from the cataloging 
rules of the \acro{GBV} library network \cite{Pica2010}. 
Assuming that the
fields somehow refer to things in the reality realm, we cannot directly
map from tags to these things. Fields like ``place and publisher'' also
indicate that the referent can be quite artificial: most people know places
and maybe publishers, but what kind of ontological status has their 
combination? Beside the intangible referent of such artificial identifiers, 
textual descriptions make no good identifiers, because one can write them in
several forms (the original descriptions are given in German) and because 
there may be different fields with same description. Tags in contrast can 
at least identify field descriptions and tags of the other kind (Pica3 to 
Pica+ and vice versa), because they all exist in the data realm. The 
repeatability flag finally identifies a thing from the conceptual realm, 
namely the set of fields, that are repeatable (\verb|yes|) or not repeatable
(\verb|no|). In practice one must always take in mind, in which context an identifier 
is used. A Pica3 tag like \verb|5010| may refer to the corresponding
Pica+ flag \verb|045F|, to a concrete set of field value, or to the abstract
concept of the field. In data structuring we often deal with cascaded 
identifiers that only link to the reality realm in a last step. For instance 
\verb|045F| refers to the ``DDC'' field, which in a bibliographic records 
contains a notation from the \tacro{Dewey Decimal Classification}{DDC},
which by itself is another identifier. % TODO: see part ...


\subsection{Namespaces and qualifiers}
\label{sec:qualifiers}
% there seems to be no general literature on namespaces (!)

\Term[namespace]{Namespaces} and \Term[qualifier]{qualifiers} are both used to
avoid the problem of \term{homonymy}. In addition they can provide context and
refer to authority through a hierarchy of identifiers. A possible term to 
describe both is \Term{qualified identifier}. Qualified identifiers are used in 
formal systems like programming languages and knowledge organization systems
(thesauri, classifications, authority files etc.) where a name must always 
refer to one distinct object. A defined syntax in a formal language is needed 
to separate the namespace or qualifier part and the local part of a qualified 
identifier. Otherwise it would be ambiguous for instance whether `band-spectrum'
refers to a `band' of radio communication frequencies, or to the Australia band 
`spectrum' which formed in 1969, or whether the minus sign does not indicate
the existence of a namespace qualifier at all. Namespaces are typically 
prepended to the local identifier and qualifiers are added in parentheses
(see example~\ref{ex:contextids}). If the context is known as by definition of 
a \Term{default namespace}, one may also omit the qualifier or namespace.

\begin{example}[h]
\centering
\begin{tabular}{l|l|l|l|l}
identifier & local & qualifier & syntax & description \\
\hline
\texttt{Frankfurt/Main} & \texttt{Frankfurt} & \texttt{Main} 
 & \texttt{$L$/$Q$} & city name \\
\texttt{Dublin, Ohio} & \texttt{Dublin} & \texttt{Ohio} 
 & \texttt{$L$, $Q$} & city name \\
\texttt{US-OH} & \texttt{OH} & \texttt{US} 
 & \texttt{$Q$-$L$} & ISO 3166-2 area code \\
\texttt{std::set} & \texttt{set} & \texttt{std} 
 & \texttt{$Q$::$L$} & C++ identifier \\
\texttt{rdf:type} & \texttt{type} & \texttt{rdf} 
 & \texttt{$Q$:$L$} &  \acro{URI} reference in \acro{RDF} \\
\texttt{10.1000/182} & \texttt{1000/182} & \texttt{10} 
 & \texttt{$Q$.$L$} &  \acro{DOI} as specific Handle \\
\texttt{sgn-US} & \texttt{US} & \texttt{sgn} 
& \texttt{$Q$-$L$} &  \acro{IANA} language \& subtag\\
\end{tabular}
\caption{Qualified identifiers with local part ($L$) and qualifier ($Q$)}
\label{ex:contextids}
\end{example}

For some prefixed types of namespaces, the qualifier is not fixed, but
can be replaced by another prefix for the same namespace. For instance
\texttt{http://dx.doi.org/} and other known resolver addresses are actually 
used as prefix for the namespace of \Tacro{document object identifiers}{DOI}
but one could also use the \acro{DOI} as \Tacro{Uniform Resource 
Identifier}{URI} with prefix \texttt{info:uri/}. Another example is 
a prefixed element name in the \term{Extensible markup language} (\acro{XML}, 
see section~\ref{sec:xml}): some \acro{XML} applications ignore the prefix and
respect a locally defined namespace, some ignore the prefix, and some need both
to identify an element (see example~\ref{ex:xmlns} at page \pageref{ex:xmlns}). 
Finally, there are
systems in which a namespace is just an abbreviation that can be defined
and expanded as needed (see the \texttt{@prefix} statement from 
\term{Notation3} as described in section~\ref{sec:rdf} and 
table~\ref{tab:n3syntax}).

Qualifiers, more often than namespaces, may also encode a special meaning,
especially when they are used for syntactic indexing in knowledge organization
languages. For instance the identifier \texttt{Marx, Karl, 1818-1883} from 
the Library of Congress name authority file include the qualifier 
\texttt{1818-1883}. This qualifier specifies the years of birth and death of 
the identified person.\footnote{The first part of the identifier
(\texttt{Marx}) may also be seen as a namespace of all people's identifier 
with this surname, and the local part \texttt{Karl} signifies a given personal 
name, so in this example all parts encode some meaning if one takes person 
names as meaningful.} In other cases the primary role of a qualifier is to 
disambiguate, so one is more free to choose, for instance \texttt{Paris~(city)} 
and \texttt{Paris~(mythology)}, \texttt{Paris~(place)} and \texttt{Paris~(person)},
or just \texttt{Paris~(1)} and \texttt{Paris~(2)} for two distinct concepts. 
A general problem of meaningful qualifiers is the limitation to one aspect. 
For instance there are multiple early computers referred to as ``Mark~I''. 
One can either use their location as qualifier (\texttt{Mark~I~(Harvard)} and
\texttt{Mark~I~(Manchester)}) or their year of completion 
(\texttt{Mark~I~(1944)} and \texttt{Mark~I~(1949)}). Combining multiple
aspects quickly gets complicated, similar to nesting of multiple namespaces
in one mono-hierarchical system, as described in the next part.

Above all, namespaces and qualifiers do not solve the general problem of
identification but they only shift it to another domain, level, or authority.
Namespaces and qualifiers only draw aside avoid homonymy and provide context 
in some known identifier system and both are identifiers in their own right. 
To avoid an infinite chain of qualified identifiers one hasto start at some 
authority as root context, which is also known as the global namespace.

\subsection{Identifier Systems}
\label{sec:idsystems}
\label{sec:dns}

\begin{quotation}%
URIs don't change. People change them.\\
\quotationsource{\Person[Tim]{Berners-Lee} (\citeyear{BernersLee1998})}
\end{quotation}

\noindent
All identifiers are artificially created -- either explicit by naming
or implicit by providing a mechanism that creates identifiers. An 
\Term{identifier system} defines which identifiers exist (registry); or how
identifiers are created and managed (assignment politics); how recorded 
associations between identifier and referent can be looked up (resolving); 
which syntax rules as naming conventions apply (grammar); or which relations 
to other identifier systems exist. Popular digital identifier system covered 
as examples in the following are: \tacro{Uniform Resource Identifier}{URI}, 
its counterpart \tacro{Internationalized Resource Identifier}{IRI}, 
\tacro{Uniform Resource Locator}{URL}, \tacro{Uniform Resource Name}{URN},
\tacro{Domain Name System}{DNS}, \tacro{Internet Protocol}{IP},  
\tacro{International Standard Book Number}{ISBN}, and
\tacro{European Article Number}{EAN}. Relations between these systems, 
together with their character encodings are illustrated in 
figure~\ref{fig:idsysrel}.

\begin{figure}
\centering
\begin{tikzpicture}[line width=0.25mm]
\entity (URI) {URI}
 [edge from parent/.style=subtype,nodes=entity]
 child { node (URL) {URL} }
 child { node (URN) {URN} };
\entity[below=6mm of URN] (URNISBN) {URN-ISBN}; 
\draw[subtype] (URN) to (URNISBN);

\entity[below=6mm of URL] (IP) {IP};
\entity[left=5mm of IP] (DNS) {DNS};
\draw[dotted,->] (IP) -> (DNS);
\draw[dotted,->] (URL) -> (DNS);
\draw[dotted,->] (URL) -> (IP);
\draw[<->] (DNS) to (IP);

\limits ($(URI)!.65!(URL)$) to 
 node[constraint=exclusive] {} ($(URI)!.65!(URN)$);

\entity[above=6mm of URI] (IRI) {IRI};
\entity[right=14mm of URI] (ASCII) {US-ASCII};
\entity[above=6mm of ASCII] (UCS) {UCS};

\draw[dotted,->] (IRI) to (UCS);
\draw[dotted,->] (URI) to (ASCII);

\draw[suptype] (ASCII) to (UCS);
\entity[right=5mm of URNISBN] (ISBN) {ISBN}
 [edge from parent/.style=subtype,nodes=entity,
 sibling distance=19mm]
 child { node (ISBNa) {ISBN-13} }
 child { node (ISBNb) {ISBN-10} };
\limits ($(ISBNa)!0.35!(ISBN)$) to 
 node[constraint=exclusive] {} ($(ISBNb)!0.35!(ISBN)$);
\draw[<->] (ISBNa) to (ISBNb);

\entity[left=7mm of ISBNa] (EAN) {EAN};
\draw[subtype] (EAN) to (ISBNa);

\draw[<->] (ISBN) to (URNISBN);
\draw[<->] (IRI) to (URI);

%\draw[dotted,->] (ISBN) -> (ASCII);
\begin{scope}[xshift=48mm,yshift=-12mm]
\draw[dotted,->] (-6mm,2mm) to (-1mm,2mm);
\node[anchor=west] at (0,2mm) {depends on};

\draw[subtype,->] (-6mm,-5mm) to (-1mm,-5mm);
\node[anchor=west] at (0,-5mm) {subset of};

\node[constraint=exclusive] at (-3mm,-11mm) {};
\node[anchor=west] at (0,-11mm) {exclusion};

\draw[<->] (-6mm,-18mm) to (-1mm,-18mm);
\node[anchor=west] at (0,-18mm) {(partial) mapping};
\end{scope}
\end{tikzpicture}
\caption{Relations between several identifier systems}
\label{fig:idsysrel}
\end{figure}

In general any sequence of bits or other digital symbols can act as 
digital identifier. To define a possible set of sequences, an identifier system 
includes a formal language (see~\ref{sec:formallanguages})
as identifier syntax (see section~\ref{sec:formallanguages}). Every identifier 
symbols must conform to this syntax, so its grammar rules help to discover and 
use identifiers: with a well-defined syntax one does not need to resolve each 
string to check whether it is an identifier, but one can validate possible 
identifiers based on their shape. Often only parts of the grammar are defined in 
a schema language (see section~\ref{sec:bnf}), and there may be additional 
informal agreements to exclude some sets of characters and sequences. For 
instance whitespace characters are less used in identifiers: even if allowed,
at least multiple consecutive whitespace characters are not encountered in 
practice. If identifier system depend or reuse each other, for instance 
as namespaces and qualifiers (see the dotted relations in 
figure~\ref{fig:idsysrel}), there can be difficulties to embed one identifier
within the syntax of the other. If such an embedding may lead to 
disallowed character sequences or to syntax ambiguities, the host identifier 
system usually defines a method to escape the embedded identifier. A typical 
escaping mechanism, is \Term{percent-encoding}. A character code of one byte 
in percent encoding is replaced by the percent character ``\verb|%|'' followed
by the two hexadecimal digits representing that byte's numeric value
\cite[section~2.1]{BernersLee2005}. However, the question when and which parts
of an identifier must be encoded, is a frequent source of confusion. Quite
often an identifier reuses another identifier that already includes an 
embedding, so one ends up with a complex hierarchy of dependencies and 
nested encodings.

If readability is no primary requirement, digital identifiers can be plain 
numbers or sequences of bits (see part~\ref{sec:orderedids}). An example are 
\acro{IP} addresses, which can be mapped to more readable \acro{DNS} names.
But even a very simple identifier syntaxes such as number ranges can cause 
problems: in 2011 the \acro{IP} system version 4 ran out of identifiers because
it was limited to $2^{32}$ distinct numbers. The update to \acro{IP} version 6 
takes several years, because version 4 is used in many other specifications and 
implementations that must also be updated accordingly. A similar but less 
extensive update of an identifier
system was the extension of \term{International Standard Book Number}s
from ten digits (nine plus one check digit) to thirteen (nine plus 
bookland namespace and check digit). The assignment of ISBN identifiers
is delegated to national agencies who then delegate it to publishers. Therefore
the system is not always used as intended: in theory an ISBN can 
never be reused and every edition of a title must have a new ISBN. In practice
new editions often reuse the ISBN of the previous edition and some publishers 
even assign existing ISBNs to totally different books.\footnote{For instance
ISBN~3-453-52013-0 was used for two unrelated books by Heyne-Verlag
in 1974 and 2004 and ISBN~3-257-21097-3 was assigned to every single work 
of \Person[B.]{Traven} published by Diogenes-Verlag.} The update from 
ISBN-10 to ISBN-13 was based on an already existing encoding  of ISBN-10 in
the \tacro{European Article Number}{EAN}. Nevertheless, it required a lot 
of marketing and modifications in library 
systems \cite{vanHalm2005}. Among other things difficulties resulted 
from different syntaxes to express equivalent ISBNs (example~\ref{ex:isbns}).
I fact all syntaxes (except ISBN-13 with bookland namespace \texttt{979}) 
can be mapped to each other, so it is an arbitrary decision, which form to 
take as the `real' ISBN. Similar problems of synonymy are also present in
other identifier systems.

\begin{example}[h]
\centering
\begin{tabular}{l|l}
ISBN-10 with hyphen & \texttt{1-4909-3186-4} \\
ISBN-10 with space  & \texttt{1 4909 3186 4} \\
plain ISBN-10 & \texttt{1490931864} \\
EAN & \texttt{9781490931869} \\
EAN barcode aligned & \texttt{9 78149 093186 9} \\
ISBN-13 with hyphen & \texttt{978-1-4909-3186-9} \\
plain ISBN-13 & \texttt{9781490931869} \\
URN-ISBN & \texttt{URN:ISBN:1-4909-3186-4} \\ % (RFC 3187)
\end{tabular}
\caption{Different syntaxes that express equivalent ISBNs}
\label{ex:isbns}
\end{example}

Today the most used identifier systems, apart from character encodings, are
\Tacro{Uniform Resource Location}{URL} (referred to as web addresses in common
speech), \Tacro{Uniform Resource Identifier}{URI}, and \Tacro{Uniform Resource
Name}{URN}. These systems are often confused because they all evolved together 
with the \Tacro{World Wide Web}{WWW} and the \Tacro{Hypertext Transfer 
protocol}{HTTP}. The \acro{WWW} was introduced by \Person[Tim]{Berners-Lee} 
in 1990. His first design includes considerations on document naming as 
``probably the most crucial aspect of design and standardization in an open 
hypertext system''. \textcite{BernersLee1991} discusses
addressing, naming, and uniqueness as three different properties and 
introduces \acro{URL} to cover the addressing as part of a global naming system. 
\acro{HTTP} as defined by
\textcite{BernersLee1992} allowed for use of different types of ``Universal Resource 
Identifiers'', but only listed \acro{URL} and other addressing schemes (\acro{FTP},
\term{gopher}, etc.) as examples.\footnote{An exception was the 
\Tacro{Content-ID}{cid} scheme which did not include a server as 
physical address. \acro{cid} was later specified as 
``URL scheme'' (sic!) with RFC~2111 (1997) and RFC~2392 (1998) but it never
got fully adopted in practice. However, it is a good example of a mapping
between an identifier system and a key-value structure: for instance the identifier 
\texttt{cid:foo} corresponds to the \acro{MIME} header 
\texttt{Content-ID: <foo>} with field name \texttt{Content-ID} and 
field value \texttt{foo}.} In 1992 an \acro{URI} working group was formed to
define other types of ``Uniform Resource Identifiers'' \cite{Emtage1992}
but it was difficult to come to consensus. The working group concluded in 1995 
with several recommendations after \textcite{BernersLee1994a} had published 
his vision of the \acro{URI} system with subtypes \acro{URL} for addresses and 
\acro{URN} for persistent names. \textcite{BernersLee1994} first describe the aim of
\acro{URI} as ``[encoding] the names and addresses of objects on the Internet''.
In the same document they broaden the scope to a more universal identifier 
system as they write: ``in order to abstract the idea of a generic object, the 
web needs the concepts of the universal set of objects, and of the universal 
set of names or addresses of objects.'' After several years and revisions the 
current version of \acro{URI} is specified together with \acro{URL} by
\textcite{BernersLee2005}. The standard defines a hierarchical namespace 
architecture with \acro{URI} schemes on the global namespace level. This
common identifier system is useful because it provides a common formal 
language that other identifier systems can be embedded into with schemes as
namespaces %(figure~\ref{fig:uriirisyntax}). 
Embedded identifier systems, 
however, do not need to define normalization rules, so equivalent identifiers
such as listed in example~\ref{ex:isbns} are quite common and impossible
to detect for general \acro{URI}s. The system neither solves the 
conflict between addressing and lookup as one purpose of an identifier
and persistent identification as another. Most \acro{URI}s are used 
primarily to retrieve documents, either directly via \acro{HTTP} or by 
embedding other \acro{URI} types in an \acro{URL}.\footnote{Originally,
\acro{HTTP} was designed to retrieve information about resources 
identified by any kind of \acro{URI}, possibly mediated via proxy 
servers. This property was partly removed, beginning with dropping 
of \acro{URI}-related header fields in the \acro{HTTP} specification 
drafts between August 3rd and 13th, 1995.} 
For this reason the identifiers actually identify a location, that may hold
different objects, but not an object, that may be available at different
locations. The example of ambiguous house numbers at page~\pageref{ex:housenum}
shows that confusing location and located object can lead to unexpected results.

Several suggestions have been made to clarify the distinction between
access and reference as independent functions of an \acro{URI}, for
instance by \textcite{Mealling2001}. Some of these developments even further 
complicated the use of \acro{URI} as global identifier system. For instance the 
\term{Resource Description Framework} (see section~\ref{sec:rdf}) claims to 
build on \acro{URI} but instead introduces its own concept `\term{URI reference}' 
that slightly differs from \term{URI}.\footnote{In particular, URI references 
may contain characters that are disallowed in an \acro{URI}. For details
see page~\ref{term:uriref}, \textcite[section 6.4]{Klyne2004},
and \textcite[section 1.2]{Hayes2004}.} Similar problems
exist with the \Tacro{Internationalized Resource Identifier}{IRI} system.
Contrary to popular belief, \acro{IRI} as defined with RFC~3987 by 
\textcite{Duerst2005} is not a superset of \acro{URI}, but a complement 
that is defined in \acro{UCS} character sequences instead of \acro{ASCII}
character sequences. The misleading statement ``every \acro{URI} is by 
definition an \acro{IRI}'' that can be found in section~3.1 of RFC~3987 
only means that if one tries to convert an \acro{URI} with the defined 
\acro{IRI}-to-\acro{URI} mapping, the original \acro{URI} is not 
modified. Instead there is a conversion algorithm that maps \acro{URI} 
to \acro{IRI}. If \acro{URI} was a subset of \acro{IRI}, no such mapping
would be needed. Both mappings use UTF-8 as intermediate format and 
\term{percent-encoding} of special characters.

% Both conversions are \term{round trip} only
% (except for potential case differences in percent-encoding and for 
% potential percent-encoded unreserved characters)!

\ignore{
\begin{figure}[h]
\begin{lstlisting}[language=BNF]
URI          = scheme ":" hierpart ( "?" query )? ( "#" fragment )?
IRI          = scheme ":" ihierpart ( "?" iquery )? ( "#" ifragment )?

scheme       = alpha ( alpha | [0-9] | "+" | "-" | "." )*
alpha        = [a-zA-Z]

URIreference = URI | relativeref
IRIreference = IRI | irelativeref
relativeref  =  relativepart ( "?"  query )? ( "#"  fragment )?
irelativeref = irelativepart ( "?" iquery )? ( "#" ifragment )?

query        = (  pchar | "/" | "?" )*
iquery       = ( ipchar | "/" | "?" | iprivate )*
fragment     = (  pchar | "/" | "?" )*
ifragment    = ( ipchar | "/" | "?" )*

pchar        =  unreserved | pctencoded | subdelims | ":" | "@"
ipchar       = iunreserved | pctencoded | subdelims | ":" | "@"
unreserved   = alpha | [0-9] | "-" | "." | "_" | "~"
iunreserved  = alpha | [0-9] | "-" | "." | "_" | "~" | ucschar
pctencoded   = "%" hexdigit hexdigit
hexdigit     = [a-zA-Z0-9]
subdelims    = "!" | "$" | "&" | "'" | "(" | ")" | 
               "*" | "+" | "," | ";" | "="

ucschar      = [#xA0-#xD7FF] | [#xF900-#xEFFFF] - noncharacter
iprivate     = [#xE000-F8FF] | [#xF0000-#x10FFFF] - noncharacter
\end{lstlisting}
\caption{Adopted subset of a formal grammar of URI and IRI}
\label{fig:uriirisyntax}
\end{figure}% adopted from RFC 2396

The non-terminal \format{noncharacter} is from the Unicode definition as given at page XXX.
}


The ongoing problems of \acro{URI} and related identifier systems have multiple
reasons. For instance the assumption of an ``universal set of objects'' leads
to paradoxes because real world objects and identifiers have no rank or category
in terms of set theory. With the \texttt{data:} \acro{URI} scheme one can even
convert any piece of data into an identifier \cite{Masinter1998}. 

Above all, many goals of an identifier system cannot be
solved on a purely technical level. The \Tacro{domain name system}{DNS} gives
examples how politics and social power structures shape identifier systems 
\cite{Rood2000}. Identifier systems can also be implemented and interpreted
differently by different users. As noted by \textcite{BernersLee1998}, people 
change identifiers, by purpose or by accident. Like all social constructs,
identifier systems can also become outdated: for instance the \acro{URI} scheme
\texttt{info:} was launched in 2003 but closed in 2010 in favor of \acro{URL}
\cite{deSompel2006,OCLC2010}. Last but no least identifier systems often try
to solve problems that cannot be solved together:
\textcite{Zooko2001} showed that an identifier system cannot provide securely 
unique, memorable (readable), and decentralized (distributed) identifiers at 
the same time but only two of these properties can be combined: local identifier
systems can generate readable and distributed identifiers but they are not globally
unique, centralized systems such as \acro{DNS} can globally unique and readable 
identifiers, and cryptographic hashes are distributed and unique but not
readable. Partial solutions to ``\term{Zooko's triangle}'' \cite{Stiegler2005} 
involve multiple layers of identifier systems, which is another example of
the importance to study relationships in combined identifier systems such as
depicted in figure~\ref{fig:idsysrel}.

\begin{table}
\begin{tabularx}{\linewidth}{|r|X|}
\hline
unambiguous   & each identifier must have only one referent \\
unique        & each referent must be associated with only one identifier \\
global        & identifiers should not require a specific context \\
persistent    & associations do not change or expire \\
readable      & identifiers are easy to read and remember \\
structured    & identifiers are described by a formal grammar \\
uniform       & identifiers are uniformly distributed \\
performant    & identifiers are easy to compute and to validate \\
descriptive   & identifiers contain information about the referent or association \\
actionable    & identifier can be used, for instance to retrieve the referent \\
distributed   & identifiers do not require a central institution \\
ordered       & identifiers have a known strict and total order \\
\hline
\end{tabularx}
\caption{Summary of designated identifier properties}
\label{tab:idprops}
\end{table}

\subsection{Descriptive identifiers}
\label{sec:descrids}

In general the association between an identifier as symbol and the thing 
it refers to is rather arbitrary unless the thing already has a natural
identifier. \Term{Descriptive identifiers} circumvent this limitation by
defining a general and independent association for all possible things, 
by which one can derive an identifier from its referent. In a broader sense
descriptive identifiers subsume so-called natural, smart, or intelligent 
keys from database and information systems and hash codes which are described
in part~\ref{sec:hashes}. In a narrower sense a descriptive identifier symbol
must reveal some information about the object it references.

% Meaningful names:
% Variablennamen in programmen als identifier - was sagen sie über
% die konzepte/objecte aus?
% \cite{Lawrie2006}

If one knows the method by which descriptive identifiers are created, an
identifier tells something about the object it refer to. For instance
one could define a descriptive identifier for bibliographic resources by
concatenating the first author's surname and the year of publication, so
one already knows this attributes by looking at the identifier. Descriptive
identifiers are easy to remember and they do not require a central registry
as identifier system. However two distinct objects may share the same 
attribute values, so they accidently get the same 
identifier. For this reason, a descriptive identifier often identifies 
something else than originally intended -- in this example the set of all
publications from a given year and a given surname. Another example is a
descriptive identifier for a houses, based on its postcode, street name, 
and house number. This decriptive identifier actually identifies one or
more addresses as locations, but not necessarily a house: some houses have 
multiple numbers, so one only identifies a part of a house, and some house numbers
refer to a complex of multiple buildings. \label{ex:housenum} 
A third example is taken from
\textcite{Eriksson2010}: the Swedish person identification number contains 
of ten digits where digit one to six represent a person's day of birth 
(\texttt{YYMMDD}) and the tenth position is a check number that can be 
calculated from the digit one to nine. This implies that one cannot derive
the full day of birth, because the century is not included (2005 and 1905 
both become \texttt{05}). To distinguish people born at the same day 
(withount century), digit seven to nine of the identifier contain a 
unique sequence number that is only partial descriptive. The ninth position
is odd if the number identifies a man and even if it identifies a woman.
To be more precice, the ninth position can only describe the assumed sex
of a person at the time when the identifier was created, because for some 
people the sex may have changed during their life. Furthermore some attributes
may be unknown: when more and more immigrants with unknown birthdate got a 
Swedish person identification number, the first of January or the first of
July was recorded instead and the identifier system ran out of numbers
\cite{Eriksson2010}. Such problems are always possible if an identifier is
not based on inherent properties but on properties that are attributed to an
object.\footnote{The difference between attributed and inherent is not obvious.
For instance most people would see gender as given while others as purely 
artificial \cite{Butler1990}. However most identifier conflicts origin from
different interpretations what kind of object the referent actually is.}
To summarize, descriptive identifiers are 
problematic, if the attributes that they base on are not unique, not always
known, or subject to changes. This reasons are arguments for ``meaningless
identifiers'' or ``surrogate keys'' as proposed \textcite{Kimball1998}
and \textcite{Wieringa1991} among others.
% see also \textcite{Aleksic2010} for performance comparison

\ignore{
\subsection*{UUID}
% Example: UUID
\Tacro{Universally Unique Identifier}{UUID} show some difficulties in 
implementing descriptive or non-descriptive identifiers. \acro{UUID} was
designed to identify arbitrary objects in distributed computer systems.
To not assign the same identifiers 

RDF: \cite{Leach2005} as URN namespace

... is partly descriptive as it contains information about the computer
and time of the association

\ldots
Neither purely descriptive nor hash, but related to both:
on content and random UUIDs that also ensure uniqueness based on probability.
 \cite{Gladney1998} argues for UUID 
}

\subsection{Ordered identifiers}
\label{sec:orderedids}

The possibility to arrange identifier symbols in a meaningful way is seldom
cited as important to digital identifiers, allthough the basic form of digital
data is an ordered sequence of bits. 
\Term[ordered identifier]{Ordered identifiers} can be defined as any 
identifiers that have a \term[strict order]{strict} and \term{total order}. 
Simple examples in data include memory addresses and line numbers. Ordered 
identifiers have several usefull properties: First, one can sort objects
by their  identifiers, so every set of objects with distinct identifiers has
a normalized form, and second, one can specify ranges of identifiers. Sorted
ranges further allow efficient searching based on binary search.
The range \texttt{CA} to \texttt{CI}, for instance, specifies the range of all 
notations from the Regensburg Classification Scheme RVK.\footnote{See 
\url{http://rvk.uni-regensburg.de/} for more information about the classification.}
In practice however, collation is often not simply determined by the order of 
characters that the identifier is build from. For instance the identifier 
\texttt{9X} could be sorted after the identifier \texttt{10Y} if the first 
character is given most importance ($9 > 1$), but it could also be reverse if the 
identifiers are interpreted as starting with numbers ($9 < 10$). The more
structured identifiers are, the more complex it can be to compare them.
Sorting rules for ordering personal names, for instance, depend on language
and culture and on the ability to break a name into given name, surname, and 
other parts.

Ordered identifiers are easy to implement if there is a finite number of
items or if new items are added sequentially. For instance in the 
\Term{numerus currens} system of library shelving books get signatures 
(and locations) in order of their acquisition. Another examples are
bates numbers that are used to assign consecutive numbers with a stamp.
The order implies a mapping from identifier symbols to the natural numbers
$\mathbb{N}=\set{1,2,3,\ldots}$,\footnote{or $\mathbb{N}=\set{0,1,2,\ldots}$ 
depending on personal preferrence.} or to a subset of $\mathbb{N}$. In many
applications the natural numbers are directly used as identifier symbols 
without any mapping function in between. However, for many ordered identifiers 
no specific mapping to $\mathbb{N}$ (or to a subset 
$\mathbb{N}_m=\set[x\in\mathbb{N}]{x \le m}$) is known. A meaningful
mapping may also be \term{injective} instead of \term{bijective} with gaps 
of numbers that no identifier is mapped to. A known bijective mapping without 
gaps is useful because it adds some properties to ordered identifiers: first,
the last identifier gives the total number if objects, and second, one 
can always tell the number of objects in a given range of identifiers. The 
latter is usefull especially because it allows to calculate with 
identifiers like coordinates. 

\ignore{ 
\footnote{These recommendation include:
A Vision of an Integrated Internet Information Service (RFC~1727),
Using the Z39.50 Information Retrieval Protocol in the Internet Environment (RFC~1729)
Functional Recommendations for Internet Resource Locators (RFC~1736),
Functional Requirements for Uniform Resource Names (RFC~1737),
and Universal Resource Locators (RFC~1738), all published in december 1994.
}
}

\subsection{Hash codes}
\label{sec:hashes}

A \Term{hash function} is a computable function that maps arbitray large 
sequences of bits into smaller bit sequences of fixed length
(figure~\ref{fig:hashfunction}). The output of
a hash function is called \Term{hash code}, hash key, digest, or just hash 
and the input is also called message especially for cryptographic 
hash functions. A good hash funtion should be easy to compute\footnote{
Some hash algorithms allow the hash of a composite object to be computed 
from the hashes of its constituent parts. For instance the Rabin fingerprint
$f$ of a concatenation $A.B$ of two strings $A$ and $B$ can be computed via 
the equality $f(A.B)=f(f(A).B))$ \cite{Broder1993}.} and it should map typical
input values to uniformly distributed hash codes, so every code is 
generated with the same probability. Distinct input values that are mapped to
the same hash code are called a \Term{collision}. Depending on properties of a 
hash function and its application, collisions either imply equivalent input
values or they are so unlikely that in practice hash codes are virtually 
unique. Thus, hash codes can be used as compact and distributed identifiers,
either of equivalent or of unique digital objects. The main applications of 
hash functions are storage, duplicate detection, and cryptography. Hash 
functions for storage in hash tables or data caches utilize the uniform 
distribution of hash codes but they may allow some collisions. This makes 
them rather identifiers of addresses computed from data objects than 
identifiers of data objects. Hash functions for duplicate detection neither
directly identify digital objects but sets of objects that are assumed to be
equivalent based on their content. In contrast to hash functions for the other
two types of applications, hash functions for duplicate detection highly
depend on the type of input values as they only take into account a 
significant part of the input. For instance the bibliographic hashkey for
bibliographic records in the social cataloging platform BibSonomy only uses
specific parts of the fields of author, editor, title, and year 
\cite{Voss2009}. The quality of a hash function for duplicate detection
depends on the ability to define which object properties count as significant
and when two objects should be treated as equal - a problem that is far
from trivial \cite{Renear2009,Yeo2010}. If the function is not choosen
good enough, it can better be described as heuristic or classifier with
error rate of false positives and false negatives instead of a kind of
identifier.

\begin{figure}[h]
\centering
\begin{tikzpicture}[orm]%line width=0.25mm]
\draw[->] (-1.9,0.15) to (1.9,0.15);
\node[anchor=south] at (0,0.15) {hash function};
\node[anchor=north] at (0,-0.15) {impractical (one-way)};
\draw[<-,dotted] (-1.9,-0.15) to (1.9,-0.15);
\entity[anchor=east,minimum width=26mm] (r) at (-2,0) {referent};
\entity[anchor=west,minimum width=22mm] (i) at (2,0) {identifier};
\draw[|-|,dotted] (-4.6,0.6) to node[pos=0.5,fill=white] {variable width} (-2,0.6);
\draw[|-|] (4.2,0.6) to node[pos=0.5,fill=white] {fixed width} (2,0.6);
\end{tikzpicture}
\caption{(Cryptographic) hash function}
\label{fig:hashfunction}
\end{figure}

Cryptographic hash functions treat the whole input as significant part: any 
change of an input value must result in a different hash code, so attackers 
cannot modify messages without modifying the message digest. In addition the
function should fullfill the following properties: First, the hash code should
not reveal more information about the input that its own expression.%
\footnote{In a broader sense, all hash
functions are descriptive because their hash codes are defined based on the
full digital content as property of the identified object. In a narrower sense
cryptographic hash keys are not descriptive because they only describe the 
hash code as property of the object's content.} Second, the hash function must
be a \Term{one-way function}: given a hash code it must be very hard to find 
a message that is mapped to this digest. This property is also needed to
prevent creating a message as collision of another given message. Third, an 
even more strict requirement used to evaluate the strength of cryptographic
hash functions is the lack of a method to find any collisions. Given that the
number of possible input values is much large than the number of possible
hash codes, there always exist collisions, but it is very difficult to find
them. For instance the cryptographic hash function SHA1 \cite{Eastlake2001}
has codes of 160 bit length, so there are $2^{160}$ different SHA1 hash 
codes. According to rules of probability the expected number of hashes that 
can be generated before an accidental collision (``birthday paradox'') is
$2^{80}$. The sun will expand in around 5 billion years (less than $2^{58}$
seconds from now), making life on earth impossible. Until then one can
generate $2^{22}$ (4 million) hashes per second and collisions are still
unlikely. With systematic crypographic attacks the number can be 
smaller but it is still much larger than other sources of error.

% TODO: where are cryptographic identifiers used?
% Examples! MD5, CRC, SHA1...
% Example: use of SHA-checksums as object identifiers in git distributed
% revision control system.

