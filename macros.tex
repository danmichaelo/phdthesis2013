\usepackage[utf8]{inputenc}
\usepackage[ngerman,english]{babel}
\usepackage{appendix}

%%%%%%%%%%%%%%%%%%%%%%%%%%%%%%%%%%%%%%%%%%%%%%%%%%%%%%%%%%%%
%% Index

\usepackage[protected]{splitidx}% makeindex will brake layout when sindex in table cells
%\usepackage[makeindex,protected]{splitidx}
\newindex[General Index]{idx}
\newindex[Name index]{person}
\newindex[List of acronyms]{acronym}

\setindexpreamble[idx]{
Page numbers set in bold font indicate a more detailed description. 
Persons and acronyms are excluded and included in an index of its
own respectively.}% TODO reference the page number of other indexes
\setindexpreamble[acronym]{The world of data is full of acronyms.
Bold page numbers indicate a further description of the entry, 
normal page numbers indicate where an acronym is used in the text.}

\usepackage{multicol}
\extendtheindex{}%
{\useindexpreamble
\small\begin{multicols}{2}}%
{\end{multicols}}% after the index
{}% no change after the ending

% required hack to mix hyperref and formatted page numbers
% \newcommand{\bfhref}[1]{\textbf{\hyperpage{#1}}}

\setindexpreamble[person]{
This index lists all persons that are cited by their works (page numbers in
normal font) or directly referenced by name (page numbers in \textbf{bold}) font.
}

% style of the page numbers in person index
\newcommand*{\personindexcitepage}[1]{\hyperpage{#1}}
\newcommand*{\personindextextpage}[1]{\textbf{\hyperpage{#1}}}

% command to create a person index entry
\newcommand{\personindex}[1]{\sindex[person]{#1}}

% command to create a specific type of person indext entry (text or cite)
\newcommand{\personindexcite}[1]{\personindex{#1|personindexcitepage}}
\newcommand{\personindextext}[1]{\personindex{#1|personindextextpage}}

\newcommand{\Person}[2][]{% print and index full name
\ifthenelse{\isempty{#1}}%
{\personindextext{#2}#2}%
{\personindextext{#2, #1}#1 #2}}

\newcommand{\person}[2][]{% print surname only but index full name
\ifthenelse{\isempty{#1}}%
{\personindextext{#2}#2}%
{\personindextext{#2, #1}#2}}

% See http://www.anindexer.com/about/abbr/abbrindex.html for style tips

%%% I want multiple indexes
% \usepackage[abshang,hangindent=1em,font=small]{idxlayout}

% Declared in biblatex: maybe change to suppress editors?
% \DeclareIndexNameAlias{author}{default}
% \DeclareIndexNameAlias{editor}{default}
% \DeclareIndexNameAlias{bookauthor}{default}

\newcommand{\unnumberedsection}[1]{
  \clearpage
  \phantomsection
  \section*{#1}
  \addcontentsline{toc}{section}{#1}
}

%%%%%%%%%%%%%%%%%%%%%%%%%%%%%%%%%%%%%%%%%%%%%%%%%%%%%%%%%%%%

\usepackage[automark]{scrpage2}
\setheadsepline{0pt}% no vertical line below header
\lehead{\upshape\headmark}% chapter
%\lehead{\small\upshape\thechapter~\chaptermark}% chapter
\rohead{\small\upshape\headmark}% section
\pagestyle{scrheadings}
\automark[section]{chapter}
%%%%

% thanks to http://mintaka.sdsu.edu/GF/bibliog/latex/floats.html
\renewcommand{\topfraction}{0.9}	% max fraction of floats at top
\renewcommand{\bottomfraction}{0.8}	% max fraction of floats at bottom

%   Parameters for TEXT pages (not float pages):
\setcounter{topnumber}{2}
\setcounter{bottomnumber}{2}
\setcounter{totalnumber}{4}     % 2 may work better
\setcounter{dbltopnumber}{2}    % for 2-column pages
\renewcommand{\dbltopfraction}{0.9}	% fit big float above 2-col. text
\renewcommand{\textfraction}{0.07}	% allow minimal text w. figs
				    %   Parameters for FLOAT pages (not text pages):
\renewcommand{\floatpagefraction}{0.7}	% require fuller float pages
% N.B.: floatpagefraction MUST be less than topfraction !!
\renewcommand{\dblfloatpagefraction}{0.7}% adjust style of chapter and section headers

\setkomafont{disposition}{\normalcolor\rmfamily\bfseries}
\addtokomafont{section}{\LARGE}
\addtokomafont{subsection}{\large}  

% chapter title
\addtokomafont{chapter}{\Huge}
\renewcommand*{\chapterformat}{\LARGE\chaptername~\thechapter}

% chapter mark in the header
\renewcommand*{\chaptermarkformat}{\thechapter\enskip}

%% Use serife from kpfonts as standard font
\usepackage[light,onlyrm,largesmallcaps]{kpfonts} % serife

% defines \textgreater and \textless
\usepackage{textcomp}% only required in one image (???)

% Zeilenabstand
% \linespread{1.1}
\usepackage{microtype}
\SetExpansion[context=sloppy,shrink=60]{encoding={OT1,T1,TS1}}{}

\usepackage{rotating} % for sidewaystable
\usepackage{tipa,textcomp} % needed for \textbeta in combination with hyperref

\usepackage{afterpage}

%%% extended tables
\usepackage{tabularx}
\usepackage{multirow}
\usepackage{upquote}

\usepackage{fancybox}
\usepackage{float}
\makeatletter
\newcommand{\fs@shadow}{%
  \fs@plain
  \def\@fs@pre{%
    \setbox\@currbox\vbox{\shadowbox{\box\@currbox}}%
  }
  \let\@fs@iftopcapt=\iffalse}
\makeatother

\newcommand{\subsectionexample}[1]{
  \stepcounter{example}
  \subsection*{Example \theexample: #1}
  \addcontentsline{example}{example}{\protect\numberline{\theexample}#1}
}

\let\tq\textquotesingle
\let\sp\textvisiblespace

%\usepackage{mathabx} % THANKS to http://carlo-hamalainen.net/blog/?p=8
% \usepackage{MnSymbol}
%\usepackage{tabularx}
\usepackage{graphicx}

%%% Add a space after footnote marks (only problematic if referenced)
\let\myfootnote\footnote
\renewcommand{\footnote}[1]{\myfootnote{~#1}}

% Zeilenabstand vorübergehend ändern
%%\usepackage{setspace}

%%% tikz and tkz-orm
%\usepackage{tikz-er2} % TODO: say thanks
\usepackage{tkz-orm}
\usetikzlibrary{positioning}
\usetikzlibrary{arrows}
\usetikzlibrary{decorations.pathreplacing}
\usetikzlibrary{matrix}
\usetikzlibrary{shapes} % for shape ellipse
\usetikzlibrary{chains}
\usetikzlibrary{calc}
\usetikzlibrary{mindmap,trees} 
\usetikzlibrary{snakes}

% some tikz styles for data matrices
\tikzset{ucs/.style={font=\ttfamily\scriptsize,color=black!70,align=center}}
\tikzset{datamatrix/.style={matrix of nodes,nodes in empty cells,
                            nodes={anchor=west,font=\ttfamily}}}

% some tikz styles for drawing graphs
\tikzstyle{gnode}=[circle,fill=black,minimum size=2mm,inner sep=0pt]
\tikzstyle{garc}=[solid,thick,->,shorten >=0.75pt,>=latex]
\tikzstyle{gedge}=[thick]
\tikzstyle{glink}=[thick,dashed,->,shorten >=0.75pt,>=latex]
\tikzstyle{ggraph}=[every node/.style={gnode},every edge/.style={garc}]

% some tikz styles for drawing ERM diagrams
\tikzstyle{erm}=[font=\sffamily,node distance=4mm,
          label distance=1.25mm,line width=0.25mm]
\tikzstyle{ermnode}=[draw,minimum size=6mm]

\newcommand{\ignore}[1]{}

%%% Captions of figures, tables, etc.
\usepackage[hang,small,bf]{caption}
%\renewcommand{\captionfont}{\rmfamily} 
%\renewcommand{\captionlabelfont}{\rmfamily\textbf}


%% additional mathematical symbols, align etc.
\usepackage{amsmath,amsthm,amssymb}

% relation
\newcommand{\rel}{\longrightarrow}
\newcommand{\bij}{\leftrightarrow}

\usepackage[normalem]{ulem} % underline, strikeout etc.
%\newcommand{\pattern}[1]{\dashuline{#1}} % or dotuline
%\normalem % use standard layout for emph instead of underline

%\newcommand{\pattern}[1]{\emph{#1}}
\newcommand{\pattern}[1]{\hyperref[pattern:#1]{\emph{#1}}}

\usepackage{acroterm}
\usepackage{relsize}
%\renewcommand{\acrostyle}[1]{\textsc{\smaller #1}}
%\renewcommand{\Acrostyle}[1]{\textsc{#1}}
\renewcommand{\acrostyle}[1]{#1}
\renewcommand{\Acrostyle}[1]{\textit{#1}}


%%%%%%%%%%%%%%%%%%%%%%%%%%%%%%%%%%%%%%%%%%%%%%%%%%%%%%%%%%%%
% Syntax highlighting
\PassOptionsToPackage{x11names,rgb}{xcolor}
\usepackage[procnames]{listings}

\definecolor{mygray}{rgb}{0.4,0.4,0.4}
\definecolor{mydarkblue}{rgb}{0.0,0.0,0.4}
\definecolor{mycyan}{rgb}{0.0,0.4,0.4}
\definecolor{mygreen}{rgb}{0.0,0.5,0.0}

\lstset{
  basicstyle=\ttfamily,
  showstringspaces=false,
  commentstyle=\color{mygray}\upshape,
  stringstyle=\color{black},
  numberstyle=\color{black}\footnotesize,
  keywordstyle=\color{mycyan},
  identifierstyle=\color{mydarkblue}
}

\lstdefinelanguage{XML}
{
  columns=fullflexible,
  alsoletter={-.:},
  morecomment=[s]{!--}{--},
  morestring=[b]",
  morestring=[s]{>}{<},
  morecomment=[s]{<?}{?>},
}
\lstdefinelanguage{DTD}{
  alsoletter={-\#.:},
  keywords={,CDATA,DOCTYPE,ATTLIST,ELEMENT,EMPTY,ANY,ID,%
      IDREF,IDREFS,ENTITY,ENTITIES,NMTOKEN,NMTOKENS,NOTATION,%
      INCLUDE,IGNORE,SYSTEM,PUBLIC,NDATA,PUBLIC,%
      \#PCDATA,\#REQUIRED,\#IMPLIED,\#FIXED,}%
  morecomment=[s]{<!--}{-->},
  morestring=[d]",
  morestring=[d]',
  stringstyle=\color{mygreen},
}
\lstdefinelanguage{rnc}{
  keywords={,attribute,default,datatypes,div,element,empty,external,grammar,%
    include,inherit,list,mixed,namespace,notAllowed,parent,start,string,%
	text,token,},
  alsoletter={-.:},
  morecomment=[l]{\#},
  columns=flexible,
}
\lstdefinelanguage{BNF}
{
  columns=flexible,
  morestring=[d]",
  morestring=[d]',
  morestring=[s]{\[}{\]},
  morestring=[s]{\#}{\ },
%  stringstyle=\itshape\color[rgb]{0.0,0.5,0.0},
  stringstyle=\color{mygreen},
%  stringstyle=\color[rgb]{0.0,0.5,0.5},
  identifierstyle=\sffamily\color{mydarkblue} %darkblue? 
}
\lstdefinelanguage{turtle}
{
  columns=fullflexible,
  morekeywords={@prefix,@base,@forSome,@forAll,@keywords},
  morecomment=[l]{\#},
  alsoletter={-?}, % allowed in names
%  morecomment=[l][\color{cyan}]{@}, % TODO: language tags
  morecomment=[s][\color{mydarkblue}]{<}{>},
%  basicstyle=\ttfamily\color{gray}, % . , ;
%  numberstyle=\color{darkblue},
  morestring=[b][\color{black}]\",
}
\newcommand\rdf[1]{\lstinline[language=turtle]{#1}}
\newcommand\xml[1]{\lstinline[language=XML]{#1}}
\newcommand\dtd[1]{\lstinline[language=DTD]{#1}}
\newcommand\bnf[1]{\lstinline[language=BNF]{#1}}
\newcommand\sql[1]{\lstinline[language=SQL,morekeywords={SCHEMA}]{#1}}

\newcommand\bnfstring[1]{\texttt{\color{mygreen}#1}}

%%%%%%%%%%%%%%%%%%%%%%%%%%%%%%%%%%%%%%%%%%%%%%%%%%%%%%%%%%%%
%\usepackage{mdframed}
%\mdfdefinestyle{exampleStyle}{topline=false,bottomline=false,leftline=true,rightline=true}
%
%\newcommand\fs@myExampleStyle{\def\@fs@cfont{\bfseries}\let\@fs@capt\floatc@plain
%\def\@fs@pre{\begin{mdframed}[style=examleStyle]}%
%\def\@fs@mid{}%
%\def\@fs@post{\end{mdframed}}\let\@fs@iftopcapt\iffalse}

%\floatstyle{myExampleStyle}
%\floatstyle{border}
\newfloat{example}{htp}{example}
\floatname{example}{Example}
%\floatstyle{plain}

%%%%%%%%%%%%%%%%%%%%%%%%%%%%%%%%%%%%%%%%%%%%%%%%%%%%%%%%%%%%
%%% Bibliography and citation style
\usepackage[
  style=authoryear-comp,
  sorting=nyvt,hyperref=true,
  indexing=cite,
  url=true,
  isbn=false,
  doi=false, 
  backend=biber
]{biblatex}

%\DeclareFieldFormat*{url}{}
\DeclareFieldFormat*{url}{\url{#1}}
\DeclareFieldFormat[book]{url}{}
%\DeclareFieldFormat[misc]{url}{\url{#1}}
%\DeclareFieldFormat[standard]{url}{\url{#1}}
%\DeclareFieldFormat[report]{url}{\url{#1}}
%\DeclareFieldFormat[misc]{url}{\url{#1}}
%\DeclareFieldFormat[misc]{article}{\url{#1}}
\AtEveryBibitem{% remove url if pages field
	\iffieldundef{pages}{%
	}{%
		\clearfield{url}
	}%
}

% show note in parens
\DeclareFieldFormat[misc]{note}{\mkbibparens{#1}}


% let cite behave like \parencite
\let\cite\parencite

\DeclareCiteCommand{\citeyear} 
  {\boolfalse{citetracker}% 
   \boolfalse{pagetracker}% 
   \usebibmacro{prenote}} 
  {\printtext[bibhyperref]{\printfield{year}}} 
  {\multicitedelim} 
  {\usebibmacro{postnote}} 

\setlength{\bibitemsep}{1pt}% TODO


\usepackage{xifthen}

\DeclareIndexNameFormat{default}{%
  \usebibmacro{index:name}{\personindexcite}{#1}{#3}{#5}{#7}}

\DeclareIndexFieldFormat{indextitle}{}% do not index titles


%%%%%%%%%%%%%%%%%%%%%%%%%%%%%%%%%%%%%%%%%%%%%%%%%%%%%%%%%%%%
% Size of Quotation / Quote / Verse
%\let\oldquote\quote
%\renewcommand\quote{\par\singlespacing\small\oldquote}
%\let\oldquotation\quotation
%\renewcommand\quotation{\par\singlespacing\small\oldquotation}
%\let\oldverse\verse
%\renewcommand\verse{\par\singlespacing\small\oldverse}

% TODO calc \textwidth - \parindent
\renewenvironment{quotation}%
{\begin{center}\begin{minipage}[h]{0.95\textwidth}\small\noindent}%
{\end{minipage}\end{center}\noindent}

\newcommand{\quotationsource}{\hspace*{0.5em} --- }

% chapter numbers roman
%\def\thechapter{\Roman{chapter}}

%%% enable '|' to type verbatim
\usepackage{fancyvrb}% TODO: really needed?

%%% Unicode codepoint
\newcommand{\U}[1]{\texttt{U+#1}}
\newcommand{\Ox}[1]{\texttt{0x#1}}

\newcommand\format[1]{\textsf{#1}}
\newcommand{\formati}[2]{\ensuremath{\format{#1}_{\textsf{#2}}}}

\newcommand{\tuple}[1]{\langle #1 \rangle}

% Set brackets and set builder notation
\newcommand{\set}[2][]{%
  \ifthenelse{\isempty{#1}}%
    {\ensuremath{\{#2\}}}
    {\ensuremath{\{\,#1\,|\,#2\,\}}}
}

%%%%%%%%%%%%%%%%%%%%%%%%%%%%%%%%%%%%%%%%%%%%%%%%%%%%%%%%%%%%

\newcommand{\nwc}[1]{\ensuremath{#1^\uparrow}}
\newcommand{\nwr}[1]{\ensuremath{#1^\downarrow}}

%%%%%%%%%%%%%%%%%%%%%%%%%%%%%%%%%%%%%%%%%%%%%%%%%%%%%%%%%%%%

%\usepackage{enumerate}
\usepackage{enumitem}
\setlist{leftmargin=*,rightmargin=\parindent,font=\rmfamily}
\setlist[description]{labelsep=0.5em,leftmargin=0.5em}
%TODO: this make build brake:?!
%\setlist[description]{labelindent=\parindent,label=--,topsep=2pt,itemsep=2pt,parsep=0pt}

%%%%%%%%%%%%%%%%%%%%%%%%%%%%%%%%%%%%%%%%%%%%%%%%%%%%%%%%%%%%
%\newcommand{\lowercasepatternlabel}[1]{pattern:\lowercase{#1}}
%\newcommand{\patternlabel}[1]{%
%    \expandafter\label\lowercasepatternlabel\expandafter{#1}%
%}
\newcommand{\patternsection}[2]{%

  % horizontal line
  \vskip\medskipamount % or other desired dimension
  \leaders\vrule width \textwidth\vskip0.4pt % or other desired thickness
  \vskip\medskipamount % ditto
  \nointerlineskip

  \expandafter\subsection\expandafter{#1 pattern}
  \expandafter\label\expandafter{#2}
}

%%%%%%%%%%%%%%%%%%%%%%%%%%%%%%%%%%%%%%%%%%%%%%%%%%%%%%%%%%%%

\newcommand{\TODO}[1]{\vspace{\parskip}\noindent%
\begin{tikzpicture}
\node (b) [font=\scriptsize,rectangle,
  inner sep=5pt,
  text width=\linewidth,
  %text justified,
  anchor=south
]{\parbox[t]{\linewidth}{#1}};
\draw[dashed,very thick,red] (b.north west) to (b.south west)
to +(10pt,0);
\draw[dashed,very thick,red] (b.north west) to +(10pt,0);
\end{tikzpicture}}

%%%%%%%%%%%%%%%%%%%%%%%%%%%%%%%%%%%%%%%%%%%%%%%%%%%%%%%%%%%%

\usepackage{hyperxmp}
\usepackage[
  pdfborder={0 0 0},
  %colorlinks=false, % linkcolor=purple, citecolor=olive, urlcolor=violet,
  linkcolor=purple, citecolor=olive, urlcolor=violet,
  bookmarksnumbered=true
]{hyperref}

\hypersetup{
    pdfauthor={Jakob Voß},
    pdftitle={Describing data patterns. A general deconstruction of metadata standards},
    pdfsubject={Common features across all methods of data structuring and description},
    pdfkeywords={data, data description, data modeling, metadata, pattern language},
    pdfmetalang={en},
    pdfcopyright={CC-BY-SA by Jakob Voß},
    pdflicenseurl={http://creativecommons.org/licenses/by-na/3.0/},
}

%%%%%%%%%%%%%%%%%%%%%%%%%%%%%%%%%%%%%%%%%%%%%%%%%%%%%%%%%%%%

\bibliography{bibliography}

\raggedbottom % prevent LaTeX from adding vertical white space in strange places on pages that it cannot fill properly.

% numerate subsubsections in small roman numbers but don't shown them in ToC
\setcounter{tocdepth}{2}
\setcounter{secnumdepth}{3}
\renewcommand{\thesubsubsection}{\Roman{subsubsection}}

%\usepackage{tocstyle}
%\usetocstyle{allwithdot} 
\renewcommand{\contentsname}{Table of Contents}
%%\renewcommand\cfttoctitlefont{\Large\hfill\MakeUppercase}
%%\renewcommand{\cftaftertoctitle}{\hfill}
%\renewcommand{cftloftitlefont


