\section{File systems}
\label{sec:filesystems}

A file is the most basic method to store and manage digital data. Other methods
are build on files by one means or another.  A \Term{file system} is a method
of storing, organizing and retrieving files on all sorts of storage devices,
such as a hard disk, flash drive, and magnetic tape. Provided as core part of
the operating system, the file system abstracts from underlying storage media.
Thus application can work on files without having to bother where and how their
content is physically stored. Operating systems may also provide file systems
as interface to read and write from and to devices and programs.\footnote{This
design principle is brought to a head in \term{Plan 9}: this successor to the
\term{Unix} operating system represents every object as file.} In the following
we will limit a \Term{file} to an object that holds a (possibly empty) finite
stream of sequent bytes. Issues of performance and security -- the main driving
force behind file system development -- will be ignored as well as any relation
to physical storage media. General introductions to file systems are given by
\textcite{Tanenbaum2008}, \textcite{Reimer2008}, and \textcite{Giampaolo1999}.

\subsection{Origins and evolution}

Despite all variety, basic properties of file systems have not changed much
since their introduction in the early 1960s. Compared to other trends in
computing the evolution of file systems is very slow, because they are
deep-rooted in operating systems and bound to properties of storage media. The
basic layout of todays file systems evolved parallel to the change from storage
devices with sequential access (piles of \term{punched cards} or magnetic
tapes) to \term{disks} that allowed \term{random access}. The next shift may
take place today with techniques like cloud computing and solid state drives
(SSD), that blur the borders between local and external storage, and between
random-access main memory and hard disk drives (see section~\ref{sec:nosql}).

In most early operating systems such as \term{Multics}~(1969),
\term{CP/M}~(1975), and \term{Apple DOS}~(1978) there was only one `native'
file system which could not be separated from the operating system.  Later
systems such as \term{SunOS~2.0}~(1985), \term{System~V}~(Release~3 in 1986)
and \term{Linux} (in~1992) introduced a \term{virtual file system} as
abstraction layer to access multiple file systems in a uniform way. However a
clear separation between operating system and file system is still difficult
because the operating system may impose additional restrictions on files.

To overcome incompatibilities between different \term{Unix} dialects, the
\tacro{Portable Operating System Interface}{POSIX} was standardized in 1988
\cite{POSIX1988}. \acro{POSIX} defines a common set of utilities and
programming interfaces, among them an API to access different file systems in a
consistent way. Most modern file systems conform to \acro{POSIX}. This is good
for interoperability but limits the conceptual evolution of file systems to a
least common denominator. \textcite{Nelson1965}, in his first influental talk
about ``\term{hypertext}'',\footnote{Actually, he had already given a talk
about hypertext at the conference of the \tacro{International Federation for
Information and Documentation}{FID}, in the same year \cite{Nelson1965a}.  The
abstract is reprinted in \textcite[p. 154]{Nelson2010}. Nelson later revised
his proposed design of the ``evolutionary list files'' to the \term{Xanadu}
project and the \term{Zzstructure}.}  asserted that ``there are probably
various possible file structures that will be useful in aiding creative
thought''.  Regularly he complained about ``The tyranny of the file'' and
hierarchical directory structures \cite{Nelson1986}.\footnote{Everyone who ever
tried to use a shared network folder in a cooperation should know that
monohierarchies just do not work.  However file system characteristics are so
deep-rooted in (our perception of) computer systems that we can hardly imagine
alternatives. See the video `Ted Nelson on Software' at
\url{http://www.youtube.com/watch?v=zumdnI4EG14} for a short overview (8
minutes) of his critique on traditional file systems and their impact.} His
article was referenced in the specification of Multics file system
\cite{Daley1965} but never really picked up among file system developers, so
today \acro{POSIX} remains the standard way of thinking about files.


\subsection{Components and properties}

Operating systems provide methods to access a file system by `mounting' it from
a specific disk or other location. Files can then be accessed independent of
their location by APIs such as \acro{POSIX}. \Term[virtual file system]{Virtual
file systems} combine and wrap multiple file systems into one.  In general we
can call every mountable storage system a file system.  This also includes
\term{revision control system}s, \term{HTTP} (especially with \term{WebDAV}),
and archive files (see section~\ref{sec:archivfiles}).  The following analysis
of general file system components and properties abstracts from different
access methods. Therefore file name prefixes such as protocol, host name,
device, disc, volume, port etc.  are not included --- these namespace
identifiers should not be treated as part of a file system and its file names,
but as part of an \acro{API} for file access.  The following description is
organized chronologically as most properties are based on historic trends that
date back to the origins of computer systems.

\subsubsection{Files and file names}

The first commercial disk drive, the \term{IBM~350} was announced in 1956 and
stored 5~million 6-bit-characters (4~megabytes). The drives had the size of a
wardrobe and were also called ``files'', leading to the modern usage of the
term. The file concept gained importance with time-sharing operating systems
that allowed user to directly interact with a computer. Before this data was
primarily exchanged in form of physical storage media. In the early 1960s the
\tacro{Compatible Time-Sharing System}{CTSS} introduced the concept of
\term{user files}:

\begin{quotation}%
These are files of information which a user wishes to store away for future
reference. They may consist of programs, data for programs, or any other
information the user desires. They are kept on the disk indefinitely and allow a
user to retrieve a program several weeks after he wrote it. Thus, the disk
replaces the decks of cards and reels of magnetic tape usually associated with a
large computer installation.\\
\quotationsource \textcite[p. 3]{Saltzer1965}
\end{quotation}\label{quot:files}

File systems normally do not restrict the content of files, even files of zero
byte length are allowed. The maximum file size, the maximum number of files,
and the maximum sum of file sizes may be limited in a given context.  To
uniquely identity files, each file should have a \Term{file name}.  This simple
assumption is complicated by operating systems and file systems which impose
different restrictions on file names. In its most general form a file name is a
non-empty sequence of bytes. All systems exclude at least the NUL byte
\texttt{0x00}. File systems also limit the length of a file name to a maximum
number of bytes and/or Unicode characters. Modern file systems (\term{NTFS},
\term{HFS+}, \term{ZFS} etc.) only accept valid Unicode characters, so file
names are (possibly normalized) Unicode strings. Nevertheless there is a strong
tradition in the \term{Unix} community to view file names sequences of as raw
bytes. Some file systems are \term{case sensitive} (one can have two distinct
file names \verb|A| and \verb|a|), some are case insensitive, and some are case
insensitive but case preserving (\verb|A| and \verb|a| refer to the same file
but its name can be named either of them).  Moreover each system disallows some
special characters or bytes, for instance the directory separator \verb|/| or
\verb|\|. Other special characters include quotes (\verb|"| and \verb|'|),
brackets (\verb|<|, \verb|>|, \verb|[|, \verb|]|), dot (\verb|.|), colon
(\verb|:|), vertical bar (\verb#|#), asterisk (\verb|*|), and question mark
(\verb|?|).

\subsubsection{Extensions and types}

In the first \term{CTSS} system, files were composed of two parts, each with up
to six characters. The first part was used as descriptive name and the second
indicated the file's type. Many operating systems followed this convention and
supported or required \Term[file extension]{file extensions}. However the
extension may not reflect a well-defined type or a file may not have an
extension at all. Therefore many programs treat the extension as one of
multiple indicators to determine \term{file type}. Depending on the context
file(name) extensions are part of the the file name or additional metadata of
the file.

\subsubsection{Versions}

% TODO: HammerFS, LogFS, NILFS, BtrFS ...

\Term[versions!in file systems]{Versioning} files is not common in todays file
systems although it was already supported in early time sharing system systems
such as \term{ITS} \cite{Eastlake1972} and \term{TENEX} \cite{Bobrow1972}. The
ITS system simply treated numeric file extensions as \term{version numbers}.
The user could select to read a file with the highest version number of a given
name or to increment the highest version number and create a new version for
writing. In TENEX and its successors (\term{TOPS-20} and \term{OpenVMS}) the
version number is an additional part of the file name. Today versioning is
rarely implemented in the file system but on top of it in applications, for
instance in \term{revision control system}s (\term{Subversion}, \term{git},
\term{mercurial}, etc.), or in storage services that can be accesed like a file
system such as \term{Amazon~S3} \cite{S3DevGuide}. Other file system features
like cloning or snapshots as provided in \term{ZFS} and \term{NTFS} can emulate
versioning to some extend.  Just like file extensions version numbers can be
considered as part of the file name or as special kind of additional metadata.
However version numbers allow multiple versions to share the same file name,
and they define a strict order between all versions of a files (or a partial
order in revision control systems that allow branches and merges). The version
number itself does not have to be a simple integer but can also be a timestamp.

\subsubsection{Directories and hierarchies}

\Term[directory]{Directories} are a common method to group files. In
\acro{CTSS} each user had a private directory and there were common
\term{directories} for sharing. A directory acts like a \term{namespace} for
file names (see section~\ref{sec:qualifiers}): different files in different
directories can share the same name. From a conceptual point of view there is
little difference between simple directories and other file system prefixes
such as volume or disk: both hold simply a set of file names.  In a broader
context each directory must have a unique name just like a file.  Hierarchical
file system as introduced with \term{Multics} typically apply the same rules to
file names and directory names while flat file systems such as \term{Apple DOS}
and \term{Amazon~S3} have different rules for directory names and file names.
In addition the maximum number of files per directory can be limited.

In the mid-1960s \Term[hierarchy!in file systems]{hierarchies} were introduced
in the \term{Multics} project that led to the development of \term{Unix} after
\term{Bell Labs} pulled out of Multics in 1969 \cite{Ritchie1979}. In Multics a
directory is a special kind of file and the file structure is a tree of files,
some of which are directories \cite{Daley1965}. The user is considered to be
operating in one directory at any one time, called his `working directory'.
File names must only be unique with respect to the directory in which it
occurs. For every file and directory one can define a unique name by prepending
the chain of directory names required to reach the file from the root. This
chain is called \term{absolute path} (`tree name' in Multics). One can also
construct a relative path starting from a given working directory.  Pathes
require a special character as directory separator that must not occur in file
names. It should be noted that POSIX does not require a file system to form a
tree. The first Unix file system had the shape of a general directed graph
\cite{Ritchie1979}.  However the requirement to have unique pathes was more
important. For this reson other hierarchies but trees such as directed
(acyclic) graphs, or polytrees are forbidden by the file system or operating
system. Other restrictions can be put on the maximum depth directory nesting
and the maximum length of a path.

\subsubsection{Links}
\label{sec:symlinks}

In simple file systems each file is identified by its name. \term{Multics} and
\term{Unix} changed this one-to-one relationship by seperating files and
directory entries (hard links) and by adding pointers to file names (symbolic
links). Both kinds of links are included in POSIX and supported by most modern
file systems today.\footnote{Symbolic links were available in Multics from the
beginning but they were ported to Unix not until the Berkeley Fast File System
implemented them in the BSD-branch of Unix in 1983.}

A \Term{hard link} is a equipollent name of a file. Internally each file is identified
by a unique number (\term{inode} number in POSIX) that all names link to. In
reverse, the inode only stores a link count telling how many hard links point to
it. The file is only deleted if the last hard link is removed (or even later if
the file is held open by a running program). Usually all hard links to one file
must lie on the same physical disk. Moreover hard links are restricted to normal files
to avoid breaking the directory hierarchy. The operating system defines two
unchangeable exceptions: each directory contains a file named `\verb|.|' (dot) as
hard link to itself and a file named `\verb|..|' as hard link to the parent
directory (or to itself in case of the root directory).

\Term[symbolic link]{Symbolic links} are special types of files that consist of
a pointer to another file or directory. The pointer is stored literally as
relative or absolute path.  Symbolic links, in contrast to hard links, can 
span different file systems and point to non existing targets. For most
applications, the use of symbolic links is transparent: opening a symbolic link
opens the target file for reading or writing. \term{NTFS} supports similar
links named \term{junction points}.

Both hard links and symbolic links are unidirectional: There is no simple
lookup table to get all names a file is known under. One can think of
alternative link mechanisms but few are implemented in the file or operating
system. One example is the Linking file system (LiFS) proposed by
\textcite{Amens2005} that introduces arbitrary links between files. Each link
holds a set of attributes that express the nature of the relationship. The
containment of a file within a directory is simply one relationship among many
that can be expressed with these links. 

In addition to hard links and symbolic links, there are link types such as
`alias' in \term{Mac OS}, Windows shortcurts, desktop icons in KDE or GNOME.
These links cannot be used on the file system level but require additional
APIs. 

% TODO: see Gremlink (graph database)

\subsubsection{Attributes}

\Term{File attributes} contain additional metadata about, or associated with a 
file. POSIX defines a fixed set of attributes for file type (regular file, symbolic
link, or directory), file size, access permissions, and the times of creation,
last modification, and last access. The set and purpose of this so called regular
attributes is fixed and defined by the operating system. Their content and
effect cannot freely be choosen by the user, some attributes (such as the file
size) are even automatically derived from other properties. Therefore one must
carefully ponder whether a given attribute is part of the conceptual level or
only a technical artifact.

In addition to regular attributes, most file systems support \Term[extended
file attribute]{extented file attributes} that can be choosen by the user. A
file's extended attributes consist of a map between names and values. Both may
be arbitrary sequences of bytes or Unicode characters depending in the specific
system.  Typically the size and number of attribute names and values per file
is limited. A \term{fork} is a special kind of extended file attribute
introduced in the \term{Macintosh File System} (MFS) in 1984. Unlike other
attributes the fork can be a byte sequence of arbitrary size just like the file
content. Forks are available in Apple's successor file systems HFS/\term{HFS+}
and as \term{Alternate Data Streams} in Microsoft's \term{NTFS} (1993).

From a conceptual point of view the file content can be treated as one attribute
or fork of the file among others. In practice extended attributes are rarely
used beyond system applications because of limited support in user
interfaces. The \term{BeOS} file system supported typed attributes (string,
time, double, float, int, boolean, raw, and image) and indexing, searching, and
sorting by any attribute field similar to a databases \cite{Giampaolo1999}. A
query to all files with specific attribute properties could also be used as
``virtual directories'' similar to views in a database.


\subsection{Wrapping file systems} \label{sec:archivfiles} File system
instances can also be embedded in a single file of another file system. The
container file is called \Term{archive file} or \Term{disk image} depeding on
the main purpose. Archive files are used to package and transfer files that
otherwise may get corrupted or split up when directly copied from one system to
another. In addition one can apply compression and encryption to the set of
contained files. Two of the most used archive formats are \term{TAR}
\cite{GNUtar} and \term{ZIP} \cite{ZIP2007}. The archive format defines the
conceptual properties of its file system. For instance extended attributes or
forks are not always included and older versions of TAR imposed more rigid
restrictions on size and names of contained files \cite[section 8]{GNUtar}.  In
contrast to most file systems, files in a TAR archive have an order and are
permitted to have more than one member with the same name -- both cannot
losslessly be mapped to most other file systems. Properties of the ZIP format
have also changed, for instance to support plain UTF-8 file names since version
6.3.2 \cite{ZIP2007}. Other extensions of ZIP such as file comments and extra
fields cannot directly be mapped to other file systems without additional
agreements. Archive files are also used as wrappers for more specialized file
formats, for instance \term{OpenDocument} \cite{OpenDocument2012} and Java
Archives (\texttt{.jar}) are ZIP files, and Debian software packages
(\texttt{.deb}) are \texttt{.ar} archive files that each contains two TAR
archives (thus two file systems wrapped in a file system wrapped in a file).
Other file formats use non-standard archive file systems like the
\term{Compound Document format} which many Microsoft products are based
on\footnote{The Apache~POI project reverse-engineered and implemented this
format and named it POIFS (Poor Obfuscation Implementation File System).} %
% http://poi.apache.org/poifs/
% http://msdn.microsoft.com/en-us/library/ms693383(VS.85).aspx
A remarkable curiosity of compressed archives is the possibility to create an
infinite chain of archives. \textcite{Cox2010} created a ``self-reproducing zip
file''. It contains one file that is identical to the original. Each file
contains another ZIP file so there is always one level more inside. See
section~\ref{sec:collectionstopic} for more about the question whether this
file contains itself or not.  \label{p:zipfile}

Disk images store a file system's underlying raw stream of bytes in one file.
Having access to the storage media one can transform every file system into
an archive file.\footnote{On Unix one can copy a
whole disk partition with its file system to a file on another partition and
mount the resulting disk image this way:\\
\texttt{dd if=/dev/partition of=/mnt/otherpartition/image}\\
\texttt{mount -o loop image /mnt/image}
}
The program or location to mount a given disk image is also called
\term{virtual drive}. Virtual drives are common to abstract from storage
media, for instance when one emulates another computer system in a virtual
machine. Mounting specific archive files is also possible but less
common.\footnote{See \url{http://code.google.com/p/fuse-zip/} for a
file system that mounts ZIP files.} Another way of abstracting a file system
is to route all file system API calls through another API that wraps the
underlying file system -- this is how virtual file systems like 
GnomeFS/GVfs/gio\footnote{\url{http://library.gnome.org/devel/gio/}}
and KIO\footnote{\url{http://api.kde.org/4.x-api/kdelibs-apidocs/kio/html/}}
are implemented. % TODO: OLE could also be seen as VFS?

% I have some notes about Amazon~S3 which can both be seen as
% file system or as database system. This could be used as example.
% Or better show the POSIX model (?)

