\section{Conceptual modeling languages}
\label{sec:modelangs}

\begin{quotation}%
Essentially, all models are wrong, but some are useful.\\
\quotationsource \Person[George]{Box} (on statistical models)
% source: see http://en.wikiquote.org/wiki/George_Box
\end{quotation}

\index{data structure diagram|see{data model diagram}}

\noindent \Term{Conceptual modeling languages} \cite{Frederiks1997} or
\Term{semantic modeling languages} \cite{Hull1987,Peckham1988} are used to
formally capture an \term{universe of discourse} (see
section~\ref{sec:datamodeling} on data modeling).  They often come in a
graphical notation, known as \term{data model diagram} or \term{data structure
diagram}. Originally, such diagrams were introduced by \textcite{Bachman1969}
for logical schema design of network databases. They have since evolved to
many notations from simple labeled graphs to detailed graphical rules (see
section~\ref{sec:diagrams}).

The by far most influential conceptual modeling language, both to academia
\cite{Chen2007} and to practitioners \cite{Simsion2007},  is the
\Tacro{Entity-relationship model}{ERM}, presented in part~\ref{sec:erm}.
\acro{ERM} is rarely used in its original form but in form of various variants
and extensions. An extensive evolution of modelling languages has been
identified by \cite{Patig2006} and the plethora of modeling languages has even
been named `yama syndrome' (Yet Another Modeling Approach) \textcite{Oei1992}.
The amount of academic work on conceptual modeling languages notwithstanding,
modelers often ignore these languages and prefer to work directly in a
\term{data description language} (see section~\ref{sec:schemas}), as pointed
out by \textcite[p.  345]{Simsion2007}.  At the same time, modelers try to
express not only a logical model but a model of reality that only exist in our
minds, The reality as `territory' is easy to confuse with a conceptual model as
`map` and the conceptual model is easy to confuse with a logical model as `map
copy'.\footnote{The expression ``the map is not the territory'' is attributed
to \textcite{Korzybski1933} and applied to data modeling by \cite{Kent1978}.}
In the following the term `model' refers to a conceptual model, as symbolic
abstraction between mind and logical schema (see
figure~\ref{fig:simplifieddatamodeling} on page
\pageref{fig:simplifieddatamodeling}). Examples of conceptual models from the
cultural domain include \acro{FRBR} \cite{FRBR1998}, \acro{CIDOC-CRM}
\cite{Crofts2011}, and \acro{CHARM} \cite{GonzalezPerez2012}.

Beside \acro{ERM} (part~\ref{sec:erm}) this section presents fact-oriented
\tacro{Object Role Modeling}{ORM} (part~\ref{sec:orm}) and the \tacro{Unified
Modeling Language}{UML} (part~\ref{sec:uml}) as concrete examples of unique
conceptual modeling languages. Their common conceptual properties as base of
general modeling paradigms and patterns can further be identified by
meta-modeling and a look at the motivation of domain specific modeling
(part~\ref{sec:metamodeling}). Modeling languages for processes and dynamic
systems, such as \term{Petri Nets} and \tacro{Business Process Model and
Notation}{BPMN} are not included because of the focus on stable digital
documents.  

\subsection{Entity-relationship model}
\label{sec:erm}

The \Tacro{Entity-Relationship Model}{ERM} is known as the most influential
conceptual modeling language. It was originally proposed by \textcite{Chen1976}
to unify network databases and relational database views (see
section~\ref{sec:databases}). The basic components of an \acro{ERM} model are
\Term[entity types!in ERM]{entity types}, \Term{relationship types}, and
\Term{attributes}, together with rules how to connect them: relationship types
must connect at least two entity types and attributes must be connected to
entity types or relationship types. Strictly speaking, entities and
relationships in \acro{ERM} refer to concrete objects in the universe of
discourse. For instance, you can think of different books and authors as
entities and authorships as relations. Entities and relationships are then
modeled with entity types such as `book` and `author`, and relationship types
such as `authorship`. For simplicity the terms entity and relationship are
often used for entity types and relationship types and concrete objects referred
to as instances of entities or relationships.  Entity-Relationship models are
mainly expressed and explained by graphical notations.  A graphically
expressed model is called an \term{Entity Relationship Diagram}.
Figure~\ref{ex:ermodel} shows an \acro{ERM} diagram in Chen's notation.
Elements of other \acro{ERM} notation variants (\acro{IE} and \acro{Barker
Notation} together with \acro{ORM} and \acro{UML}) are included in
table~\ref{tab:ermnotations}. 

\begin{example}
\centering
\begin{tikzpicture}[orm,node distance=8mm,
  every edge/.style={line width=0.25mm,draw},
  every node/.style={line width=0.25mm}]
\node[draw,rectangle,minimum height=6mm] (book) {Book};
\node[draw,ellipse,below=of book] {Title} edge (book);
\node[draw,diamond,aspect=1.7,right=of book] (aut) {wrote} edge (book);
\node[right=0mm of book,anchor=north west] {N};
\node[right=0mm of book,anchor=south west] {work};
\node[draw,rectangle,minimum height=6mm,right=10mm of aut] (author) {Person} edge (aut);
\node[left=0mm of author,anchor=north east] {M};
\node[left=0mm of author,anchor=south east] {creator};
\node[draw,ellipse,below=of author] {Name} edge (author);
\node[draw,diamond,aspect=1.7,right=of author] (death) {died} edge (author);
\node[draw,ellipse,above right=of death] (year) {Year} edge (death);
\node[draw,rectangle,minimum height=6mm,below right=of death] (place) {Place} edge (death);
\node[right=0mm of author,anchor=north west] {N}; % (sic!)
\node[below right=0mm of death,xshift=4mm] {1};
\node[draw,ellipse,right=of place] {Name} edge (place);
\end{tikzpicture}
\caption{\acrostyle{ERM} diagram in Chen's notation plus attributes}
\label{ex:ermodel}
\end{example}

Since introduction of \acro{ERM}, a large number of variants have evolved
\cite{Patig2006}. These variants often add new features and come with different
graphical notations.  The diversity of \acro{ERM} variants and notations
requires to check carefully which conventions and semantics apply in a given
application.  As shown by \textcite{Hitchman1995}, many additional constructs
like subtypes, n-ary relationships, and naming both ends of a relationship are
rarely used and not well understood.  In its original form, \acro{ERM} is
rarely used in data modeling practice \cite[pp. 49, 345]{Simsion2007}. Widely
used variants and notations include \tacro{information engineering
notation}{IE}, also known as ``crow's foot notation''
\cite{Martin1985,Finkelstein1989} and \term{Barker Notation} \cite{Barker1990}.
Both add some additional constructs to \acro{ERM} but also limit relationship
types to binary associations only \cite[pp.  318ff.]{Halpin2008}. The full
variety of \acro{ERM} dialects between 1975 and 2003, has been analyzed in a
study on \acro{ERM} evolution: \textcite[p. 72ff]{Patig2006} identified 33
conceptual constructs in addition to labeled entity types, relationship types,
attributes and the basic rules how to connect them.  These constructs can be
grouped in constructs of the basic model structure, integrity conditions on
model instances, and constructs that are motivated by specific applications or
by specific domains. The following list summarizes basic some conceptual
constructs, derived from similar collections by \textcite{Patig2006} and
\textcite{Kent1983b}: 

%\caption{conceptual constructs of \acrostyle{ERM} variants}
%\label{tab:ermconstructs}

\label{ermconceptualconstructs}
\begin{description}

\item[entity types] group sets of entities to be modeled, for
  instance people in a {\ormtext Person} entity type.

\item[relationship types] must be connected to at least two 
  entity types.

\item[attributes] can be connected to entity types and to 
  relationship types. 

\item[attribute values] may explicitly be defined with data types or lists.

\item[multivalued attributes] can have multiple values 
  For instance a person may have more than one name. In addition,
  multiple values may be ordered or unsorted.

\item[attributes of attributes] may be allowed, for instance the
  percentage of a value or the language of a name.

\item[labels] uniquely identify and describe entity types, 
  relationship types, and attributes.

\item[roles] uniquely identify and describe connections. They
  are needed, in particular if there are multiple connections between
  two components (recursive or circular relations). % TODO: example!

\item[aggregations] treat selected relationship types as 
  entity types, so they can be connected to other relationship types.
  In \acro{ORM} this is known as `\term{objectification}'.

\item[conditions] limit the set of possible entity instance, 
  relationship instances, or attribute values, based on arbitrary 
  propositions. Frequent types of conditions are often expressed by 
  additional modeling constructs, such as the following:

\item[primary keys] mark selected attributes relationship types to have
  unique values/instances among all entity instances.

\item[inference rules] can define that parts of a model instance
  (entity types, relationships, attribute values etc.) can be inferred from 
  other parts. For instance the current age of a person could be inferred
  from its date of birth and the current date.

\item[frequency constraints] limit the total number of concrete entities or
  attributes of some type in a model instance.

\item[cardinality constraints] limit the minimum and/or maximum
  number of times that an entity can be connected to a relationship. In detail, 
  you can distinguish between simple participation cardinality and look-across 
  cardinality, which is only relevant to relationships with more than two entities.
  For instance a library user may only loan a maximum number of books (participation
  cardinality), but only another maximum number of books on each single day
  (look-across cardinality).

\item[optionality and mandatory constraints] define that specific
  connections are optional or required. These constraints are often confused
  with minimum cardinality constraints zero or one, but they can also be used
  independently. For instance a publication does not need to have a review
  (optionality),  but if it is reviewed, it must have at least two reviews
  (cardinality).

\item[exclusive constraints] mark connections as mutually exclusive.

%\term*{inheritance} % TODO: fuer index!
\item[inheritance] allows direct connections between a supertype 
  entity and a subtype entity, which then shares all attributes and relationship 
  connections with its supertype. For instance an {\ormtext Author} can be
  a subtype of a {\ormtext Person} (see example~\ref{ex:ormdiagram}). Simple 
  cases of subtyping can also be
  expressed by \term{euler diagram}s. Subtypes can have their own subtypes
  but all subtyping connections must form a \term{directed acyclic graph}.

\item[identity relationships] connect two or more entity types
  to state that their instances refer to the same objects in the 
  universe of discourse. For instance an author entity could be 
  identical with a translator entity.

\item[specialized types] group selected entity types, relationship
  types or attributes as being of a same kind. Specialized types can be
  domain-specific, for instance geographic entity types or causal relationship
  types, or more general. The most common specialized types are aggregation and
  composition relationships, both available in \acro{IE} and in \term{Barker
  Notation}.

\item[transactions] describe possible changes of model instances.

\item[temporal connections] distinguish structure and constraints of 
 the model at different times.

\item[multidimensionality] helps to highlight components of a schema
 as dimensions and facts for aggregation in data warehouses.

\item[uncertainty] can be introduced to mark selected parts of a model 
  (entities, relationships, attributes and connections) not known exactly.
\end{description}

As found by \textcite[345]{Simsion2007} ``the impact of the very substantial
amount of work on modeling languages appears to be minimal, with modelers
apparently preferring to work with the \acro{DBMS} language.'' If \acro{ERM} is
used, it is mostly used in a limited variant. The most important modification
is a limitation to binary relationship types, which are then drawn as simple
lines instead of diamonds. $N$-ary relationship types can be modified to entity
types with $N$ mandatory connections, as shown in example~\ref{ex:ermtrans}.
% TODO: figure X shows the 16 possible constraint patterns 
% for optionality and cardinality on relationships in IE (see Halpin2008, p320 and p311)
Together with roles, such simplification actually means removal of
relationships from \acro{ERM} as implemented in \term{Object role modeling}
(see part~\ref{sec:orm}). Attributes can also be replaced by entity types, as
suggested by \textcite{Kent1983b,Kent1984} and also shown in 
example~\ref{ex:ermtrans}, but most applications keep attributes for simplicity
and brevity. It should be noted that such transformations are often subject to
interpretation because they may result in different models (see 
example~\ref{ex:contrans} in section~\ref{sec:entcon}).

\begin{example}
\centering
\begin{tikzpicture}[orm,node distance=7mm,
  every edge/.style={line width=0.25mm,draw},
  every node/.style={line width=0.25mm},
  e/.style={draw,rectangle,minimum height=6mm},
  r/.style={draw,diamond,aspect=1.7},
  s/.style={draw,line width=0.15mm,dotted,->}
]
\node[e] (person) {Person};
\node[r,left=of person] (creator) {created} edge (person); 
\node[e,above=of creator] (authorship) {Authorship} edge (creator); 
\node[r,above=of authorship] (made) {made} edge (authorship); 
\node[e,above=of made] (book) {Book} edge (made); 
\node[below=0 of book,anchor=north west] {1};
\node[above=0 of authorship,anchor=south west] {1};
\node[below=0 of authorship,anchor=north west] {M};
\node[left=0 of person,anchor=south east] {N};
\node[above=0 of person.north east,anchor=south east] {1};
\node[below=0 of person.south east,anchor=north east] {1};

\node[r,below right=of person] (died) {died} edge (person);
\node[e,right=of died] (death) {Death} edge (died);
\node[r,below right=of death] (in) {in} edge (death);
\node[r,above right=of death] (at) {at} edge (death);
\node[e,right=of in] (year) {Year} edge (in);
\node[e,right=of at] (place) {Place} edge (at);
\node[left=0 of place,anchor=south east]{N};
\node[left=0 of year,anchor=south east]{N};
\node[left=0 of death,anchor=south east]{1};
\node[above=0 of death.north east,anchor=south east] {1};
\node[below=0 of death.south east,anchor=north east] {1};

\node[r,above right=of person] (phas) {named} edge (person);
\node[e,right=of phas] (pname) {PersonName} edge (phas);
\node[left=0 of pname,anchor=south east]{N};

\node[r,right=of book] (bhas) {titled} edge (book);
\node[e,right=of bhas] (title) {Title} edge (bhas);
\node[right=0 of book,anchor=south west]{1};
\node[left=0 of title,anchor=south east]{N};

\node[r,above=of place] (labeled) {labeled} edge (place);
\node[e,above=of labeled] (placename) {PlaceName} edge (labeled);
\node[below=0 of placename,anchor=north west] {N};
\node[above=0 of place,anchor=south west] {1};

\node[above right=of authorship] (s1) {relationship transformed to entity} edge[s] (authorship);
\draw[s] (s1) to (death);

\node[right=of title] (s2) {attribute transformed to entity} edge[s] (title);
\draw[s] (s2) to (year);
\draw[s] (s2) to (pname);
\draw[s] (s2) to (placename);

\end{tikzpicture}
\caption{\acrostyle{ERM} model transformed from example \ref{ex:ermodel}}
\label{ex:ermtrans}
\end{example}

Another limitation of most \acro{ERM} variants is less obvious: all entity
types are assumed to be disjoint, unless they are connected by \term{inheritance} or
by identity relationships.  For instance in example~\ref{ex:ermodel} and
\ref{ex:ermtrans}, an entity is {\em either} instance of {\ormtext Book},
{\ormtext Person}, or {\ormtext Place}. This implicit rule can cause problems with
entity types that are less easy to separate, such as {\ormtext Title}, {\ormtext PersonName},
and {\ormtext PlaceName} in example~\ref{ex:ermtrans}: here the disjointness
constraint becomes an artifact of the model, that is not present in the
universe discourse --- normally a book, a place, and a person can have the same
name without being the same object. The disjointness assumption of \acro{ERM}
can also be found in other conceptual modeling languages and in most schema
languages.\footnote{An exception are \acro{RDF} schema languages 
(section~\ref{sec:rdfschemas}), because of the
\term{Open World Assumption}: An \acro{RDF} entity (`resource' in
\acro{RDF} terminology) can be of multiple entity types (`classes'), unless 
an explicit disjointness constraint is enforced by a specific
ontology.} 


% \term{fact-oriented modeling} see ORM

\subsection{Object Role Modeling}
\label{sec:orm}

% G.M. Nijssen. Deemed NIAM, short for “Nijssen’s Information Analysis
% The purpose is to show representations of relationships instead of showing
% types of entities as relational table analogs.

% \cite{Kent1984}: Fact based data analysis and design

% NORMALIZATION: an elementary fact is an an
% elementary fact because it cannot be split into smaller statements collectively
% provide the same information.

% Good example: diagram, semantics, logical model
% http://www.ormfoundation.org/forums/storage/44/1476/Course01.png

\Tacro{Object-Role Modeling}{ORM} is a fact-oriented modeling language that
evolved from the \Tacro{Natural-language Information Analysis Method}{NIAM} by
\Person[Gerardus M.]{Nijssen} and \Person[Eckhard]{Falkenberg}
\cite{Nijssen1989,Falkenberg1976}. The current version (\acro{ORM2}) is mainly
based on works of \Person[Terry]{Halpin} and best described by
\textcite{Halpin2008}. Unlike \acro{ERM} and \acro{UML}, \acro{ORM} is build
from a linguistic basis using structured sentences in natural language as a
starting point. \acro{ORM} does not make use of the notion of attributes but
views the universe of discourse in terms of objects playing roles.  Objects are
classified in sets of \Term[entity types!in ORM]{entity types} and \Term[value
type (ORM)]{value types}. Entities can be any (possibly abstract) concepts and
they are references by values. For instance a person (entity) may be referenced
by its name (value). Modeling starts with factual examples that are split into
\Term[elementary fact (ORM)]{elementary facts} and translated into
\Term[predicate (ORM)]{predicates}.  This process will be exemplified in the
following.

The sentence `Frankenstein was written by \person[Mary]{Shelley}' is an
elementary fact because it cannot be split into smaller statements collectively
provide the same information. With background knowledge about its meaning can
be translated into `The book referred to by title `Frankenstein' was written by
the person referred to by name `Mary Shelley'{}'.  You can then infer first the
entity types {\ormtext Book} and {\ormtext Person}, second the value types
{\ormtext BookTitle} and {\ormtext PersonName}, and third the predicate
{\ormtext \ldots was written by\ldots}.  General prredicates in \acro{ORM} can
be named in two directions ({\ormtext \ldots was written by\ldots /\ldots
wrote\ldots}) and connect any positive number of objects. For instance in the
fact `Mary Shelley is dead' the predicate {\ormtext \ldots is dead} is unary
and in the fact `Mary Shelley died 1851 in London' the predicate
{\ormtext\ldots died\ldots in\ldots} is ternary. Each ``{\ormtext \ldots}''
slot of a predicate is called a \Term[role (in ORM)]{role}. Roles may be named
and the number of roles is the predicates \Term{arity} as in \term{predicate
logic}. 

\acro{ORM2} includes a detailed graphical notation to express models with
objects types, roles, and constraints (see figure~\ref{fig:orm2basics} for a
very simple example).  An object type is drawn as rectangle with rounded
corners containing its name.  Entity types use solid border lines and value
types use dashed border lines.  Predicates are drawn as sequence of
concatenated role boxes that are linked to object types by lines.  Role names
can be shown in square brackets and blue color next to a role box.
Figure~\ref{ex:ormdiagram} shows an \acro{ORM} diagram of the model derived
from example~\ref{ex:ermodel}. Value types that uniquely identify instances of
an entity type (in this example {\ormtext BookTitle}, {\ormtext PersonName},
and {\ormtext PlaceName}) are shown as \Term{reference mode} below the entity's
name. In addition the model introduces the {\ormtext Author} entity type which
is a subtype of the {\ormtext Person} entity type, so every author is a person.
Subtype connections are indicated by bold arrows that show the
\term{inheritance}. The diagram also contains some \Term[constraint
(ORM)]{constraints}, which are shown in magenta: a dot at the line connecting
the {\ormtext Author} with the {\ormtext creator} role depicts a
\Term{mandatory role constraint}. This kind of constraint demands that every
author must have written at least one book.  The arrow between the first role
of {\ormtext \ldots is dead} and of {\ormtext\ldots died\ldots in\ldots} shows
an \Term{external constraint}, in detail a \Term{subset constraint}. It states
that the set of people who died in a specific year and place must be a subset of
the set of people who are dead. To create a precise conceptual model, you must
carefully reveal such connections which may be obvious only to experts in the
universe of discourse.  The bar below the first role of the predicate
{\ormtext\ldots died\ldots in\ldots} is a \Term{uniqueness constraint}. It can
be read as ``each person died at most once''. On all predicates there is an
implicit uniqueness constraints that spans all roles combined.  For instance a
person is only dead once: if ``Mary Shelley died 1851 in London'' is an
elementary fact, it does not make sense to include it twice in the same model
population. This is also due to the interpretation of \acro{ORM} predicates as
factual statements in \term{predicate logic}.

A \Term{model population} (or \Term{model instance}) is a set of objects and
roles played by them, that fulfills all model's constraints. In terms of
ontology languages, a model is a \term{TBox} and a model population is a
\term{ABox}.  \acro{ORM} offers practical means of communicating via
unambiguous, controlled natural language language and examples of data by
\Term{verbalization} \cite{Halpin2004}. Given some basic language templates you
can even provide verbalizations in multiple languages, as implemented by
\cite{Jarrar2006}. This verbalization is similar to standardized verbalization
forms of fact-oriented \term{business rule} such as those based on the
\tacro{Semantics of Business Vocabulary and Business Rules}{SBVR}.\footnote{See
\url{http://www.rulespeak.com/} for an example. In brief, \acro{SBVR} was
influenced by fact-oriented modeling, but it lacks a graphical notation.}

\begin{example}
\centering
\begin{tikzpicture}[orm]
% entity type (one with reference mode, one without)
\entity (person) {Person\\(.name)};
\entity (author) [left=6mm of person] {Author};

% unary relationship with label
\unary (dead) [right=of person,yshift=7mm,label=\ldots is dead] {};

% binary relationship with labels in both directions
\binary (wrote) [left=13mm of author,
  label=\ldots is written by\ldots,
  label=below:\ormleft{\ldots wrote\ldots}] {};

% role names
\node[role name,anchor=north east] at (wrote.west) {[work]};
\node[role name,anchor=north west] at (wrote.east) {[creator]};
\plays (person) to (dead.west);

% mandatory role constraint
\entity (book) [left=10mm of wrote] {Book\\(.title)};
\plays[mandatory] (author) to (wrote);
\plays (wrote) to (book);

% ternary relationship with uniqueness constraint
\ternary (died) [right=of person,yshift=-7mm,unique=-1,
                 label=below:{\ldots died\ldots in\ldots}] {};
\plays (person) to (died.west)
   (died.east) to +(4mm,0) node[entity,anchor=west] (place) {Place\\(.name)};

% value type
\value (year) [above=of place.north west,anchor=south west] {Year};
\plays (died.north) to (year.west);

% subtyping 
\draw[subtype] (person) to (author);

% subset constraint
\draw[limits to] (died.one north) to (dead.south);
\node[constraint=subset] at ($(died.one north)!0.4!(dead.south)$) {};

\end{tikzpicture}
\caption{ORM diagram expressing a conceptual model}
\label{ex:ormdiagram}
\end{example}

Among the special \acro{ORM} features not included in
example~\ref{ex:ormdiagram}, there are ring constraints, frequency and value
constraints on entity types and roles, deontic rules, and
\term{objectification}.  Ring constraints can be applied to pairs of roles that
may be populated by the same entities. The simplest case if a binary
relationship with both roles played by the same entity type, but rings can also
occur on longer predicates and indirectly because of subtypes. There are 10
ring constraints (reflexive, irreflexive, purely reflexive, symmetric,
asymmetric, antisymmetric, transitive, intransitive, strongly Intransitive,
acyclic) with 26 legal combinations. Example~\ref{ex:ormore} {\ormtext a)}
shows a model in which a {\ormtext Person} can be {\ormtext child of} a
{\ormtext Person}.  The predicate is constraint as acyclic (left), because no
one can be its own child or ancestor. A frequency constraint (`$\le 2$') is
added to the {\ormtext parent} role to state that a person is child of at most
two people. An additional strongly intransitive ring constraint is given as
deontic (right, in blue). This means, if one person is child of another, there
should be no other chain of child-of-relationships between the two. By this
deontic rule, incest is forbidden, but still possible, while circular
ancestorship or more than two parents of one person are impossible.

A special feature of \acro{ORM} that is rarely found in other conceptual
modeling languages is \Term{objectification} or `nesting'. Objectification
allows instances of relationships to be treated as entities in their own right
\cite[ch. 10.5.]{Halpin2008}. In contrast to transformations of relationships
to entities in example~\ref{ex:ermtrans}, an objectified relationship may still
be used as relationship. In example~\ref{ex:ormore} {\ormtext b)} the
relationship {\ormtext wrote} is objectified as {\ormtext Writing}. In natural
language objectification is related to the activity of nominalization. For
instance the statement `\person[Mary]{Shelley} wrote Frankenstein' may be
nominalized as `\person[Mary]{Shelley}'s writing of Frankenstein'. The
interpretation of objectified facts, bears some difficulties: relationships one
the one hand represent possible propositions, which can either be true or
false: Shelley either has written Frankenstein or not.  Entities on the other
hand are states of affairs. It makes no sense to say that Shelley's writing of
Frankenstein is true or false, but you can make statements about this event,
for instance it started in summer 1816 in Geneva, it was not known when the
novel was first published anonymously in 1818, and it is described in other
books. For this reason a relationship and its objectification should be seen
as distinct objects connected by a 1:1 relationship. If you further analyze 
nominalization, different ontological types of objectification may be 
distinguished \cite{Moltmann2007}, for instance to differentiate statements
like `I know that Shelley wrote Frankenstein' and `I know the particular 
circumstances of Shelley's writing of Frankenstein'. General problems of
mapping between relationships and entities will be dealt with in 
section~\ref{sec:entcon}. 

% Reification in RDF or \term{objectification} in \acro{ORM}

\begin{example}
\centering
\begin{tikzpicture}[orm]

\entity (P) {Person};
\node[left=8mm of P] {a)};
\binary[below=of P] (r) {};
\node[below=4mm of r] {child of};
\plays (P) to (r.one north) (P) to (r.two north);
\node[constraint,below=0 of r.two south] {$\le$2};
\node[role name,right=0 of r]{[parent]};
\limits (r.north) to +(-1,0.4) node [constraint=acyclic]{};
\begin{scope}[deontic]
\limits (r.north) to +(1,0.4) node [constraint=strongly intransitive] (c) {};
\end{scope}

\begin{scope}[xshift=45mm,yshift=-2mm]
\entity (person) {Person};
\node[left=5mm of person,yshift=2mm] {b)};
\binary (wrote) [right=8mm of person, label=``Writing !''] {};
\entity (book) [right=8mm of wrote] {Book};
\plays (person) to (wrote);
\plays (wrote) to (book);
\entity at (wrote) (a) [minimum size=8mm,minimum width=11mm,fill=none,yshift=-1.5mm] {};
\node [below=0mm of wrote]{wrote};

\binary[below left=of a,label=below:\ormleft{took place in}] (d) {};
\plays (d.east) to (a);
\value[left=of d] {Year} edge[plays] (d); 

\binary[below right=of a,xshift=2.2mm,label=below:described in] (d) {};
\plays (d.west) to (a);
\plays (d.two north) to (book);

\end{scope}

\end{tikzpicture}
\caption{Additional constraints and features in \acrostyle{ORM}}
\label{ex:ormore}
\end{example}

% It should be noted that the model does only specify what is known
% if names are not unique than either the model or the information 
% to be modeled is wrong.
% the model cannot hold information about a person that died in a
% unknown place but known year

% information resource: ORM valuee type, non-information resources: ORM entity types.


\subsection{Unified Modeling language}
\label{sec:uml}

The \Tacro{Unified Modeling language}{UML} was developed in the 1990s as
modeling language for \term{object-oriented} software systems. It was
standardized by the \tacro{Object Management Group}{OMG}, published as
\acro{ISO} specification in 2005, and since extended to UML 2.4.1
\cite{OMG2011}.  Similar to \acro{CORBA}, \acro{ORM}'s other popular standard,
\acro{UML} at large is fairly complex and expressive.\footnote{Complexity and
standardization without reference implementation has been identified by
\textcite{Henning2006} as reasons why \acro{CORBA}'s failed. \acro{UML} in
contrast is quite popular, but its broad coverage maked it difficult to know,
to what in particular applications refer to with \acro{UML}.} \acro{UML}
provides graphical notations for several types of diagrams, that can also be
combined.  The major kinds of \acro{UML} diagrams are structure diagrams and
behavior diagrams. Behavior diagrams are not relevant to this thesis because of
the static nature of digital documents. Structure diagrams show the static
structure of a software system and its parts, so they can also be used to
depict data structures. Relevant structure diagram types for modeling in
\acro{UML} are:

\begin{description}
\item[class diagrams] describe the structure of a system in terms of
  \term{object orientation}: \Term[class!in UML]{classes} correspond to
  \acro{ERM} \term{entity type}s and \Term{objects} correspond to entities. 
  \acro{UML} classes can have attributes, which may have datatypes, and 
  cardinality constraints, among other properties.
  Classes may further be connected by 
  \term{inheritance}, dependency, composition, aggregation, and general binary
  relationships (called \term[associations!in UML]{associations}).
  \acro{UML} objects can either be simple class instances or so called 
  (static) \term[classifier!UML]{classifiers}, which are shared among all 
  class instances. An instance of a model, which is depicted by an \acro{UML}
  class diagram, can have multiple instances per class 
  but only one classifier.

\item[object diagrams] show (partial) model instances by depicting concrete 
  objects with their class memberships and attribute values.

\item[component diagrams] and \textbf{package diagrams} group parts of an 
 \acro{UML} diagram, to better abstract from distinct parts and layers.

\item[deployment diagrams] show how parts of a software system are located
  on different computers and other resources, including storage.

\item[profile diagrams] can define extensions of \acro{UML} in form of 
  specialized classes or other constructs. These extensions are called 
  stereotypes and they may introduce their own graphical notation. Most
  applications of \acro{UML} stereotypes can also be replaced by 
  \term{inheritance} between normal classes. To some degree profile diagrams 
  allow \term{metamodeling}, because they specify the way that other models
  can be expressed.

\item[composite structure diagrams] show the internal structure of a class or
  another component. The internal structure may imply the existence of other
  classes and relationships, but the meaning of this diagram type is not
  well understood among engineers \cite{Oliver2006}.

\end{description}

With class diagrams, \acro{UML} can be seen as a variant of \acro{ERM},
enriched with other diagram types and metamodeling. As suggested by the
attribute `unified', \acro{UML} in theory subsumes other modeling languages and
provides a tool to exchange conceptual models. In practice, however, support
and interpretation of \acro{UML}`s exchange format \tacro{XML Metadata
Interchange}{XMI} varies among tools. In addition, the semantics of \acro{UML}
constructs differs among users, as described by \textcite{Oliver2006} for
composite structure diagrams. Typical misues and wrong expectations of
\acro{UML} have been collected by \textcite{Bell2004}. A general problem is the
primary use of \acro{UML} for describing logical and physical data models which
can directly be transformed to software. Conceptual concepts in contrast, must
rather match a specific domain, independent from its technical implementation.

Example~\ref{ex:uml} shows an \acro{UML} class diagram that depicts a
conceptual model similar to the \acro{ERM} and \acro{UML} models above. The
ternary relationship between {\ormtext Person}, {\ormtext Year}, and {\ormtext
Place} has been replaced by a {\ormtext DeathEvent} entity. The model makes use
of attributes and data types ({\ormtext string} and {\ormtext year}). n-ary
relationships and additional entities connected to binary relationships are
also supported by \acro{UML} but not shown in example~\ref{ex:uml}.

\begin{example}
\centering
\begin{tikzpicture}[orm,
  every entity/.style={rectangle split,rectangle split parts=2,rounded corners=0},
  every edge/.style={line width=0.25mm,draw},
  every node/.style={line width=0.25mm}]
  
  \entity (person)
    {Person\nodepart{second}Name: string};

  \entity[right=30mm of person] (book)
    {Book\nodepart{second}Title: string} edge (person);
 
  \draw (person) to node[yshift=2mm] (x) {wrote} (book);

  \node[right=0 of person,yshift=2mm,anchor=west] {1..*};
  \node[left=0 of book,yshift=2mm,anchor=east] {0..*};

  \entity[below=10mm of person] (event)
    {DeathEvent\nodepart{second}Year: year} edge (person);

  \node[below=0 of person,anchor=north west] {1};
  \node[above=0 of event,anchor=south west] {0..1};

  \draw (person) to node[anchor=east] {died} (event);

  \entity[right=31mm of event] (place)
    {Place\nodepart{second}Name: string} edge (event);

  \draw (event) to node[yshift=2mm] {occurred in} (place);

  \node[right=0 of event,yshift=2mm,anchor=west] {0..*};
  \node[left=0 of place,yshift=2mm,anchor=east] {1};

\end{tikzpicture}
\caption{\acrostyle{UML} diagram with a model similar to example
\ref{ex:ermodel} and \ref{ex:ormdiagram}}
\label{ex:uml}
\end{example}


\subsection{Domain specific modeling and metamodeling}
\label{sec:metamodeling}

In addition to generalized conceptual modeling languages, such as \acro{ERM},
\acro{ORM}, and \acro{UML}, the idea of \Tacro{Domain Specific Modeling}{DSM}
and \Tacro{Domain Specific Languages}{DSL} have recently gained popularity
\cite{Kelly2008}. In short, a domain-specific modeling language is a custom
formalism for a specific domain, for instance the domain of mobile applications
or the automotive industry. Each language is designed to model an universe of
discourse within the specific domain. A \acro{DSL} consist of a set of language
concepts and their rules, together with a graphical notation. For instance a
business modeling language could consist of customer types, contract types and
service types, instead of general \term{entity type}s and \term{relationship
type}s. \acro{DSM} and \acro{DSL} have been proven useful in software
engineering especially to bridge the gap between domain experts and software
architects and to maintain changing models \cite{Cao2009}. In addition to
conceptual modeling, specialized modeling languages are also used for automatic
generation of software systems. For this reason the boundaries between domain
specific modeling languages, domain specific \term{schema language}s, and
domain specific programming languages are fluid. The lack of a clear separation
increases the existing danger of confusing real world models and software
models \cite{Genova2005}.

Domain specific modeling languages have a history in \tacro{Computer-aided
software engineering}{CASE} tools with the \tacro{Problem Statement
Language/Problem Statement Analyzer}{PSL/PSA} by \textcite{Teichroew1977} as
first instance. To some degree you can also use generalized modeling languages
that allow some customization, for for instance \acro{UML} with its
\term{profile diagram}s. Most \acro{DSL}, however, are created with specialized
\acro{DSM} tools such as \term{MetaEdit+} \cite{Kelly2008} and 
\acro{DSL} tools of \acro{Microsoft Visual Studio} \cite{Cook2007}.
The former has its origin in a research project at University of Jyv\"askyl\"a
which laid the theoretical and practical foundation of \Term{metamodeling}.

The term \Term{metamodel} first refers to any specialized \acro{DSL} which can
be used to create concrete models in its domain. The task of creating
specialized modeling languages, is then called metamodeling. Each \acro{DSM}
tool provides its own (graphical) language for metamodeling; this metamodeling
language is based on a meta-metamodel which defines how metamodels can be
expressed. Several meta-metamodels are compared by \textcite{Kern2011}, among
them \term{GOPRR} from MetaEdit \cite{Kelly2008}, the metamodel from Microsoft
DSL tools \cite{Cook2007}, and \term{Ecore} from Eclipse Modeling Framework
\cite{Steinberg2009}. The list of metamodeling concepts from this comparison
is similar to the list of conceptual \acro{ERM} constructs in
part~\ref{sec:erm}. There are \term{entity type}s (also refered to as
\term[object type|see{entity type}]{object type}s) and \term{relationship
type}s (possibly limited to binary relationships), \term{attribute}s,
\term{role}s, and \term{inheritance}, which each can be subject to several
constraints and extensions. Additional concept in meta-modeling include
entity-sets and composition, \term{port types}s that further define how
entities can be connected by relationships, grouping of metamodel elements, and
model types to group and refer to models with common properties.  An example of
a modeling language specified by a metamodeling is \acro{ConML}
\cite{GonzalezPerez2012c}, which was specified using \acro{UML}. \acro{ConML}
on its part supports creation of conceptual models such as \acro{CHARM} for
cultural heritage, including aspects such as subjectivity and temporality.
%(see figure~\ref{fig:conml} at page~\pageref{fig:conml} for an illustration).

Several independent attempts have been made to unify conceptual modeling
languages by mathematical descriptions, for instance with \term{category
theory} \textcite{Frederiks1997} and with \term{description logics}
\cite{Keet2008b}. The practical outcome of this (meta)-metamodeling, however,
is questionable as it comes with more complexity and less readability.  This
contradicts the original purpose of conceptual modeling as translating between
domain experts and programmers. Alternative uses of modeling languages for
\term{knowledge representation} and \term{schema} design may better benefit
from mathematical metamodels, but both veers away from the `territory' of
reality, which is not build from entities, relationships, attributes, and
roles, but from experience and assumptions.  A general limitation of conceptual
data modeling is its foundation on crisp set theory, in contrast to the fuzzy
nature of natural language. Even if conceptual language is based on language
(like \acro{ORM}), it only uses traditional logic statements with sets of
disjoint objects. And even an exact metamodel with clearly defined semantics
does not ensure that the meaning of a model is equal to all participants: As
shown for both \acro{ERM} \textcite{Hitchman1995} and \acro{UML}
\cite{Oliver2006} the ``semantics [of UML constructs] are often based on the
engineer’s expectations and {\em perceived meaning} rather than on the actual,
intended semantics''.  Nevertheless conceptual modeling is practiced, either
explicit with conceptual modeling languages or implicit by directly creating
logical schemas. The analysis of important modeling languages and metamodeling
principles in this section has revealed some common constructs which lead to
more concrete patterns in chapter~\ref{ch:patterns}.

% Not included: inverse
% GOPRR: properties with data types with properties
% \cite{Parr2010}: parsing, patterns

\begin{table}
\centering
\begin{tabular}{lllll}
\textbf{Notation} & \textbf{IE} & \textbf{Barker} & \textbf{ORM} & \textbf{UML} \\
\hline
zero or more 
& % IE
\begin{tikzpicture}[erm]
\node[ermnode] (p) at (0,-10mm) {E};
\draw (p.east) to +(6mm,0) +(0mm,1mm) to +(2mm,0mm) to +(0mm,-1mm);
\draw[fill=white] (p.east) +(3mm,0) circle (1mm);
\end{tikzpicture}
& % Barker
\begin{tikzpicture}[erm]
\node[ermnode] (p) {E}; \draw[dotted] (p.east) to +(6mm,0);
\draw (p.east) +(0mm,1mm) to +(2mm,0mm) to +(0mm,-1mm);
\draw (p.east) to +(2mm,0);
\end{tikzpicture}
& % ORM 
\begin{tikzpicture}[orm]
\entity (p) {E}; \plays (p.west) to +(-5mm,0) node[role] {};
\end{tikzpicture}
& % UML
\begin{tikzpicture}[erm]
\node[ermnode] (p) {E}; \draw (p.east) to +(6mm,0); \node[anchor=south west] at (p.east) {0..*};
\end{tikzpicture}

\\ \hline
one or more 
& % IE 
\begin{tikzpicture}[erm]
\node[ermnode] (p) {E};
\draw (p.east) to +(6mm,0) +(0mm,1mm) to +(2mm,0mm) to +(0mm,-1mm) +(2mm,1mm) to +(2mm,-1mm);
\end{tikzpicture}
& % Barker
\begin{tikzpicture}[erm]
\node[ermnode] (p) {E}; \draw (p.east) to +(6mm,0);
\draw (p.east) +(0mm,1mm) to +(2mm,0mm) to +(0mm,-1mm);
\end{tikzpicture}
& % ORM
\begin{tikzpicture}[orm]
\entity (p) {E}; \plays[mandatory] (p.west) to +(-5mm,0) node[role] {};
\end{tikzpicture}
& % UML
\begin{tikzpicture}[erm]
\node[ermnode] (p) {E}; \draw (p.east) to +(6mm,0); \node[anchor=south west] at (p.east) {1..*};
\end{tikzpicture}

\\ \hline
zero or one 
& % IE 
\begin{tikzpicture}[erm]
\node[ermnode] (p) {E};
\draw[fill=white] (p.east) to +(6mm,0) +(4mm,1mm) to +(4mm,-1mm);
\draw[fill=white] (p.east) +(2mm,0) circle (1mm);
\end{tikzpicture}
& % Barker
\begin{tikzpicture}[erm]
\node[ermnode] (p) {E}; \draw[dotted] (p.east) to +(6mm,0);
\end{tikzpicture}
& % ORM
\begin{tikzpicture}[orm]
\entity (p) {E}; \plays (p.west) to +(-5mm,0) node[role,unique=1] {};
\end{tikzpicture}
& % UML
\begin{tikzpicture}[erm]
\node[ermnode] (p) {E}; \draw (p.east) to +(6mm,0); \node[anchor=south west] at (p.east) {0..1};
\end{tikzpicture}

\\ \hline
one 
& % IE 
\begin{tikzpicture}[erm]
\node[ermnode] (p) {E};
\draw[fill=white] (p.east) to +(6mm,0) +(1mm,1mm) to +(1mm,-1mm) +(2mm,1mm) to +(2mm,-1mm);
\end{tikzpicture}
& % Barker
\begin{tikzpicture}[erm]
\node[ermnode] (p) {E}; \draw (p.east) to +(6mm,0);
\end{tikzpicture}
& % ORM
\begin{tikzpicture}[orm]
\entity (p) {E}; \plays[mandatory] (p.west) to +(-5mm,0) node[role,unique=1] {};
\end{tikzpicture}
& % UML
\begin{tikzpicture}[erm]
\node[ermnode] (p) {E}; \draw (p.east) to +(6mm,0); \node[anchor=south west] at (p.east) {1};
\end{tikzpicture}

\\ \hline
roles and ranges
& % IE 
%\tikz\node{n/a};
& % Barker
%\tikz\node{n/a};
& % ORM
\begin{tikzpicture}[orm]
\entity (p) {E}; \plays (p.west) to +(-8mm,0);
\node[anchor=south east,constraint] at (p.west) {n..m};
\node[anchor=north east,role name] at (p.west) {[role]};
\end{tikzpicture}
& % UML
\begin{tikzpicture}[erm]
\node[ermnode] (p) {E}; \draw (p.east) to +(6mm,0); 
\node[anchor=south west] at (p.east) {n..m};
\node[anchor=north west] at (p.east) {role};
\end{tikzpicture}

\\ \hline
attribute 
& % IE 
%\tikz\node{n/a};
& % Barker
\tikz[orm]\node{Name: Type};
& % ORM
%\tikz\node{n/a};
& % UML
\tikz[orm]\node{Name: Type};

\\ \hline
primary key 
& % IE 
%\tikz\node{n/a};
& % Barker
\tikz\node[orm]{\# Name};
& % ORM
\tikz[orm]\node{(Name)};
& % UML
\tikz[orm]\node{Name \textless\textless{}PK\textgreater\textgreater};

\\ \hline
inheritance 
& % IE
\begin{tikzpicture}[erm]
\path[use as bounding box] (3mm,5mm) rectangle (16mm,-4mm);
\node[ermnode,minimum width=13mm,minimum height=8mm] at (4mm,0) {}; 
\node[ermnode] at (6.5mm,0) {F}; \node{E};
\end{tikzpicture}
& % Barker
\begin{tikzpicture}[erm]
\path[use as bounding box] (3mm,5mm) rectangle (16mm,-4mm);
\node[ermnode,minimum width=13mm,minimum height=8mm] at (4mm,0) {}; 
\node[ermnode] at (6.5mm,0) {F}; \node{E};
\end{tikzpicture}
& % ORM
\begin{tikzpicture}[orm]
\entity (p) {E};
\draw[subtype] (p.east) to +(8mm,0);
\end{tikzpicture}
& % UML
\begin{tikzpicture}[erm]
\node[ermnode] (p) {E};
\draw (p.east) to +(8mm,0); 
\draw [fill=white] (p.east) to +(3mm,1mm) to +(3mm,-1mm) to (p.east);
\end{tikzpicture}

\\ \hline
or 
& % IE 
\begin{tikzpicture}[erm]
\path[use as bounding box] (0mm,-5mm) rectangle (12mm,-5mm);
\draw (0,0) to (4mm,0);
\draw (4mm,0) |- (12mm,4mm) (4mm,0) |- (12mm,-4mm);
\draw[fill=white] (4mm,0) circle (1mm);
\end{tikzpicture}
& % Barker
%\tikz\node{n/a};
& % ORM
\begin{tikzpicture}[orm]
\path[use as bounding box] (0mm,-1mm) rectangle (14mm,9mm);
\plays (0,0) to (14mm,0) (0,8mm) to (14mm,8mm);
\limits (7mm,0) to node[constraint=mandatory] {} (7mm,8mm);
\end{tikzpicture}
& % UML
\begin{tikzpicture}[erm]
\path[use as bounding box] (0mm,-5mm) rectangle (14mm,-5mm);
\draw (0,0) to (4mm,0);
\draw (4mm,0) |- (14mm,4mm) (4mm,0) |- (14mm,-4mm);
\draw[dotted] (6mm,4mm) to node[anchor=west] {\{or\}} (6mm,-4mm);
\end{tikzpicture}

\\ \hline
xor 
& % IE 
\begin{tikzpicture}[erm]
\path[use as bounding box] (0mm,-5mm) rectangle (12mm,-5mm);
\draw (0,0) to (4mm,0);
\draw (4mm,0) |- (12mm,4mm) (4mm,0) |- (12mm,-4mm);
\fill (4mm,0) circle (1mm);
\end{tikzpicture}
& % Barker
\begin{tikzpicture}[erm]
\path[use as bounding box] (0mm,-5mm) rectangle (14mm,-5mm);
\draw (0,0) to (4mm,0);
\draw (4mm,0) |- (14mm,3mm) (4mm,0) |- (14mm,-3mm);
\draw (8mm,-4mm) to[out=110,in=-110] (8mm,4mm);
\end{tikzpicture}
& % ORM
\begin{tikzpicture}[orm]
\path[use as bounding box] (0mm,-1mm) rectangle (14mm,9mm);
\plays (0,0) to (14mm,0) (0,8mm) to (14mm,8mm);
\limits (7mm,0) to node[constraint=xor] {} (7mm,8mm);
\end{tikzpicture}
& % UML
\begin{tikzpicture}[erm]
\path[use as bounding box] (0mm,-5mm) rectangle (14mm,-5mm);
\draw (0,0) to (4mm,0);
\draw (4mm,0) |- (14mm,4mm) (4mm,0) |- (14mm,-4mm);
\draw[dotted] (6mm,4mm) to node[anchor=west] {\{xor\}} (6mm,-4mm);
\end{tikzpicture}
\\
\hline
\end{tabular}
\caption{Some conceptual modeling notation variants}
\label{tab:ermnotations}
\end{table}
% see also http://www.agiledata.org/essays/dataModeling101.html
% Not included here:
%   association labels  (both directions)
%   aggregation, composition


